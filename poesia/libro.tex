%definisco la classe del documento
\documentclass[11pt]{book}
 
%predispongo il foglio per la stampa in formato a5
\usepackage[a5paper]{geometry}
 
%includo i pacchetti per la lingua
\usepackage[T1]{fontenc}
\usepackage[utf8]{inputenc}
\usepackage[italian]{babel}
 
%includo il pacchetto specifico per scrivere le poesie
\usepackage{verse}

%includo i pacchetti per la grafica come l'inserimento di immagini e testo colorato
\usepackage{graphicx,xcolor,subfig}
%in questa cartella metterò tutte le immagini che devo includere nel file
\graphicspath{{immagini/}}
 
%permette di togliere la numerazione alle pagine (volendo potete anche toglierlo)
\makeatletter
\let\ps@plain\ps@empty
\ps@empty
\makeatother 
 
%definisco il comando che mi permette di mettere le informazioni sulla poesia come 
%la data in cui è stata scritta e il luogo
\newcommand{\attrib}[1]{\nopagebreak{\raggedleft\footnotesize #1\par}}
 
% definisco l'ambiente che si preoccuperà di gestire la poesia
\newcommand{\poesia}[3]{\vspace*{\fill}%
\poemtitle{#1}%
\begin{verse}%
#2%
\end{verse}%
\attrib{#3}
\vspace*{\fill}\clearpage
}

\begin{document}
 
\clearpage
%predispongo la pagina per la copertina
\begin{titlepage}
 
\begin{figure}[htbp]
      \centering
      \includegraphics[scale=.3]{copertina} % qui ci va il nome del fle della copertina
      %il valore scale=0.3 è da modificare a seconda della dimensione dell'immagine
\end{figure}
 
\begin{center}
\begin{LARGE}
\textsc{\textbf{Titolo}}
\end{LARGE}
 
\begin{flushleft}
\vspace*{1 cm}\hspace{3.0 cm}\LARGE Autore\\
\end{flushleft}
 
\end{center}
\end{titlepage}
 
%pagina in cui mettere ringraziamenti, motivazioni, etc
\section*{Intro}
\newpage
 
%il titolo della poesia deve avere tutte le caratteristiche del titolo delle sezioni
\renewcommand{\poemtoc}{section}
 
%utilizzo l'ambiente poesia che ho creato prima
\poesia{Titolo}
{Primo verso prima strofa\\
secondo verso prima strofa\\
terzo verso prima strofa
 
primo verso seconda strofa\\
secondo verso seconda strofa\\
terzo verso seconda strofa\\
}{Luogo, data}
 
\end{document}