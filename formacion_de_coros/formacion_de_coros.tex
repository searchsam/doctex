% %%% Definimos el estilo del documento
\documentclass[letterpaper, 12pt]{book}

% Utilizamos un paquete para gestionar acentos y e\~nes
\usepackage[spanish]{babel}
\usepackage[utf8]{inputenc}
\usepackage[breaklinks=true]{hyperref}

% Definimos TÍTULO
\title{\Huge{\textbf{Formaci\'on de Coros}}}
\author{\Large{Fray Nesol Medina O.C.D.}}
\date{\large{2021}}

% %%% Empieza el documento
\begin{document}
    % Generar título
    \maketitle
    % Generar indice
    \tableofcontents
    
    \part {INTRODUCCI\'ON A LA LITURGIA}
    
    \chapter{CONCEPTO DE LITURG\'IA}
    La palabra \textbf{Liturgia} viene del griego \textit{\underline{leitourgia}} y quiere decir servicio p\'ublico, generalmente ofrecido por un individuo a la comunidad. En la tradici\'on cristiana significa <<\textit{Obra del pueblo para adorar a Dios}>>. Hoy se usa para designar todo el conjunto de la oraci\'on pú\'ublica de la Iglesia y de la celebraci\'on sacramental.
    
    \section{Algunos conceptos del Concilio Vaticano II}
    \textit{``La Liturgia es el \underline{ejercicio del sacerdocio de Jesucristo}. En ella, los signos sensibles significan y cada uno a su manera realizan la santificación del hombre, y as\'i el Cuerpo M\'istico de Jesucristo, es decir, la Cabeza y sus miembros, ejerce el culto p\'iblico \'integro.\\
    En consecuencia, toda celebración lit\'urgica, por ser \underline{obra de Cristo sacerdote y} \underline{de su Cuerpo}, que es la Iglesia, \underline{es acción sagrada por excelencia}, cuya eficacia, con el mismo t\'itulo y en el mismo grado, no la iguala ninguna otra acci\'on de la Iglesia''.}
    
    \section{De esto podemos concluir que la Liturgia es:}
    \begin{enumerate}
        \item Es el ejercicio del sacerdocio de Cristo. Es decir, en la Liturgia, Cristo act\'ua como sacerdote, ofreci\'endose al Padre, para la salvaci\'on de los hombres.
        \item Los signos sensibles realizan la santificaci\'on de los hombres en lo que quieren decir. Por ejemplo, el agua en el \textbf{Bautismo} significa y realiza la purificaci\'on y es principio de vida, el pan en la \textbf{Eucarist\'ia} alimenta el esp\'iritu del hombre.
        \item En la acci\'on lit\'urgica, Cristo y los cristianos, que forman el Cuerpo M\'istico, ejercen el culto p\'ublico.
        \item Es la \underline{acci\'on sagrada por excelencia}, que ninguna oraci\'on o acci\'on humana puede igualar por ser obra de Cristo y de toda su Iglesia y no de una persona o un grupo.
        \item \textit{\textbf{La Liturgia es la cumbre a la que tiende la actividad de la Iglesia y, al mismo tiempo, la fuente de donde mana toda su fuerza.}}
    \end{enumerate}
    
    \chapter{EL A\~NO LIT\'URGICO}
    El ritmo semanal con el domingo como d\'ia central es el primer eslab\'on de la cadena del A\~no Lit\'urgico. Con el tiempo, un domingo destac\'o sobre los dem\'as: fue el domingo de Pascua. En rigor, todos los domingos del a\~no son domingos pascuales, pascua semanal. La Iglesia desde el Siglo V ha impuesto la obligaci\'on de santificar el d\'ia del Se\~nor, d\'ia que comienza en las V\'isperas, o sea, en la tarde anterior\footnote{La tarde del s\'abado anterior.} siguiendo la costumbre jud\'ia de contar los d\'ias. Tambi\'en las solemnidades comienzan en la V\'ispera. Por este motivo la misa vespertina del s\'abado \textit{``vale''} para cumplir el precepto dominical porque en rigor ya es domingo.\newline
    
    El domingo pascual, n\'ucleo del A\~no Lit\'urgico, qued\'o fijado por el Concilio de Nicea reunido el a\~no 325 que dispuso que la Pascua se celebrase el domingo posterior a la primera luna llena que haya despu\'es del 22 de marzo. Por este motivo, la Pascua de Resurrecci\'on es fiesta variable, ya que depende de la luna y necesariamente deber\'a oscilar entre el 22 de marzo y el 25 de abril. Una vez fijado el domingo pascual de cada a\~no se establecen los dem\'as tiempos movibles y sus fiestas: El Tiempo Pascual\footnote{Cincuenta d\'ias posteriores a la Pascua del Se\~nor.} y el Tiempo de Cuaresma\footnote{Cuarenta d\'ias anteriores a la Senama Santa.} adem\'as de las solemnidades que dependen de la fecha de Pentecost\'es.\footnote{Sant\'isima Trinidad, Corpus Christi, Sagrado Coraz\'on.}\newline
    
    El A\~no Lit\'urgico puede decirse que se compone de \underline{tiempos ``fuertes''}\footnote{Adviento, Navidad, Cuaresma y Pascua.} en los cuales se celebra un misterio concreto de la historia de la Salvaci\'on y otro tiempo llamado \underline{Tiempo Ordinario} en el cual no se celebra ning\'un aspecto concreto sino m\'as bien el mismo misterio de Cristo en su plenitud, especialmente en los domingos. Este Tiempo Ordinario transcurre partido y dura treinta y tres o treinta y cuatro semanas.
    
    \section{Tiempo de Adviento}
    El A\~no Lit\'urgico comienza en las v\'isperas del primer Domingo de Adviento, que es siempre el domingo m\'as cercano al d\'ia 30 de Noviembre, festividad de San Andr\'es. Dura cuatro semanas con sus respectivos domingos.
    
    \section{Tiempo de Navidad}
    Abarca desde el 25 de diciembre hasta el domingo posterior a la Epifan\'ia (\textit{6 de Enero}). Ese domingo celebramos el bautismo del Se\~nor.
    
    \section{Tiempo Ordinario:}
        \subsection{Primera Parte}
        Abarca desde el Lunes posterior a la fiesta del Bautismo del Se\~nor hasta el Martes anterior al Mi\'ercoles de Ceniza.
        
    \section{Tiempo de Cuarema}
    La Cuaresma, tiempo de preparaci\'on para la Pascua de Cristo, es un tiempo claramente penitencial. Incluye cuarenta d\'ias de penitencia, excluyendo los cinco Domingos de Cuaresma y el de Ramos\footnote{El Domingo siempre es d\'ia festivo.} y a\~nadiendo los d\'ias del Viernes y S\'abado Santo, ya en pleno Triduo Pascual. En sentido estricto, la Cuaresma abarca desde el Mi\'ercoles de Ceniza hasta la misa vespertina de la Cena del Se\~nor del Jueves Santo.
    
    \section{Semana Santa}
    Es la semana que abarca desde el Domingo de Ramos en la Pasi\'on del Se\~nor hasta la Vigilia Pascual del S\'abado Santo. Incluye al Triduo Pascual, que comienza con la Misa vespertina en la Cena del Se\~nor, del Jueves Santo y se prolonga Viernes, S\'abado Santo y el Domingo de Resurrecci\'on. Triduo del Se\~nor muerto, enterrado y resucitado.
    
    \section{Tiempo Pascual}
    Abarca los cincuenta d\'ias posteriores a la Pascua de Resurrecci\'on\footnote{Cincuentena pascual.}, incluyendo el domingo pascual, y se distinguen tres per\'iodos:
    
    \begin{itemize}
        \item \textbf{Octava de Pascua}. Son los ocho d\'ias posteriores y deben considerarse como un solo d\'ia festivo. Termina en las V\'isperas del II Domingo de Pascua.
        \item \textbf{Tiempo Pascual} hasta la \textbf{Ascensi\'on}.
        \item \textbf{Tiempo Pascual} despu\'es de la \textbf{Ascensión}.
    \end{itemize}
    
    El \underline{Domingo de Pentecost\'es}, que se celebra a los cincuenta d\'ias de Pascua, es el colof\'on del ciclo pascual, no debe pues considerarse como una nueva Pascua.
    
    \section{Tiempo Ordinario:}
        \subsection{Segunda Parte}
        Abarca desde el Lunes posterior a Pentecost\'es hasta las V\'isperas del primer domingo de Adviento. El domingo anterior al primero de Adviento, \'ultimo del A\~no Lit\'urgico, celebramos la solemnidad de Cristo Rey.\newline

    Los d\'ias que no son domingos de cualquier tiempo se llaman ferias. Seg\'un la costumbre latina, el lunes recibe el nombre de \textit{``feria segunda''} y as\'i sucesivamente hasta la feria sexta\footnote{D\'ia Viernes.}. El s\'abado tiene su nombre propio heredado de los jud\'ios\footnote{Sabbat que significa descanso.}. El dies dom\'inica\footnote{En griego \textit{kyriak\'e emera}.}, es el domingo, el día del Se\~nor. Ese d\'ia fue el de la Resurrecci\'on de Cristo. As\'i nos lo cuentan los evangelistas\footnote{Mateo 28, 1-7; Marcos 16, 1-8; Lucas 24, 1-12; Juan 20, 1-10.}.\newline
    
    Es tambi\'en ese d\'ia el elegido por Jes\'us Resucitado para aparecerse a sus disc\'ipulos en el camino de Ema\'us y en el Cen\'aculo. Tambi\'en al domingo se la ha llamado el ``Octavo D\'ia'' por los Padres de la Iglesia, haciendo referencia al tiempo nuevo que abre la Resurrecci\'on y en otro sentido se le ha llamado el ``Tercer D\'ia'' si se mira desde la perspectiva de la Cruz.\newline
    
    Terminamos con las palabras que la Constituci\'on Lit\'urgica del Vaticano II\footnote{Sacrosanctum Concilium.} nos dice sobre el A\~no Lit\'urgico:\newline
    
    \textit{``La Santa Madre Iglesia considera deber suyo celebrar con un sagrado recuerdo en d\'ias determinados a trav\'es del a\~no la obra salv\'ifica de su divino Esposo. Cada semana en el d\'ia que llaman del Se\~nor, conmemora su resurrecci\'on, que una vez al a\~no celebra, junto con su santa pasi\'on, en la solemnidad de la Pascua. Adem\'as, en el círculo del a\~no desarrolla todo el misterio de Cristo, desde la Encarnaci\'on y la Navidad hasta la Ascensi\'on, Pentecost\'es y la expectativa de la dichosa esperanza y venida del Se\~nor. Conmemorando as\'i los misterios de la redenci\'on, abre las riquezas del poder santificador y de los m\'eritos de su Se\~nor, de tal manera que, en cierto modo, se hacen presentes en todo tiempo para que puedan los fieles ponerse en contacto con ellos y llenarse de la gracia de la salvaci\'on.\\
    En la celebraci\'on de este c\'irculo anual de los misterios de Cristo, la santa Iglesia venera con amor especial a la bienaventurada Madre de Dios, la Virgen Mar\'ia, unida con lazo indisoluble a la obra salv\'ifica de su Hijo... Adem\'as, la Iglesia introdujo en el c\'irculo anual el recuerdo de los m\'artires y de los dem\'as santos que, llegado a la perfecci\'on por la multiforme gracia de Dios, y habiendo ya alcanzado la salvaci\'on eterna, cantan la perfecta alabanza de Dios en el cielo e interceden por nosotros.''}\footnote{Sacrosanctum Concilium 102, 103, 104.}
    
    \chapter{CELEBRACIONES LIT\'URGICAS}
    Las celebraciones de la Iglesia Cat\'olica se dividen en celebraciones del Se\~nor, de la Virgen y de los Santos, y a su vez, cada uno de estos grupos y dependiendo de su grado de importancia en tres clases:
    
    \begin{itemize}
        \item \textbf{Solemnidades}: D\'ias que por ser considerados muy importantes por la Iglesia se equiparan a domingos\footnote{Pascua semanal.} y comienzan a celebrarse, por lo tanto, en las v\'isperas. Son catorce:
        
        \begin{enumerate}
            \item Anunciaci\'on.
            \item Asunci\'on.
            \item Corpus Christi.
            \item Cristo Rey del Universo.
            \item Epifan\'ia.
            \item Inmaculada Concepci\'on.
            \item Natividad de Mar\'ia.
            \item Natividad de San Juan Bautista.
            \item Navidad.
            \item Sagrado Coraz\'on de Jes\'us.
            \item San Jos\'e.
            \item Santisima Trinidad.
            \item Santos Pedro y Pablo.
            \item Todos los Santos.
        \end{enumerate}
        
        Estas solemnidades tienen todas un propio como las lecturas, prefacio, oraciones, etc. La solemnidad por excelencia es el Domingo de Pascua, en que celebramos la Resurrección del Se\~nor.
        
        \item \textbf{Fiestas}: Hoy d\'ia son veinticinco. Son d\'ias lit\'urgicos de menor rango que las solemnidades y se celebran dentro del d\'ia natural, salvo que se traten de fiestas del Se\~nor que caigan en Domingo, teniendo entonces primeras V\'isperas. Citaremos las fiestas de los distintos Ap\'ostoles, el Bautismo de Jes\'us, Sagrada Familia y otras.
        
        \item \textbf{Memorias}: Pueden ser obligatorias o libres, las obligatorias en el calendario universal son sesenta y tres. Las memorias, tanto las obligatorias como las libres, son conmemoraciones de los Santos y algunas de la Virgen.
    \end{itemize}
    
    Algunas solemnidades tienen \underline{Octava}, como Navidad y Pascua, aunque la Octava de Pascua excluye totalmente otras celebraciones, cosa que no pasa en Navidad, que admite en su octava las fiestas de San Esteban, San Juan Evangelista, Los Santos Inocentes, Sagrada Familia y María, Madre de Dios. La octava de Pentecostés está suprimida.\newline
    
    Adem\'as, seg\'un el calendario lit\'urgico, tienen categor\'ia de solemnidad las siguientes celebraciones propias de cada lugar:
    
    \begin{itemize}
        \item Solemnidad del Patr\'on principal del lugar, sea pueblo o ciudad.
        \item Solemnidad de la Dedicaci\'on y Aniversario de la Dedicaci\'on de la Iglesia propia.
        \item Solemnidad del T\'itulo de la Iglesia propia.
        \item Solemnidad o del T\'itulo, o del Fundador, o del Patrono principal de la Orden o Congregación religiosa.
    \end{itemize}
    
    \chapter{PREEMINENCIA DE LOS D\'IAS LIT\'URGICOS}
    Dentro del calendario lit\'urgico existe un orden de precedencia, de importancia, por tanto, sería el siguiente:
    
    \begin{enumerate}
        \item Triduo Pascual de la Pasi\'on y Resurrecci\'on del Se\~nor.
        \item Natividad del Se\~nor, Epifan\'ia, Ascensi\'on y Pentecost\'es.  Domingos de Adviento, Cuaresma y Pascua. Mi\'ercoles de Ceniza. Semana Santa\footnote{De Lunes a Jueves} y Octava de Pascua.
        \item Solemnidades del Se\~nor, de la Sant\'isima Virgen Mar\'ia y de los Santos. Conmemoraci\'on de todos los Fieles Difuntos.
        \item Solemnidades propias tales como:
            \begin{itemize}
                \item Solemnidad del Patrono principal del lugar\footnote{Sea Pueblo, Ciudad o Naci\'on}.
                \item Solemnidad de la Dedicaci\'on y aniversario de la Iglesia propia. 
                \item Solemnidad del T\'itulo de la Iglesia propia.
                \item Solemnidad bien del T\'itulo, Fundador o Patrono Principal de una Orden o Congregación. 
            \end{itemize}
        \item Fiestas del Se\~nor.
        \item Domingos del tiempo de Navidad y del tiempo Ordinario.
        \item Fiestas de la Virgen y de los Santos.
        \item Fiestas propias\footnote{Patronos, Dedicación de la Catedral, Fundadores, etc.}.
        \item Ferias de Adviento que van del 17 al 24 de Diciembre, Ferias de Cuaresma y Octava de Navidad.
        \item Memorias obligatorias.
        \item Memorias obligatorias propias\footnote{Patronos Secundarios de un lugar, otras Memorias inscritas en cada Diócesis, Orden o Congregación}.
        \item Memorias libres.
        \item Resto de los d\'ias feriales\footnote{Adviento hasta el 16 de Diciembre, Navidad desde el 2 de enero al S\'abado posterior a Epifanía, Ferias del Tiempo Pascual fuera de la Octava y todas las ferias del Tiempo Ordinario}.
    \end{enumerate}

    \chapter{LA PARTICIPACI\'ON DE LOS LAICOS EN LA LITURGIA}
    Hablar de creatividad y participaci\'on lit\'urgica es un tema que puede malinterpretarse. La liturgia es ejercicio del sacerdocio de Cristo, que se hace visible en la Iglesia. Toda celebraci\'on lit\'urgica es acci\'on de Cristo. En este sentido, la liturgia es de la Iglesia, no de nadie en particular, por lo cual sus ministros no pueden adue\~narse de ella. \textit{``A nadie le est\'a permitido, ni siquiera al sacerdote, ni a grupo alguno, a\~nadir, quitar o cambiar algo por propia iniciativa''}\footnote{Sacrosanctum Concilium 22 y tambi\'en CDC 846}. Para mejor ilustrar esta cuesti\'on valgan unas l\'ineas tomadas de un libro del Papa Benedicto XVI titulado \textit{``El espíritu de la Liturgia. Una introducción''}. La cita, larga pero obligada, dice as\'i:\newline
    
    \textit{``La <<creatividad>> no puede ser una categor\'ia aut\'entica en la realidad lit\'urgica. Por lo dem\'as, este t\'ermino ha crecido en el \'ambito de la cosmovisi\'on marxista. <<Creatividad>> significa que, en un mundo privado de sentido, al que se ha llegado por una evoluci\'on ciega, el hombre crea finalmente un mundo nuevo y mejor, partiendo de sus propias fuerzas. En las modernas teor\'ias del arte se alude con ello a una forma nihilista de creaci\'on: el arte no debe imitar nada; la creatividad art\'istica es el libre gobierno del hombre, que no se ata a ninguna norma ni a finalidad alguna, y que tampoco puede someterse a ninguna pregunta por el sentido. Puede que en estas visiones se perciba un clamor de libertad que, en un mundo dominado por la t\'ecnica, se convierte en un grito de socorro. El arte, as\'i concebido, aparece como el \'ultimo reducto de la libertad. El arte tiene que ver con la libertad, eso es cierto. Pero la libertad as\'i concebida est\'a vac\'ia: no libera, sino que deja que aparezca la desesperaci\'on como la \'ultima palabra de la existencia humana. Este tipo de creatividad no puede tener cabida en la liturgia. La liturgia no vive de las <<genialidades>> de cualquier individuo o de cualquier comisi\'on''}\newline
    
    Pese a lo anterior no debe pensarse que en la liturgia todo está cerrado y los ministros deben limitarse a una mera repetici\'on mec\'anica de los ritos, oraciones y r\'ubricas. Liturgia no es sin\'onimo de rigidez aunque no admite la arbitrariedad. Precisamente la no arbitrariedad es una de las características de la liturgia: se sustrae a la intervención del individuo ya que en la liturgia y mediante ella se entra en contacto con algo superior\footnote{Revelaci\'on.} y se crea una comuni\'on universal que supera las iglesias locales. El Misal es sumamente rico y variado en oraciones, prefacios, misas, como para que pueda decirse que no hay una gran variedad de textos para escoger, dependiendo l\'ogicamente del calendario lit\'urgico y otras circunstancias. En este aspecto es donde hay que encajar la creatividad lit\'urgica, escogiendo dentro de la variedad y no inventando lo que no existe. La mejor pastoral que puede hacerse consiste en una buena liturgia, no debe existir esa excusa tan recurrida de lo pastoral para justificar una liturgia mal hecha.\newline
    
    \textit{``La participaci\'on es un término que viene del lat\'in participatio\footnote{Partem-capere=tomar parte.} y es sin\'onimo de intervenci\'on, adhesión, asistencia''}. En efecto, hoy día la palabra es usada frecuentemente y todo el mundo pide, en cualquier \'ambito de la vida, participar. Para los cristianos, el fundamento de la participaci\'on est\'a en el Bautismo, ya que todo bautizado est\'a revestido de la dignidad sacerdotal. Se ha interpretado la participaci\'on pensando en que consiste en la intervenci\'on del mayor n\'umero de personas posibles durante el mayor tiempo posible. !`Craso error¡ No se trata de multiplicar vana y artificialmente las acciones a realizar pensando que con eso se aumenta la participaci\'on ya que la aut\'entica participación consiste el dar paso a la acci\'on de Dios.
    
    \chapter{LA LITURGIA Y EL CULTO}
    \textit{``La verdadera formaci\'on lit\'urgica no puede consistir en el aprendizaje y ensayo de las actividades exteriores, sino en el acercamiento a la actio esencial, que constituye la liturgia, en el acercamiento al poder transformador de Dios que, a trav\'es del acontecimiento lit\'urgico, quiere transformarnos a nosotros mismos y al mundo. Claro que, en este sentido, la formaci\'on lit\'urgica actual de los sacerdotes y de los laicos tiene un d\'eficit que causa tristeza. Queda mucho por hacer.''}\footnote{Benedicto XVI.}\newline
    
    Para la Iglesia, la liturgia es el culto oficial y p\'ublico que se tributa a Dios, seg\'un defini\'o P\'io XII. Para la Iglesia posterior al Vaticano II la liturgia es \textit{``El ejercicio del sacerdocio de Cristo.''}\footnote{Sacrosanctum Concilium 7.} Se llaman lit\'urgicas aquellas celebraciones que la Iglesia considera como suyas y est\'an contenidas en sus libros oficiales y se realizan por la comunidad y los ministros se\~nalados para cada caso como la Eucarist\'ia, los sacramentos en general, la Liturgia de las Horas y los Sacramentales.\newline
    
    En definitiva, la liturgia de la cual forma parte el culto no es m\'as que la historia de los acontecimientos salv\'ificos y el ejercicio del sacerdocio de Cristo. En ning\'un caso debe considerarse la liturgia ni como la parte externa y sensible del Culto Divino ni como un conjunto de leyes y preceptos que reglamentan los ritos sagrados.\newline
    
    La liturgia, que emplea un lenguaje simb\'olico, se vale de: \underline{F\'ormulas Lit\'urgi}- \underline{cas}\footnote{Lecturas b\'iblicas, salmos, letan\'ias, c\'anticos, doxolog\'ias, himnos, colectas, etc.}, \underline{Materias Lit\'urgicas}\footnote{Pan, vino, agua, sal, aceite, ceniza, fuego, cera, ramos de flores, incienso.} y \underline{Actitudes y Gestos}\footnote{Postraciones, genuflexiones, imposici\'on de manos, se\~nal de la cruz, elevaci\'on de manos, etc.}. As\'i mismo existen libros lit\'urgicos, hoy compendiados en el Misal Romano, Leccionario, Libro de la Sede, Libro de Preces y otros.\newline
    
    \textbf{Solamente son actos lit\'urgicos las celebraciones que expresan el misterio de Cristo y la naturaleza sacramental de la Iglesia; todo lo dem\'as son actos de piedad.}\newline
    
    Tambi\'en, la liturgia integra dos facetas que se complementan: la An\'amesis\footnote{Memorial de lo sucedido.} y la m\'imesis\footnote{La imitaci\'on de lo acontecido.}. Nace as\'i la ritualidad que imita lo que la palabra recuerda\footnote{Caso de la procesi\'on del Domingo de Ramos y de toda la religiosidad popular.}. En definitiva, en conocida frase, \textit{``Aquello que la Palabra lleva al o\'ido, la imagen lleva a la vista''}. De igual manera, lo que oramos es lo que creemos\footnote{La \textbf{lex orandi} es la expresi\'on de la \textbf{lex credendi}.}, seg\'un un axioma ya cl\'asico. \textbf{El memorial que la liturgia realiza no es mero recuerdo de lo sucedido sino una presencia real que se repite}.
    
    \part{LA M\'USICA EN LA LITURGIA}
    
    \chapter{DIFERENCIA ENTRE M\'USICA RELIGIOSA Y MUSICA        SAGRADA-LIT\'URGICA}
    El tema \textit{``M\'usica''} resulta apasionante, puesto que \textbf{la m\'usica es parte importante de nuestra vida}, todos de alguna manera escuchamos, cantamos, estudiamos, aprendemos la m\'usica. La m\'usica, !`Qu\'e actividad tan antigua, qu\'e arte tan antiguo! A veces sin quererlo, una canci\'on que est\'a de moda, a las tantas veces, si no es que a la primera ya estamos repitiendo su estribillo, o tarareando la melod\'ia principal.\\
    La m\'usica en la Iglesia, no es la excepci\'on, siempre ha sido parte importante de la misma, y en ese espacio es donde la m\'usica ha tenido uno de los m\'as fuertes empujes y evoluci\'on, por la seriedad con la que se le toma, por la naturaleza de los textos con que es interpretada, por la calidad y cantidad de sus m\'usicos, coros y compositores, ya que su principal inspiraci\'on es la Sagrada Escritura, mas no la única pero s\'i muy vinculado a ella, ya que tambi\'en existe la tradici\'on, la teolog\'ia, la poes\'ia, en torno a los misterios de Dios nuestro Se\~nor, de sus santos, en especial de la Sant\'isima Virgen Mar\'ia, que ha sido el alma de innumerable cantos.\newline
    
    Pero a\'un en la misma Iglesia, la m\'usica tiene sus distinciones, seg\'un la funci\'on que desempe\~na. \textbf{Podemos enumerar tres principales distinciones}:
    
    \section{M\'usica religiosa}
    Tomamos la definición exacta de la Instrucci\'on de la Sagrada Congregación de Ritos, M\'usica Sacra et Sacra Liturgia\footnote{1958} <<M\'usica religiosa es cualquier m\'usica que, ya sea por la intenci\'on del compositor o por el tema y el prop\'osito de la composici\'on, es capaz de provocar sentimientos piadosos y religiosos […] no est\'a habilitada para el culto divino, tiene una \'indole m\'as bien libre, y no est\'a admitida en las acciones lit\'urgicas.>>\footnote{M\'usica Sacra et Sacra Liturgia n. 10.}. <<Se inspira en un texto de la Sagrada Escritura, o en la Liturgia, o que se refiere a Dios, a la Sant\'isima Virgen Mar\'ia, a los Santos o a la Iglesia.>>\footnote{M\'usica Sacra et Sacra Liturgia n. 9.}. Su utilidad consiste en <<Crear en las iglesias un ambiente de belleza y de meditaci\'on que ayude y favorezca una disponibilidad hacia los valores del esp\'iritu, incluso entre aquellos que est\'an alejados de la Iglesia.>>. Por lo tanto <<Pueden tener su propio lugar en la iglesia, pero fuera de las celebraciones lit\'urgicas.>> aqu\'i cabe la m\'usica para horas santas, retiros, actos de piedad como el santo Rosario o V\'ia Crucis, m\'usica para la evangelizaci\'on y catequesis etc.
    
    \section{M\'usica Sagrada}
    <<Se entiende por M\'usica Sagrada o M\'usica Sacra, aqu\'ella que, creada para la celebraci\'on del culto divino, posee las cualidades de santidad y bondad de formas.>>\footnote{Musicam Sacram 4a.}, de donde nace, espont\'aneo, otro car\'acter suyo la universalidad. <<Debe ser santa y, por lo tanto, excluir todo lo profano, y no s\'olo en s\'i misma, sino en el modo en que se ejecuta. Debe ser arte verdadero, porque no es posible de otro modo que tenga sobre el \'animo de los oyentes el efecto que la Iglesia desea lograr al usar en su liturgia el arte de los sonidos. A la vez debe ser universal, en el sentido de que, aun concedi\'endose a toda naci\'on que admita en sus composiciones religiosas aquellas formas particulares que constituyen el car\'acter espec\'ifico de su propia m\'usica, \'este debe estar de tal modo subordinado a los caracteres generales de la música sagrada, que ning\'un fiel procedente de otra naci\'on experimente al o\'irla una impresión que no sea buena.>>\footnote{Tra le sollecitudini 2.}. <<Bajo el nombre de música sagrada se incluyen: el Canto Gregoriano, la Polifon\'ia Sagrada Antigua y Moderna en sus diversos g\'eneros, la M\'usica para el \'Organo y otros instrumentos admitidos en la Liturgia y el Canto Popular Sagrado, o sea, Lit\'urgico y Religioso>>\footnote{Cfr. MS 4b.} \newline
    
    El Concilio Vaticano II nos dice: <<La Iglesia no excluye de las acciones lit\'urgicas ning\'un g\'enero de m\'usica sagrada, siempre que corresponda al esp\'iritu de la misma acci\'on lit\'urgica y a la naturaleza de cada una de sus partes, y no impida la debida participaci\'on del pueblo.>>\footnote{MS 9.} De acuerdo con este deseo del Concilio se \underline{compusieron} numerosas canciones de música religiosa en lengua vern\'acula pero no siempre se atuvieron a los criterios de m\'usica sagrada y religiosa que exig\'ia la Iglesia. No toda m\'usica sagrada puede ser usada en las celebraciones lit\'urgicas. Por ejemplo hay misas hermosas compuestas por grandes autores pero su tiempo se prolonga demasiado y en un equilibrio de los ritos, hacen que se pierda el ritmo de la celebración. Hay otros cantos muy hermosos pero que no evocan misterio lit\'urgico alguno. Necesita poseer ulteriores requisitos, de naturaleza m\'as externa, pero en ning\'un modo accidentales, que se pueden resumir en el concepto de \textit{``funcionalidad litúrgica''}.
        
    \section{M\'usica Lit\'urgica}
    Es una m\'usica verdaderamente lit\'urgica es la que interpreta el sentido aut\'entico del rito, y que ha sido compuesta para tales fines; por ejemplo, procesi\'on de entrada\footnote{Canto de entrada.}, Ritos iniciales\footnote{Se\~nor ten piedad y Gloria.}, etc. lo hace comprensible y, por lo tanto, permite y conduce a la implicaci\'on y a la <<Participaci\'on activa>> de los fieles. Es la que expresa el misterio que se celebra. Entre rito y m\'usica tiene que existir una relaci\'on directa de compenetraci\'on. S\'olo as\'i la m\'usica puede considerarse y convertirse en <<Parte necesaria e integral>> de la liturgia. Del rito brota la m\'usica m\'as adecuada y directamente relacionada con lo que se celebra; y como los ritos son muchos y de diferentes naturalezas, del mismo modo las expresiones musicales ser\'an diversificadas con el fin de exaltar el contenido ritual.\newline
    
    Partiendo de estas distinciones podemos componer y utilizar la m\'usica ya compuesta adecuadamente, todo en su debido lugar. Realmente hay un lugar y momento para todo. La m\'usica lit\'urgica ocupa un lugar eminente. La invitación es a aprovechar toda la riqueza que tenemos en la Iglesia, y aquellos que tienen el don y el talento seguir aportando, con sus nuevos tesoros, al patrimonio universal de nuestra santa Madre Iglesia. Las redes sociales y el Internet son precioso medio de promoci\'on, y aprendizaje, de la buena m\'usica, Dios nos de la sabidur\'ia para compartir estos dones.
    
    \chapter{LA M\'USICA SAGRADA LIT\'URGICA}
    La m\'usica sagrada es aquella que, creada para la celebraci\'on del culto divino, posee cualidades de santidad y de perfecci\'on de formas. La m\'usica sacra ser\'a tanto m\'as santa cuanto m\'as \'intimamente est\'e unida a la acci\'on lit\'urgica, ya sea expresando con mayor delicadeza la \underline{oraci\'on} o fomentando la \underline{unanimidad}, ya enriqueciendo de mayor \underline{solemnidad} los ritos sagrados.\newline
    
    La m\'usica sagrada tiene el mismo fin que la liturgia, o sea, \underline{la gloria de} \underline{Dios y la santificaci\'on de los fieles}. La música sagrada aumenta el decoro y esplendor de las solemnidades litúrgicas. \textit{``La m\'usica sacra --dir\'a el papa Juan Pablo II-- es un medio privilegiado para facilitar una participaci\'on activa de los fieles en la acción sagrada''}.\newline
    
    La m\'usica \underline{no debe dominar la liturgia, sino servirla}. En este sentido, antes de San P\'io X se celebraban muchas misas con orquestra, algunas muy c\'elebres, que se convert\'ian a menudo en un gran concierto durante el cual ten\'ia lugar la Eucarist\'ia. Ya se desvirtuaba la finalidad profunda de la m\'usica lit\'urgica, la gloria de Dios. Amenazaba la irrupci\'on del virtuosismo, la vanidad de la propia habilidad, que ya no est\'a al servicio del todo, sino que quiere ponerse en un primer plano.\newline
    
    Todo esto hizo que en el siglo XIX, el siglo de una subjetividad que quiere emanciparse, se llegara, en muchos casos, a que lo sacro quedase atrapado en lo oper\'istico, recordando de nuevo aquellos peligros que, en su d\'ia, obligaron a intervenir al concilio de Trento, que estableci\'o la norma seg\'un la cual en la m\'usica lit\'urgica era prioritario el predominio de la palabra, limitando as\'i el uso de los instrumentos.
    
    \section{G\'eneros de m\'usica sagrada que se permiten en la Iglesia}
    San P\'io X ofreci\'o como modelo de m\'usica lit\'urgica el \underline{Canto Gregoriano}, porque serv\'ia a la liturgia sin dominarla. Tras el concilio Vaticano II, con la introducci\'on de la lengua del pueblo en la celebraci\'on, la m\'usica cambi\'o y se buscaron otras melod\'ias diferentes al gregoriano. Sin embargo, el principio de que el canto debe servir a la liturgia continúa vigente.\newline
    
    Hoy, ?`Qu\'e m\'usica sagrada permite la Iglesia? Se permiten el \underline{Canto Grego}- \underline{riano}, la \underline{Polifon\'ia Sagrada Antigua y Moderna}, la \underline{M\'usica Sagrada para \'Or}- \underline{gano} y el \underline{Canto Sagrado Popular Litúrgico y Religioso}.\newline
    
    Tambi\'en el Vaticano II permiti\'o la \underline{M\'usica Aut\'octona} de los pueblos cristianos, pero adornada de las debidas cualidades. La Iglesia aprueba y admite todas las formas musicales de arte aut\'entico, as\'i vocal como instrumental. Pero de nuevo debemos recordar el principio: \textbf{La M\'usica debe servir a la liturgia, no dominarla.}\newline
    
    Entre todos estos g\'eneros musicales, la Iglesia da la \underline{preferencia al canto} \underline{gregoriano}, que es el propio de la Liturgia romana y al que San P\'io X califica de supremo modelo de toda m\'usica sagrada, el \'unico que hered\'o de los antiguos Padres, y que custodi\'o celosamente durante el curso de los siglos en sus c\'odices lit\'urgicos.\newline
    
    \section{Instrumentos que son admitidos}
    Nos contesta el Concilio Vaticano II: \textit{``En el culto divino se pueden admitir otros instrumentos, a juicio y con consentimiento de la autoridad eclesi\'astica territorial competente, siempre que sean aptos o puedan adaptarse al uso sagrado, convengan a la dignidad del templo y contribuyan realmente a la edificaci\'on de los fieles''}\footnote{Sacrosanctum Concilium, n. 120.}\newline
    
    \section{Principios que ofrece el Papa para la m\'usica dentro de las celebraciones litúrgicas cat\'olicas}
    
    \begin{itemize}
        \item \textit{``\underline{Ante todo es necesario subrayar que la m\'usica destinada a los ritos}\\ \underline{sagrados debe tener como punto de referencia la santidad}.''}
        \item \textit{``No puede haber m\'usica destinada a las celebraciones de los ritos sagrados que no sea primero verdadero arte''}. Sin embargo, \textit{``Esta cualidad no es suficiente''} advierte el Santo Padre. \textit{``La m\'usica lit\'urgica debe en efecto responder a sus requisitos espec\'ificos: La plena adhesión a los textos que presenta, la consonancia con el tiempo y el momento lit\'urgico a la que est\'a destinada, la adecuada correspondencia con los ritos y gestos que propone''.}
        \item \textit{``El sagrado \'ambito de la celebraci\'on lit\'urgica no debe convertirse jam\'as en laboratorio de experimentos o de pr\'acticas de composici\'on y ejecuci\'on introducidas sin una atenta revisi\'on}, dice adem\'as el Papa. El canto gregoriano, dice luego Juan Pablo II, \textit{``Ocupa un lugar particular''}; pues \textit{``Sigue siendo a\'un hoy el elemento de unidad''} en la liturgia.
    \end{itemize}

    En general, se\~nala el Papa, el aspecto musical de las celebraciones lit\'urgicas \textit{``No puede ser dejado a la improvisaci\'on, ni al arbitrio de los individuos, sino que debe ser confiado a una bien concertada direcci\'on en respeto a las normas y competencias, como fruto significativo de una adecuada formaci\'on lit\'urgica.''} Por ello, en el campo lit\'urgico, el Papa se\~nala \textit{``La urgencia de promover una s\'olida formaci\'on tanto de los pastores como de los fieles laicos.''}
    
    \section{El papa Benedicto XVI enumera otros criterios sobre la m\'usica sagrada, que son importantes destacar}
    
    \begin{itemize}
        \item La letra de la m\'usica lit\'urgica tiene que estar basada en la Sagrada Escritura.
        \item La liturgia Cristiana no est\'a abierta a cualquier tipo de m\'usica.
        \item Nuestro canto lit\'urgico es participaci\'on del canto y la oraci\'on de la gran liturgia, que abarca toda la creaci\'on. As\'i vencemos el subjetivismo y el individualismo, que llevaría al virtuosismo y a la vanidad.
    \end{itemize}
    
    \chapter{LA M\'USICA SAGRADA EN EL CATECISMO DE LA IGLESIA CAT\'OLICA}
    \textit{``La tradici\'on musical de la Iglesia universal constituye un tesoro de valor inestimable que sobresale entre las dem\'as expresiones art\'isticas, principalmente porque el canto sagrado, unido a las palabras, constituye una parte necesaria o integral de la liturgia solemne.''}\footnote{Sacrosanctum Concilium 112.} La composici\'on y el canto de Salmos inspirados, con frecuencia acompa\~nados de instrumentos musicales, estaban ya estrechamente ligados a las celebraciones lit\'urgicas de la Antigua Alianza. La Iglesia contin\'ua y desarrolla esta tradici\'on: \textit{``Recitad entre vosotros salmos, himnos y c\'anticos inspirados; cantad y salmodiad en vuestro corazón al Se\~nor.''}\footnote{Efesios 5,19; Cf. Colocenses 3, 16-17.} \textit{``El que canta ora dos veces''}\footnote{San Agustín, Salmo 72, 1.}\newline
    
    El canto y la m\'usica cumplen su funci\'on de signos de una manera tanto m\'as significativa cuanto \textit{``M\'as estrechamente est\'en vinculadas a la acci\'on lit\'urgica''}\footnote{Sacrosanctum Concilium 112.}, seg\'un tres criterios principales: La belleza expresiva de la oraci\'on, la participaci\'on un\'anime de la asamblea en los momentos previstos y el car\'acter solemne de la celebraci\'on. Participan as\'i de la finalidad de las palabras y de las acciones lit\'urgicas: \textit{``La gloria de Dios y la santificaci\'on de los fieles''}\footnote{Cf. Sacrosanctum Concilium 112}: \textit{``!`Cu\'anto llor\'e al o\'ir vuestros himnos y c\'anticos, fuertemente conmovido por las voces de vuestra Iglesia, que suavemente cantaba! Entraban aquellas voces en mis o\'idos, y vuestra verdad se derret\'ia en mi coraz\'on, y con esto se inflamaba el afecto de piedad, y corr\'ian las l\'agrimas, y me iba bien con ellas''}\footnote{San Agust\'in, Confeciones IX, 6, 14.}\newline

    La armon\'ia de los signos\footnote{Canto, m\'usica, palabras y acciones.} es tanto m\'as expresiva y fecunda cuanto m\'as se expresa en la riqueza cultural propia del pueblo de Dios que celebra\footnote{Cf. Sacrosanctum Concilium 119.}. Por eso \textit{``Fom\'entese con empe\~no el canto religioso popular, de modo que en los ejercicios piadosos y sagrados y en las mismas acciones lit\'urgicas''}, conforme a las normas de la Iglesia \textit{``Resuenen las voces de los fieles''}\footnote{Sacrosanctum Concilium 118.}. Pero \textit{``Los textos destinados al canto sagrado deben estar de acuerdo con la doctrina cat\'olica; m\'as a\'un, deben tomase principalmente de la Sagrada Escritura y de las fuentes lit\'urgicas.''}\footnote{Sacrosanctum Concilium 121.}
    
    \chapter{LOS CANTOS DE LA MISA}
    La m\'usica y el canto son esenciales en toda fiesta humana, y tambi\'en en nuestras celebraciones lit\'urgicas. No son un mero adorno, \textit{``para que la celebraci\'on salga bonita''}, sino que son oraci\'on hecha m\'usica, palabra cantada. \textit{``Cantar es orar dos veces''}, afirma un dicho tradicional de la Iglesia. Para que el canto y la m\'usica tengan en la celebraci\'on de la comunidad el lugar que les corresponde, es necesaria una buena preparación lit\'urgica y musical de quienes componen el \textit{``coro''}, y que \'este sea siempre un servidor de la asamblea, no un coro \textit{``espect\'aculo''}. Ojal\'a pudi\'eramos enriquecer siempre el coro con instrumentos: guitarras como base, pero tambi\'en panderetas, bombos, bong\'os, tri\'angulos y otros instrumentos sencillos, seg\'un el gusto y las posibilidades de cada comunidad. Para las ocasiones especiales un coro con varios instrumentos enriquece la fiesta.\newline
    
    Una celebraci\'on, especialmente una eucarist\'ia sin canto, es como un d\'ia nublado: Igual es d\'ia, pero le falta algo para ser alegre, para estar lleno de vida. Por eso, el servicio que prestan los guitarristas, vocalistas y otros instrumentistas es uno de los m\'as importantes y hermosos para la liturgia de la comunidad. Los cantos de la eucarist\'ia deber\'ian ser siempre cuidadosamente preparados por el equipo de liturgia y el coro. La improvisaci\'on, lamentablemente demasiado com\'un en muchas comunidades, empobrece nuestra posibilidad de alabar al Se\~nor. Los cantos deben ser conocidos por los fieles. Una de las tareas de todo coro es ense\~nar cantos y ensayarlos con la asamblea.\newline
    
    En la misa y en los dem\'as sacramentos, cada canto tiene su sentido y su lugar. No se deber\'ia cantar cualquier canto en cualquier momento de la misa. Ni en cualquier tiempo del a\~no, porque la liturgia pasa por momentos muy diversos y caracter\'isticos a lo largo del año lit\'urgico. En ella hay varios cantos cuyo texto se halla en el propio misal: El Acto Penitencial, el \textit{``Gloria''}, el \textit{``Santo''} y el \textit{``Cordero''}.
    
    \section{Los Propios y Ordinarios de la Misa}
    El Propio, o \underline{partes variables}, se llama as\'i por cuanto puede variar de acuerdo a la festividad o solemnidad lit\'urgica del d\'ia en que se celebra la Misa. El Ordinario, o \underline{partes fijas}, se llama as\'i por su presencia en casi todas las Misas, con excepci\'on de la de Difuntos y durante la Semana Santa.\newline
    
    El Propio y el Ordinario est\'an compuestos de los siguientes cantos:

    \subsection{Ordinario}
    
    \begin{itemize}
        \item Kyrie Eleison\footnote{Acto Penitencial/Se\~nor ten piedad.}.
        \item Gloria in excelsis Deo\footnote{Gloria a Dios en lo alto del cielo.}.
        \item Credo\footnote{Creo en Dios/Profesi\'on de fe.}.
        \item Sanctus y Benedictus\footnote{Santo.}.
        \item Agnus Dei\footnote{Cordero de Dios.}.
    \end{itemize}
    
    \subsection{Propio}
    
    \begin{itemize}
        \item Introito\footnote{Entrada.}.
        \item Salmos Responsoriales.
        \item Sequentia y Aleluya\footnote{Aclamaci\'on al Evangelio.}.
        \item Offertorium\footnote{Ofertorio.}.
        \item Communio\footnote{Comunio\'on.}.
    \end{itemize}
    
    \textbf{Los cantos del Ordinario tienen una caracter\'istica importante, NO DEBEN SER MODIFICADO SU TEXTO ORIGINAL, si es modificado en alguna de sus letras, DEJA DE SER LIT\'URGICO, se mutila su originalidad y se antepone la m\'usica al texto, lo cual es un error.}
    
    \chapter{LOS CANTOS DE LA EUCARIST\'IA: SU SENTIDO}
    
    \section{Canto de Entrada}
    Acompa\~na la apertura de la celebraci\'on. Convida a la asamblea a entrar en la acci\'on com\'un y la dispone a la alabanza. La m\'usica y las palabras crean el ambiente espiritual propicio que ayuda a los participantes a entrar en comuni\'on con el misterio del tiempo, del d\'ia o de la fiesta que se celebra. En una misa m\'as festiva, acompa\~na el ingreso en procesi\'on del sacerdote que preside y de los demás ministros y ac\'olitos.\newline
    
    Puede ser un canto entonado por todos juntos, o un diálogo entre el coro y la asamblea.\newline
    
    Para este espacio, se debe tener presente que el canto debe terminar cuando el sacerdote que preside la celebración haya llegado a la sede.
    
    \section{Acto Penitencial}
    Es el \textit{``Canto del Perd\'on''}, que nos ayuda a reconocernos pecadores y necesitados de la misericordia del Se\~nor para celebrar y para vivir consecuentemente nuestra vida cristiana. \textit{``Se\~nor, ten piedad; Cristo, ten piedad; Se\~nor ten piedad''}, es el texto que aparece en el misal.\newline
    
    Si el sacerdote que preside la celebraci\'on lo ha recitado en el acto penitencial ya no es necesario hacerlo porque se estar\'ia repitiendo lo que antes se dijo en la f\'ormula de la confesi\'on general de la asamblea.
    
    \section{Gloria}
    El Gloria es un himno antiqu\'isimo y venerable con el que la Iglesia, congregada en el Esp\'iritu Santo, glorifica a Dios Padre y glorifica y le suplica al Cordero. El texto de este himno no puede cambiarse por otro. Lo inicia el sacerdote o, seg\'un las circunstancias, el cantor o el coro, y en cambio, es cantado simult\'aneamente por todos, o por el pueblo alternando con los cantores, o por los mismos cantores. Si no se canta, lo dir\'an en voz alta todos simult\'aneamente, o en dos coros que se responden el uno al otro.\newline
    
    Se canta o se dice en voz alta los domingos fuera de los tiempos de Adviento y de Cuaresma, en las solemnidades y en las fiestas, y en algunas celebraciones peculiares m\'as solemnes.
    
    \section{Salmo Responsorial}
    Despu\'es de la primera lectura, sigue el salmo responsorial, que es parte integral de la Liturgia de la Palabra y en s\'i mismo tiene gran importancia lit\'urgica y pastoral, ya que favorece la meditaci\'on de la Palabra de Dios.\newline
    
    El salmo responsorial debe corresponder a cada una de las lecturas y se toma habitualmente del leccionario.\newline
    
    \underline{Conviene que el Salmo Responsorial sea cantado}, al menos la respuesta que pertenece al pueblo. As\'i pues, el salmista o el cantor del salmo, desde el amb\'on o en otro sitio apropiado, proclama las estrofas del salmo, mientras que toda la asamblea permanece sentada, escucha y, m\'as a\'un, de ordinario participa por medio de la respuesta, a menos que el salmo se proclame de modo directo, es decir, sin respuesta. Pero, para que el pueblo pueda unirse con mayor facilidad a la respuesta salm\'odica, se escogieron unos textos de respuesta y unos de los salmos, seg\'un los distintos tiempos del a\~no o las diversas categor\'ias de Santos, que pueden emplearse en vez del texto correspondiente a la lectura, siempre que el salmo sea cantado. Si el salmo no puede cantarse, se proclama de la manera m\'as apta para facilitar la meditaci\'on de la Palabra de Dios.\newline
    
    En vez del salmo asignado en el leccionario, puede tambi\'en cantarse el Responsorio Gradual tomado del Gradual Romano, o el salmo responsorial o aleluy\'atico tomado del Gradual Simple, tal como se presentan en esos libros.
    
    \section{Aclamaci\'on antes del Evangelio}
    Despu\'es de la lectura, que precede inmediatamente al Evangelio, se canta el Aleluya u otro canto determinado por las r\'ubricas, seg\'un lo pida el tiempo lit\'urgico. Esta aclamaci\'on constituye por s\'i misma un rito, o bien un acto, por el que la asamblea de los fieles acoge y saluda al Se\~nor, quien le hablar\'a en el Evangelio, y en la cual profesa su fe con el canto. Se canta estando todos de pie, inici\'andolo los cantores o el cantor, y si fuere necesario, se repite, pero el versículo es cantado por los cantores o por un cantor.
    
    \renewcommand{\theenumi}{\alph{enumi}}
    \begin{enumerate}
        \item El Aleluya se canta en todo tiempo, excepto durante la Cuaresma. Los versículos se toman del leccionario o del Gradual.
        \item En tiempo de Cuaresma, en vez del Aleluya, se canta el vers\'iculo antes del Evangelio que aparece en el leccionario. Tambi\'en puede cantarse otro salmo u otra selecci\'on\footnote{Tracto}, seg\'un se encuentra en el Gradual.
    \end{enumerate}
    
    \textbf{La Secuencia, que s\'olo es obligatoria los d\'ias de Pascua y de Pentecost\'es, se canta antes del Aleluya.}
    
    \section{Profesi\'on de fe}
    El S\'imbolo o Profesi\'on de Fe, se orienta a que todo el pueblo reunido responda a la Palabra de Dios anunciada en las lecturas de la Sagrada Escritura y explicada por la homil\'ia. Y para que sea proclamado como regla de fe, mediante una f\'ormula aprobada para el uso lit\'urgico, que recuerde, confiese y manifieste los grandes misterios de la fe, antes de comenzar su celebración en la Eucarist\'ia.\newline
    
    El S\'imbolo debe ser cantado o recitado por el sacerdote con el pueblo los domingos y en las solemnidades; puede tambi\'en decirse en celebraciones especiales m\'as solemnes.\newline
    
    Si se canta, lo inicia el sacerdote, o seg\'un las circunstancias, el cantor o los cantores, pero ser\'a cantado o por todos juntamente, o por el pueblo alternando con los cantores. Si no se canta, ser\'a recitado por todos en conjunto o en dos coros que se alternan.
    
    \section{Preparaci\'on de los dones}
    Es un canto de la Asamblea que acompa\~na este momento en el que se ofrece el pan y el vino que se convertir\'an en el Cuerpo y en la Sangre del Se\~nor. Este canto se alarga por lo menos hasta que los dones han sido depositados sobre el altar.

    \subsection{Criterios}
    
    \begin{itemize}
        \item Es un canto que lleve el sentir de la asamblea, que ofrece el esfuerzo realizado en la jornada o semana que culmina.
        \item Debe expresar necesariamente el ofrecimiento del pan y del vino que se convertir\'an en el Cuerpo y la Sangre del Se\~nor.
        \item Debe relacionar la vida como ofrenda que se une a la oblaci\'on del Hijo por amor.
    \end{itemize}

    \section{Santo, Aclamaci\'on Eucar\'istica y Amen}
    El \textit{``Santo''} que sigue al Prefacio es la mayor aclamaci\'on de la Misa; por eso debe ser el primer canto por orden de importancia. La \textit{``Aclamaci\'on Eucar\'istica''} es la respuesta de la Asamblea a la monición del sacerdote cuando dice: \textit{``Este es el sacramento de nuestra fe''}. Al terminar la Plegaria Eucar\'istica, la Asamblea dice \textit{``Am\'en''} para unirse a la Doxolog\'ia expresada por el sacerdote.

    \subsection{Criterios}
    
    \begin{itemize}
        \item El \textit{``Santo''} no debe ser cambiado por otro canto religioso; debe conservarse la letra que aparece en el Misal\footnote{Aunque se puede hacer alguna par\'afrasis.}.
        \item Conviene que la Asamblea responda con el canto a la monici\'on del sacerdote despu\'es de la consagraci\'on: \textit{``...Este es el sacramento de nuestra fe...''}
        \item La Doxolog\'ia\footnote{\textit{``Por Cristo, con \'El...''}} la pronuncia s\'olo el sacerdote: la Asamblea se une con el \textit{``Am\'en''}\footnote{Puede ser cantado, aunque el sacerdote no hubiera cantado la Doxolog\'ia.}.
    \end{itemize}

    \section{Padre Nuestro y Cordero de Dios}
    La \textit{``Oraci\'on Dominical''} puede ser cantada. En tal caso, debe conservarse el texto lit\'urgico, tal como aparece en el Misal.

    \subsection{Criterios}
    
    \begin{itemize}
        \item Lo mismo vale para el \textit{``Cordero de Dios''}. 
        \item No existe ning\'un texto lit\'urgico para el Canto de la paz.
        \item Ser\'ia preferible no cantar nada durante el rito de la paz para que el saludo pueda ser m\'as espont\'aneo. Pero si hay alg\'un canto, \'este no debe reemplazar al \textit{``Cordero de Dios''} que por lo menos debe ser recitado durante el rito de la \textit{``Fracción del Pan.''}
        \item Tampoco se debe prolongar el Canto de Paz y el saludo, con el peligro de romper el equilibrio de los gestos.
    \end{itemize}
    
    La s\'uplica Cordero de Dios se canta seg\'un la costumbre, bien sea por los cantores, o por el cantor seguido de la respuesta del pueblo el pueblo, o por lo menos se dice en voz alta. La invocaci\'on acompa\~na la fracci\'on del pan, por lo que puede repetirse cuantas veces sea necesario hasta cuando haya terminado el rito. La \'ultima vez se concluye con las palabras danos la paz.
    
    \section{Comuni\'on}
    Mientras el sacerdote toma el Sacramento, se inicia el canto de Comuni\'on, que debe expresar, por la uni\'on de las voces, la uni\'on espiritual de quienes comulgan, manifestar el gozo del coraz\'on y esclarecer mejor la índole \textit{``Comunitaria''} de la procesi\'on para recibir la Eucarist\'ia. El canto se prolonga mientras se distribuye el Sacramento a los fieles. Pero si se ha de tener un himno despu\'es de la Comuni\'on, el canto para la Comuni\'on debe ser terminado oportunamente.\newline
    
    T\'engase cuidado de que tambi\'en los cantores puedan comulgar en el momento m\'as conveniente.\newline
    
    Para Canto de Comuni\'on puede emplearse la ant\'ifona del Gradual Romano, con su salmo o sin \'el, o la ant\'ifona con el salmo del Graduale Simplex, o alg\'un otro canto adecuado aprobado por la Conferencia de los Obispos. Lo canta el coro solo, o el coro con el pueblo, o un cantor con el pueblo.\newline
    
    Por otra parte, cuando no hay canto, se puede decir la ant\'ifona propuesta en el Misal. La pueden decir los fieles, o s\'olo algunos de ellos, o un lector, o en \'ultimo caso el mismo sacerdote, despu\'es de haber comulgado, antes de distribuir la Comuni\'on a los fieles.\newline
    
    Terminada la distribuci\'on de la Comuni\'on, si resulta oportuno, el sacerdote y los fieles oran en silencio por alg\'un intervalo de tiempo. Si se quiere, la asamblea entera tambi\'en puede cantar un salmo u otro canto de alabanza o un himno.
    
    \subsection{Criterios}
    
    \begin{itemize}
        \item El Canto de Comuni\'on empieza cuando comulga el sacerdote y se prolonga mientras comulgan los fieles, hasta el momento que parezca oportuno.
        \item El canto debe expresar, por la uni\'on de voces, la uni\'on espiritual de quienes comulgan, demostrar la alegr\'ia del coraz\'on y hacer m\'as fraternalmente la procesi\'on de los que van avanzando para recibir el Cuerpo de Cristo.
        \item El contenido del canto ha de ser propiamente \textit{``Eucar\'istico''}\footnote{Agradecer la presencia real de Jes\'us en el sacramento y la comuni\'on que El realiza en los hermanos.}.
    \end{itemize}

    \section{Canto de Meditaci\'on o Acci\'on de Gracias}
    
    \begin{itemize}
        \item En el caso de que se entone un himno despu\'es de la comuni\'on ese canto concl\'uyase a tiempo\footnote{Para dar lugar a la oraci\'on final.}.
        \item Puede ser un Salmo, un himno de acci\'on de gracias, o alg\'un otro canto de alabanza, pero siempre inspirados en la Sagrada Escrituras\footnote{Aunque no recoja ning\'un texto b\'iblico en particular.}.
        \item En caso de celebrarse la memoria de las Bienaventurada Virgen Mar\'ia, puede entrar en este momento un canto mariano. Lo mismo si es la fiesta de un santo.
        \item En cambio, no ser\'ia lit\'urgico emplear cantos con motivos profanos.
    \end{itemize}
    
    \section{Canto Final}
    Este canto no forma parte de la tradici\'on de la Iglesia, pero es muy querido en las comunidades de nuestro continente, generalmente con un sentido mariano\footnote{Dedicado a la Virgen Mar\'ia.}, de acci\'on de gracias\footnote{Por la liturgia vivida o por la vida} o de misi\'on\footnote{Ya queal salir de la misa volvemos a retomar nuestro compromiso por el Reino.}. Tiene sentido s\'olo si es un canto con la asamblea presente. El animador debe motivarla a permanecer en la iglesia. Si se está disolviendo, m\'as vale acompa\~nar el momento con m\'usica instrumental.

    \chapter{LOS TIEMPOS LIT\'URGICOS Y SUS CANTOS}
    
    \section{Adviento}
    Es un tiempo de esperanza y de alegr\'ia, esperando a Jes\'us que nace y que viene al final de la historia. Su ritmo se lo dan cuatro domingos, el primero de los cuales es el \textit{``A\~no Nuevo''} de la Iglesia. En el pasado el Adviento era m\'as penitencial, de lo cual quedan dos signos: el color morado y la \underline{supresi\'on del} \underline{canto del Gloria hasta la Navidad}. Hoy, en cambio, se acent\'ua la conversi\'on, necesaria antes de todo momento fuerte de la fe. En el Adviento se vive una doble espera, una memorial y otra hist\'orica: esperamos el nacimiento de Jes\'us en Bel\'en\footnote{Memorial} y esperamos su venida definitiva, cuando vendr\'a a instaurar para siempre el Reinado de Dios que ya est\'a en medio nuestro desde que vivi\'o entre nosotros\footnote{Hist\'orica}. Es un tiempo de esperanza, gozo, de expectativa confiada, y \'ese es el car\'acter que prevalece en los cantos.
    
    \section{Navidad}
    Es la fiesta del nacimiento de Jes\'us y el tiempo que sigue hasta la fiesta de su bautismo, algunas semanas despu\'es. En medio al tiempo de Navidad est\'a la fiesta de la Epifan\'ia, que celebra la manifestaci\'on del Ni\~no Jes\'us a todas las naciones por la visita de los magos de Oriente. Es como si todos los pueblos de la tierra hubiesen ido esa noche a ver y a llevar regalos al Ni\~no Dios. No hay ninguna fiesta cristiana que haya inspirado tantos cantos como \'esta. Los villancicos son himnos a Dios encarnado en la historia concreta de las culturas, los pueblos y las comunidades. En esta hermosa fiesta y en su octava, es bueno cantar nuestros villancicos, que se pueden tomar al inicio de la eucarist\'ia, para la comuni\'on y como canto final.
    
    \section{Cuaresma}
    Es el gran tiempo penitencial de la Iglesia, los cuarenta d\'ias de conversi\'on y purificaci\'on interior que nos preparan a la mayor fiesta cristiana del a\~no, la Pascua.\newline
    
    Comienza el Mi\'ercoles de Cenizas. Son d\'ias de escucha atenta de la Palabra de Dios que nos vuelve a llamar a un cambio de vida seg\'un el Evangelio de Jes\'us. Desde el Mi\'ercoles de Cenizas hasta la vigilia pascual \underline{no cantamos} \underline{el Aleluya}, porque ese canto es la expresi\'on del gozo de la resurrecci\'on; lo reservamos para la noche de Pascua.\newline
    
    El \underline{Gloria tampoco se reza ni canta} en todo ese tiempo, excepto en la Misa del Jueves Santo. Pero la Cuaresma no es un tiempo triste, sino m\'as bien un tiempo recogido, de meditaci\'on, que es el ambiente que nos permite estar atentos a la Palabra, reflexionar sobre nuestra vida y dar pasos de conversi\'on. Los cantos de la eucarist\'ia deber\'ian favorecer la atm\'osfera de recogimiento y conversi\'on personal y comunitaria que caracterizan este tiempo lit\'urgico.
    
    \section{Pascua y Pentecost\'es}
    La Pascua es la cumbre de las celebraciones de nuestra fe cristiana y el fundamento de nuestra esperanza. La victoria de Cristo sobre el pecado y la muerte son la raz\'on m\'as honda de la gratitud y el gozo de los creyentes. Su victoria es la nuestra, porque vive en medio nuestro y anima nuestro empe\~no de liberaci\'on, de vida, de superaci\'on del mal, de la miseria y de la violencia. El canto del Aleluya, que no hemos cantado durante la Cuaresma, rebrota en la vigilia y expresa ese sentido y ese gozo que se prolongan en la octava de Pascua y en todo el tiempo pascual, pasando por la fiesta de la Ascensi\'on del Se\~nor, hasta la fiesta de Pentecost\'es. Esos 50 d\'ias son, como nos dice la Iglesia, como un solo d\'ia de fiesta.\newline
    
    Durante todo el tiempo pascual se deber\'ian cantar cantos de resurrecci\'on, tambi\'en en la Ascensi\'on y la venida del Esp\'iritu Santo, que no son fiestas separadas. Estas dos \'ultimas tambi\'en deber\'ian agregar cantos propios, sobre todo cantos al Esp\'iritu Santo.
    
    \section{Tiempo Ordinario}
    Se lo llama tambi\'en \textit{``Tiempo Com\'un''}. En estas 33 \'o 34 semanas la Iglesia vive un tiempo m\'as normal, sin grandes celebraciones. Es el tiempo m\'as largo del a\~no lit\'urgico, de modo que en \'el es importante cuidar la variedad de la m\'usica. Como leccionario dominical se desarrolla en tres a\~nos\footnote{A, B y C.}, ser\'a la Palabra de Dios le\'ida en cada liturgia la que indicar\'a los cantos m\'as adecuados para la celebraci\'an.\newline
    
    Pero tambi\'en en el tiempo durante el a\~no caen algunas fiestas importantes que es bueno celebrar con cantos propios: La Sant\'isima Trinidad, Cuerpo y Sangre del Cristo, Sagrado Coraz\'on de Jes\'us y, el \'ultimo domingo durante el a\~no, Cristo Rey del Universo; las Fiestas Marianas y las de santos universales como San Jos\'e y San Francisco, etc. Los tiempos lit\'urgicos ofrecen la posibilidad de variar los cantos de la eucarist\'ia y, sobre todo, de adecuarlos en su sentido al momento que vive la Iglesia. Ante la opci\'on entre dos cantos, es bueno escoger siempre aquel que pertenece al tiempo lit\'urgico en curso o a la fiesta que se celebra.
    
    \section{Mar\'ia en el A\~no Lit\'urgico}
    Mar\'ia es una figura esencial de la historia de la salvaci\'on. Dios la escogi\'o como Madre de Jes\'us, es decir madre suya, para nacer entre los humanos. Disc\'ipula fiel, modelo de docilidad y entrega a la voluntad de Dios, mujer valiente, primera cristiana y por eso principal intercesora de los creyentes, la Virgen Mar\'ia est\'a hondamente arraigada en la fe cat\'olica. Am\'erica Latina es un continente Mariano. Cada pa\'is tiene a Mar\'ia como protectora o patrona, bajo diversas advocaciones: Nuestra Se\~nora de Luj\'an, Nuestra Se\~nora de Guadalupe, Nuestra Se\~nora del Carmen, Nuestra Se\~nora de Caacup\'e, etc.\newline
    
    Las fiestas que la recuerdan se reparten en todo el a\~no lit\'urgico, y le dedicamos un tiempo especial, as\'i como la conmemoraci\'on semanal de Santa Mar\'ia en la misa del s\'abado.\newline
    
    Por tanto, para los d\'ias en que se tiene una celebraci\'on especial en la liturgia a Mar\'ia, los cantos deben responder al sentido que se celebra.
    
    \part{ANEXO}
    
    \chapter{DOCUMENTOS Y RECURSOS QUE NOS HABLAN DE LA M\'USICA SAGRADA}
    
    \section{El Magisterio De La Iglesia}
    
    \begin{itemize}
        \item Mensaje del Concilio Ecum\'enico Vaticano II dirigido a los Artistas incluido en el mensaje final del concilio dirigido a la Humanidad, 8 de Diciembre de 1965.
        \item \textbf{Motu Proprio \textit{``Tra le Sollecitudini''}} de San P\'io X sobre la M\'usica Sagrada, 22 de Noviembre de 1903.
        \item \textbf{Carta Apost\'olica \textit{``Divini cultus Sanctitatem''}} de P\'io XI sobre la M\'usica Sagrada, 20 de Diciembre de 1928.
        \item \textbf{Instrucci\'on \textit{``Musicae Sacrae''}} del Papa P\'io XII sobre la M\'usica Sagrada, 25 de Diciembre de 1955.
        \item \textbf{Instrucci\'on \textit{``Musicam Sacram''}} de la Sagrada Congregaci\'on de Ritos y del Concilium sobre la m\'usica en la sagrada liturgia. Pablo VI, 5 de Marzo de 1967.
        \item Constituci\'on Apost\'olica \textit{``Laudis Canticum''} de Pablo VI, 1 de Noviembre de 1970.
        \item Sobre la dignidad de la m\'usica sagrada, de la Constituci\'on \textbf{\textit{``Sacrosanctum Concilium''}} sobre la Sagrada Liturgia del Concilio Ecum\'enico Vaticano II.
        \item La m\'usica sagrada en el Catecismo de la Iglesia Cat\'olica.
        \item \textbf{\textit{``Instrucci\'on General del Misal Romano''}}
    \end{itemize}
    
    \section{El Magisterio de Juan Pablo II}
    
    \begin{itemize}
        \item Carta de Juan Pablo II a los Artistas, 4 de Abril de 1999.
        \item Discurso de Juan Pablo II a los participantes en el Congreso internacional de m\'usica, 27 de Enero de 2001.
        \item Quir\'ografo de Juan Pablo II sobre la m\'usica sacra en el centenario del Motu Proprio <<Tra le sollecitudini>>, 22 de Noviembre de 2003.
    \end{itemize}
    
    \section{Otros}
    
    \begin{itemize}
        \item \textit{``El Esp\'iritu de la Liturgia, Introducci\'on''}, Joseph Ratzinger.
    \end{itemize}

    \section{Recursos web}
    
    \begin{itemize}
        \item \url{http://www.musicasacratlalnepantla.org/los-documentos.html}
        \item \url{http://cantoliturgico.org/index.php/documentos/itemlist/category/300-documentos-musica-liturgica}
        \item \url{http://www.parroquiasanmartin.com/documentosobremusicaenlaiglesia.html}
        \item \url{http://elcancionerocatolico.blogspot.com/}
        \item \url{http://elcancionerocatolico.blogspot.com/p/cantos-no-liturgicos.html}
        \item \url{http://creciendoenuestrafe.blogspot.com/2016/04/taller-cantos-liturgicos-para-las-misas.html}
        \item \url{https://liturgiapapal.org/index.php/manual-de-liturgia/m%C3%BAsica-lit%C3%BArgica/486-los-propios-de-la-misa-el-gradual-romano.html}
    \end{itemize}
    
    \section{Youtube}
    
    \begin{itemize}
        \item \href{https://www.youtube.com/watch?v=fneWZZxMtCg}{Un Mensaje a los ministerios de m\'usica}
        \item \href{https://www.youtube.com/watch?v=VbwpGPDTFLs&t=4330s}{Criterios lit\'urgicos para la m\'usica lit\'urgica}
        \item \href{https://www.youtube.com/watch?v=ifKsd_WaGYs}{Cat\'olicos que escuchan m\'usica protestante - Fray Nelson Medina}
        \item \href{https://www.youtube.com/watch?v=2uRWvQKjDuE}{Federico Carranza Taller M\'usicos Cat\'olicos}
        \item \href{https://www.youtube.com/channel/UCmz3xMzWaedrM6DHjGjG5kw}{Canal de formaci\'on lit\'urgica}
    \end{itemize}

    \chapter{\'INDICE DE CANTOS CARMELITANOS}
    
    \section{\'Indice Alfab\'etico}
% %%% Termina el documento
\end{document}
