% %%% Definimos el estilo del documento
\documentclass[letterpaper, 12pt]{book}

% Utilizamos un paquete para gestionar acentos y e\~nes
\usepackage[spanish]{babel}
\usepackage[utf8]{inputenc}

% Definimos TÍTULO
\title{\Huge{\textbf{Formaci\'on de Coros}}}
\author{\Large{Fray Nesol O.C.D.}}
\date{\large{2021}}

% %%% Empieza el documento
\begin{document}
    % Generar título
    \maketitle
    % Generar indice
    \tableofcontents
    
    \chapter*{Preface}\normalsize
    The book root file {\tt bookex.tex} gives a basic example of how to use \LaTeX \ for preparation of a book. Note that all \LaTeX \ commands begin with a backslash.
    
    \section{Art\'iculo en \LaTeX}
    Los art\'iculos son los tipos de documentos \LaTeX\ m\'as ampliamente utilizados, dada la sencillez en su creaci\'on.

    \subsection{Estructura de un art\'iculo}
    Este tipo de documento se puede dividir en dos partes, los campos de identificaci\'on: t\'itulo, autor y fecha. Y el cuerpo del documento, en el cual el texto pertenece a una de las siguientes unidades divisionales: resumen, secciones, subsecciones, paragrafos, subparagrafos, etc.
% %%% Termina el documento
\end{document}
