%%%%%%%%%%%%%%%%%%%%%%%%%%%%%%%%%%%%% C A B E C E R A %%%%%%%%%%%%%%%%%%%%%%%%%%%%%%%%%%
% Definimos el estilo del documento
\documentclass[12pt, letterpaper]{report}

% Utilizamos un paquete para gestionar acentos y e\~nes
\usepackage[latin1]{inputenc}
\usepackage[T1]{fontenc}
\usepackage{xcolor}
\usepackage{pifont}
% Utilizamos el paquete para gestionar imagenes jpg
\usepackage{graphicx}
\graphicspath{ {images/} }
% Letra capiral
\usepackage{lettrine}
\usepackage{enumitem}
\usepackage{lmodern}

% Definimos la zona de la pagina ocupada por el texto
\oddsidemargin -1.0cm
\headsep -1.0cm
\textwidth=18.5cm
\textheight=23cm

\setlength{\parskip}{\baselineskip}

%Empieza el documento %%%%%%%%%%%%%%%%%%%% P R I N C I P I O %%%%%%%%%%%%%%%%%%%%%%%%%%%%%%
\begin{document}

\begin{center}
\Large {\bfseries \textcolor{red}{TIEMPO ORDINARIO}}
\end{center}

\begin{center}
\Huge {\bfseries XXIII DOMINGO DEL TIEMPO ORDINARIO}
\end{center}

\begin{center}
\Large {\bfseries \textcolor{red}{PRIMERA LECTURA}}
\end{center}

\begin{center}
\large {\bfseries \textit{ \textcolor{red}{Los o\'idos del sordo se abrir\'an, la lengua del mudo cantar\'a.}}}
\end{center}

\Large {\bfseries Lectura de la profec\'ia de  Isa\'ias \hspace{1cm} \textcolor{red}{35, 4-7a}}

\lettrine[lines=2]{\bfseries \textcolor{red}{D}}{}\Large ecid a los cobardes de coraz\'on:\\
<<Sed fuertes, no tem\'ais.\\
Mirad a vuestro Dios que trae el desquite,\\
viene en persona, resarcir\'a y os salvar\'a>>.\\
Se despegar\'an los ojos del ciego, los o\'idos del sordo se abrir\'an,\\
saltar\'a como un ciervo el cojo, la lengua del mudo cantar\'a.\\
Porque han brotado aguas en el desierto, torrentes en la estepa;\\
el p\'aramo ser\'a un estanque, lo reseco un manantial.\\

{\bfseries Palabra de Dios.}

\Large {\bfseries \textcolor{red}{Salmo responsorial \hspace{1cm} Salmo 145, 7. 8-9a. 9bc-10 (R.:1)}}

\Large {\bfseries \textcolor{red}{R/.}} \hspace{1cm} Alaba, alma m\'ia, al Se\~nor. Aleluya.

{\bfseries \textcolor{red}{V/.}} \hspace{1cm} Que mantiene su fidelidad perpetuamente,\\
. \hspace{2.5cm} que hace justicia a los oprimidos,\\
. \hspace{2.5cm} que da pan a los hambrientos.\\
. \hspace{2.5cm} El Se\~nor liberta a los cautivos.
\hspace{1cm} {\bfseries \textcolor{red}{R/.}}

{\bfseries \textcolor{red}{V/.}} \hspace{1cm} El Se\~nor abre los ojos al ciego,\\
. \hspace{2.5cm} el Se\~nor endereza a los que ya se doblan,\\
. \hspace{2.5cm} el Se\~nor ama a los justos,\\
. \hspace{2.5cm} el Se\~nor guarda a los peregrinos.
\hspace{1cm} {\bfseries \textcolor{red}{R/.}}

{\bfseries \textcolor{red}{V/.}} \hspace{1cm} Sustenta al hu\'erfano y a la viuda\\
. \hspace{2.5cm} y trastorna el camino de los malvados.\\
. \hspace{2.5cm} El Se\~nor reina eternamente,\\
. \hspace{2.5cm} tu Dios, Si\'on de edad en edad.
\hspace{1cm} {\bfseries \textcolor{red}{R/.}}

\begin{center}
\Large {\bfseries \textcolor{red}{SEGUNDA LECTURA}}
\end{center}

\begin{center}
\large {\bfseries \textit{ \textcolor{red}{?`Acaso no ha elegido Dios a los pobres para hacerlos herederos del reino?}}}
\end{center}

\Large {\bfseries Lectura de la carta del ap\'ostol Santiago \hspace{1cm} \textcolor{red}{2, 1-5}}

\lettrine[lines=2]{\bfseries \textcolor{red}{H}}{}\Large ermanos m\'ios:\\
No junt\'eis la fe en nuestro Se\~nor Jesucristo glorioso con el favoritismo.\\
Por ejemplo: llegan dos hombres a la reuni\'on lit\'urgica. Uno va bien vestido y hasta con anillos en los dedos; el otro es un pobre andrajoso.\\
Veis al bien vestido y le dec\'is: <<Por favor, si\'entate aqu\'i en el puesto reservado>>. Al pobre, en cambio: <<Est\'ate ah\'i de pie o si\'entate en el suelo>>.\\
Si hac\'eis eso, ?`No sois inconsecuentes y juzg\'ais con criterios malos?\\
Queridos hermanos, escuchad: ?`Acaso no ha elegido Dios a los pobres del mundo para hacerlos ricos en la fe y herederos del reino, que prometi\'o a los que lo aman?

{\bfseries Palabra de Dios.}

\newpage

\begin{center}
\Large {\bfseries \textcolor{red}{Aleluya \hspace{1cm} Mt 4, 23}}\\
Jes\'us proclamaba el Evangelio del reino,\\
curando las dolencias del pueblo.
\end{center}

\begin{center}
\Large {\bfseries \textcolor{red}{EVANGELIO}}
\end{center}

\begin{center}
\large {\bfseries \textit{ \textcolor{red}{Hace o\'ir a los sordos y hablar a los mudos.}}}
\end{center}

\Huge \textcolor{red}{\ding{64}} \Large {\bfseries Lectura del santo Evangelio seg\'un San Marcos \hspace{1cm} \textcolor{red}{7, 31-37}}

\lettrine[lines=2]{\bfseries \textcolor{red}{E}}{}\Large n aquel tiempo, dej\'o Jes\'us el territorio de Tiro, pas\'o por Sid\'on, camino del lago de Galilea, atravesando la Dec\'apolis. Y le presentaron un sordo que, adem\'as, apenas pod\'ia hablar; y le piden que le imponga las manos.\\
\'El, apart\'andolo de la gente a un lado, le meti\'o los dedos en los o\'idos y con la saliva le toc\'o la lengua. Y, mirando al cielo, suspir\'o y le dijo:\\
--<<Effet\'a>>, esto es <<\'Abrete>>--.\\
Y al momento se le abrieron los o\'idos, se le solt\'o la traba de la lengua y hablaba sin dificultad.\\
\'El les mand\'o que no lo dijeran a nadie; pero, cuanto m\'as se lo mandaba, con m\'as insistencia lo proclamaban ellos. Y en el colmo del asombro dec\'ian:\\
--<<Todo lo ha hecho bien; hace o\'ir a los sordos y hablar a los mudos>>.--

{\bfseries Palabra del Se\~nor.}
% Termina el documento %%%%%%%%%%%%%%%%%%%%%%%%%%%%% F I N %%%%%%%%%%%%%%%%%%%%%%%%%%%%%%%%%%%%%%%%%%%%%%%%%%%%%%
\end{document}
