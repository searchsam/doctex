%%%%%%%%%%%%%%%%%%%%%%%%%%%%%%%%%%%%% C A B E C E R A %%%%%%%%%%%%%%%%%%%%%%%%%%%%%%%%%%
% Definimos el estilo del documento
\documentclass[12pt, letterpaper]{article}

%%% - Paquetes
\usepackage{babel}
\usepackage[utf8]{inputenc}
\usepackage{xcolor}
\usepackage{pifont}
\usepackage{lettrine}
\usepackage{lmodern}

%%% - Definimos la zona de la pagina ocupada por el texto
\oddsidemargin -1.0cm
\headsep -1.0cm
\textwidth=18.5cm
\textheight=23cm

%%% - Definmos el tiempo de salto de linea
\setlength{\parskip}{\baselineskip}

% - Cuerpo
\begin{document}
  %\begin{center}
  % \Large {\bfseries \textcolor{red}{LA SAGRADA FAMILIA DE JES\'US, MAR\'IA Y JOS\'E}}
  %\end{center}

  \begin{center}
    \Huge {\bfseries ORDINARIO DE LA MISA}
  \end{center}

  \begin{center}
    \Huge {\bfseries Ritos Iniciales}
  \end{center}

  \large {\textcolor{red}{Reunido el pueblo, el celebrante con los ministros va al altar mientras se entona el canto de entrada.}}

  \noindent
  \Large {\bfseries \textcolor{red}{V/.}} \hspace{1cm} {En el nombre del Padre, y del Hijo, \Huge{\textcolor{red}{\ding{64}}} \Large y del Esp\'iritu Santo.}\\
  \noindent
  \Large{{\bfseries \textcolor{red}{R/.}} \hspace{1cm} Am\'en.}

  \large {\textcolor{red}{El celebrante, extendiendo las manos, saluda al pueblo con una de las f\'ormulas siguientes:}}\\
  \Large {\bfseries \textcolor{red}{V/.}} \hspace{1cm} La gracia de nuestro Se\~nor Jesucristo,\\
  . \hspace{1.9cm} el amor del Padre y la comuni\'on del Esp\'iritu Santo\\
  . \hspace{1.9cm} est\'e con todos ustedes.\\
  \noindent
  {\bfseries \textcolor{red}{R/.}} \hspace{1cm} \Large Y con tu esp\'iritu.

  \large {\bfseries \textcolor{red}{Acto penitencial}}\\
  \large {\textcolor{red}{A continuaci\'on se hace el acto penitencial, y el sacerdote invita a los fieles al arrepentimiento
  diciendo:}}

  \lettrine[lines=2]{\bfseries \textcolor{red}{H}}{}\Large {ermanos: para celebrar dignamente estos sagrados misterios reconozcamos nuestros pecados.}

  \large {\textcolor{red}{Se hace una breve pausa en silencio. Despu\'es el sacerdote dice:}}\\
  \Large {\bfseries \textcolor{red}{V/.}} \hspace{1cm} T\'u que has sido enviado a sanar los corazones afligidos: Se\~nor, ten piedad.\\
  \noindent
  \Large {\bfseries \textcolor{red}{R/.}} \hspace{1cm} Se\~nor, ten piedad. 

  \noindent
  \Large {\bfseries \textcolor{red}{V/.}} \hspace{1cm} T\'u que has venido a llamar a los pecadores: Cristo ten piedad.\\
  \noindent
  \Large {\bfseries \textcolor{red}{R/.}} \hspace{1cm} Cristo ten piedad.

  \noindent
  \Large {\bfseries \textcolor{red}{V/.}} \hspace{1cm} T\'u que est\'as sentado a la derecha del Padre para interceder por nosotros: Se\~nor, ten piedad.\\
  \noindent
  {\bfseries \textcolor{red}{V/.}} \hspace{1cm} Se\~nor, ten piedad. 

  \large {\textcolor{red}{El sacerdote concluye con la siguiente plegaria:}}
  
  \Large {\bfseries \textcolor{red}{R/.}} \hspace{1cm} \Large Dios todopoderoso tenga misericordia de nosotros, perdone nuestros pecados y nos lleve a la vida eterna.\\
  \noindent
  \Large {\bfseries \textcolor{red}{V/.}} \hspace{1cm} Am\'en.

  \noindent
  \lettrine[lines=1]{\bfseries \textcolor{red}{O}}{}\Large {remos.}

  \Large {\bfseries \textcolor{red}{Ant\'ifona de entrada}}

  \large {\textcolor{red}{Se dice} Gloria.}

  \Large {\bfseries \textcolor{red}{Oraci\'on colecta}}

  \noindent
  \Large {\bfseries \textcolor{red}{R/.}} \hspace{1cm} Am\'en.

  \clearpage

  \begin{center}
    \Huge {\bfseries Lit\'urgia de la Palabra}
  \end{center}

  \large {\textcolor{red}{El celebrante invita a los asistentes a participar en la celebraci\'on de la Palabra de Dios.}} 

  \Large {\bfseries \textcolor{red}{Homil\'ia}}

  \large {\textcolor{red}{Despu\'es de la lectura el celebrante hace una breve homil\'ia, para ilustrar a los oyentes sobre 1o que han o\'ido.}}

  \Large {\bfseries \textcolor{red}{Credo}}

  \large {\textcolor{red}{Acabada la homil\'ia, si la liturgia del d\'ia lo prescribe, se hace la profesi\'on de fe:}}

  \noindent
  \Large {Creo en un solo Dios,\\
    Padre Todopoderoso, Creador del cielo y de la tierra.\\
    de todo lo visible y lo invisible.\\
    Creo en un solo Se\~nor, Jesucristo, Hijo \'unico de Dios,\\
    nacido del Padre antes de todos los siglos:\\
    Dios de Dios, Luz de Luz,\\
    Dios verdadero de Dios verdadero,\\
    engendrado, no creado,\\
    de la misma naturaleza del Padre,\\
    por quien todo fue hecho;\\
    que por nosotros, los hombres,\\
    y por nuestra salvaci\'on baj\'o del cielo,
  }

  \large {\textcolor{red}{En las palabras que siguen, hasta} se hizo hombre\textcolor{red}{, todos se inclinan.}}

  \noindent
  \Large {y por obra del Esp\'iritu Santo\\
    se encarn\'o de Mar\'ia, la Virgen,\\
    y se hizo hombre;\\
    y por nuestra causa fue crucificado\\
    en tiempos de Poncio Pilato;\\
    padeci\'o y fue sepultado,\\
    y resucit\'o al tercer d\'ia, seg\'un las Escrituras,\\
    y subi\'o al cielo, y est\'a sentado a la derecha del Padre;\\
    y de nuevo vendr\'a con gloria\\
    para juzgar a vivos y muertos,\\
    y su reino no tendr\'a fin.\\
    Creo en el Esp\'iritu Santo, Se\~nor y dador de vida,\\
    que procede del Padre y del Hijo,\\
    que con el Padre y el Hijo\\
    recibe una misma adoraci\'on y gloria,\\
    y que habl\'o por los profetas.
    Creo en la Iglesia,\\
    que es una, santa, cat\'olica y apost\'olica.\\
    Confieso que hay un solo Bautismo\\
    para el perd\'on de los pecados.\\
    Espero la resurrecci\'on de los muertos\\
    y la vida del mundo futuro.\\
    Am\'en.
  }

  \Large {\bfseries \textcolor{red}{Oraci\'on de los fieles}} 

  \large {\textcolor{red}{Seguidamente se tiene la oraci\'on de los fieles.}} 

  \Large {\bfseries \textcolor{red}{Rito de la Paz}}

  \large {\textcolor{red}{El celebrante, extendiendo y juntando las manos, dice:}}

  \noindent
  \Large {\bfseries \textcolor{red}{V/.}} \hspace{0.5cm} La paz del Se\~nor est\'e siempre con ustedes.\\
  \noindent
  \Large {\bfseries \textcolor{red}{R/.}} \hspace{0.5cm} Y con tu esp\'iritu. 

  \noindent
  \Large {\bfseries \textcolor{red}{V/.}} \hspace{0.5cm} Dense fraternalmente la paz. 

  \begin{center}
    \Huge {\bfseries Lit\'ugia Eucar\'istica}
  \end{center}

  \large {\textcolor{red}{Despu\'es, de pie en el centro del altar y de cara al pueblo, extendiendo y juntando las manos, dice una de las siguientes f\'ormulas:}}

  \lettrine[lines=2]{\bfseries \textcolor{red}{O}}{} \Large ren, hermanos, para que este sacrificio, m\'io y de ustedes, sea agradable a Dios, Padre todopoderoso. 

  \noindent
  \Large {\bfseries \textcolor{red}{R/.}} \hspace{0.5cm} El Se\~nor reciba de tus manos este sacrificio para alabanza y gloria de su nombre, para nuestro bien y el de toda su santa Iglesia.

  \Large {\bfseries \textcolor{red}{Oraci\'on sobre las ofrendas}}

  \noindent
  \Large {\bfseries \textcolor{red}{R/.}} \hspace{0.5cm} Am\'en.

  \begin{center}
    \Large PLEGARIA EUCARISTICA II
  \end{center}

  \Large {\bfseries \textcolor{red}{PREFACIO}}

  \noindent
  \Large {\bfseries \textcolor{red}{V/.}} \hspace{0.5cm} El Se\~nor est\'e con vosotros.

  \noindent
  \Large {\bfseries \textcolor{red}{R/.}} \hspace{0.5cm} Y con tu esp\'iritu. 

  \noindent
  \Large {\bfseries \textcolor{red}{V/.}} \hspace{0.5cm} Levantemos el coraz\'on.

  \noindent
  \Large {\bfseries \textcolor{red}{R/.}} \hspace{0.5cm} Lo tenemos levantado hacia el Se\~nor. 

  \noindent
  \Large {\bfseries \textcolor{red}{V/.}} \hspace{0.5cm} Demos gracias al Se\~nor, nuestro Dios.

  \noindent
  \Large {\bfseries \textcolor{red}{R/.}} \hspace{0.5cm} Es justo y necesario.

  \clearpage

  \lettrine[lines=2]{\bfseries \textcolor{red}{E}}{}\Large n verdad es justo y necesario,\\
  es nuestro deber y salvaci\'on\\
  darte gracias\\
  siempre y en todo lugar,\\
  a t\'i, Padre santo,\\
  por Jesucristo, tu Hijo amado.

  \noindent
  \Large Por \'el, que es tu Palabra,\\
  hiciste todas las cosas.\\
  T\'u nos lo enviaste\\
  hecho hombre por obra del Esp\'iritu Santo,\\
  para que, nacido de Mar\'ia la Virgen,\\
  fuera nuestro Salvador y Redentor.

  \lettrine[lines=2]{\bfseries \textcolor{red}{E}}{}\Large l, en cumplimiento de tu voluntad,\\
  para destruir la muerte\\
  y manifestar la resurrecci\'on,\\
  extendio sus brazos en la cruz,\\
  y as\'i adquiri\'o para ti un pueblo santo.

  \noindent
  \Large Muriendo\\
  destruy\'o nuestra muerte,\\
  resucitando\\
  restaur\'o nuestra vida.

  \lettrine[lines=2]{\bfseries \textcolor{red}{P}}{}\Large or eso,\\
  con los \'angeles y santos,\\
  cantamos tu gloria diciendo:
      
  \large{\textcolor{red}{Al final del Prefacio, junta las manos y en uni\'on del pueblo, concluye el prefacio, cantando o diciendo en voz alta:}}

  \noindent
  \Large {Santo, Santo, Santo es el Se\~nor, Dios del universo.\\
  Llenos est\'an el cielo y la tierra de tu gloria.\\
  Hosanna en el cielo.\\
  Bendito el que viene en nombre del Se\~nor.\\
  Hosanna en el cielo.}

  \Large {\bfseries \textcolor{red}{CONSAGRACI\'ON}} 

  \large{\textcolor{red}{El celebrante, con las manos extendidas, dice:}}

  \lettrine[lines=2]{\bfseries \textcolor{red}{S}}{}\Large anto eres en verdad, Se\~nor,\\
  fuente de toda santidad.

  \large{\textcolor{red}{Junta las manos y, manteni\'endolas extendidas sobre las ofrendas, dice:}}

  \noindent
  \Large Santifiques estos dones\\
  con la efusi\'on de tu Esp\'iritu,

  \large{\textcolor{red}{Junta las manos y traza el signo de la cruz sobre el pan y el c\'aliz conjuntamente, diciendo:}}

  \noindent
  \Large de manera que sean para nosotros\\
  Cuerpo y \Huge{\textcolor{red}{\ding{64}}} \Large Sangre\\
  de Jesucristo, nuestro Se\~nor.

  \large{\textcolor{red}{Junta las manos.}}

  \lettrine[lines=2]{\bfseries \textcolor{red}{E}}{}\Large l cual,\\
  cuando iba a ser entregado a su Pasi\'on,\\
  voluntariamente aceptada,

  \large{\textcolor{red}{Toma el pan y, sosteni\'endolo un poco elevado sobre el altar, prosigue:}}

  \noindent
  \Large tom\'o pan,\\ 
  y elevando los ojos T\'i, Padre, Padre,\\
  pronuncio la bendici\'on, \\
  lo parti\'o y lo dio a sus disc\'ipulos, diciendo:

  \large{\textcolor{red}{Se inclina un poco.}} 

  \noindent
  \LARGE{\bfseries{Tomad y comed todos de \'el,\\
  porque esto es mi Cuerpo,\\
  que ser\'a entregado por vosotros.}}

  \large{\textcolor{red}{Muestra el pan consagrado al pueblo, lo deposita luego sobre la patena y lo adora haciendo genuflexi\'on. Despu\'es prosigue:}}

  \noindent
  \Large Del mismo modo, acabada la cena,

  \large{\textcolor{red}{Toma el c\'aliz y, sosteni\'endolo un poco elevado sobre el altar, prosigue:}}

  \noindent
  \Large tom\'o el c\'aliz, lleno del fruto de la vid,\\
  y elevando los ojos a T\'i, Padre, Padre,\\
  pronuncio la bendici\'on,\\
  lo pas\'o a sus disc\'ipulos, diciendo:

  \large{\textcolor{red}{Se inclina un poco.}}

  \noindent
  \LARGE{\bfseries{Tomad y bebed todos de \'el,\\
  porque este es el c\'aliz de mi Sangre,\\
  Sangre de la alianza nueva y eterna,\\
  que ser\'a derramada por vosotros\\ 
  y por todos los hombres\\
  para el perd\'on de los pecados.\\
  Haced esto como mi memorial.}}

  \large{\textcolor{red}{Muestra el c\'aliz al pueblo, lo deposita luego sobre el corporal y lo adora haciendo genuflexi\'on.}}

  \noindent
  \Large {\bfseries \textcolor{red}{V/.}} \hspace{0.5cm} \'Este es el Sacramento de nuestra F\'e.

  \noindent
  \Large {\bfseries \textcolor{red}{R/.}} \hspace{0.5cm} Anunciamos tu muerte, Se\~nor,\\
  . \hspace{1.5cm} proclamamos tu resurrecci\'on.\\
  . \hspace{1.5cm} !`Maran-ath\'a! !`Maran-ath\'a! !`Maran-ath\'a!\\
  . \hspace{1.5cm} !`Ven, Se\~nor Jes\'us!

  \large{\textcolor{red}{Despu\'es el celebrante, con las manos extendidas, dice:}} 

  \Large {\bfseries \textcolor{red}{ANAMNESIS}}

  \lettrine[lines=2]{\bfseries \textcolor{red}{A}}{}\Large s\'i, pues, Padre, al celebrar ahora\\ 
  el memorial de la muerte y resurrecci\'on de tu Hijo,\\
  te ofrecemos el pan de vida\\
  y el c\'aliz de salvaci\'on,\\
  y te damos gracias\\
  porque nos haces dignos de servirte en tu presencia.

  \lettrine[lines=2]{\bfseries \textcolor{red}{T}}{}\Large e pedimos humildemente\\
  que el Esp\'iritu Santo congregue en la unidad\\
  a cuantos participamos\\
  del Cuerpo y Sangre de Cristo.

  \noindent
  Acu\'erdate, Se\~nor, de tu Iglesia extendida por toda la tierra; \\
  y con el Papa {\bfseries \textcolor{red}{N.}},\\ 
  con nuestro Obispo {\bfseries \textcolor{red}{N.}},\\
  y todos los pastores que cuidan de tu pueblo,\\
  llevala a su perfecci\'on por la caridad.

  \lettrine[lines=2]{\bfseries \textcolor{red}{A}}{}\Large cu\'erdate tambi\'en de nuestros hermanos\\
  que se durmieron en la esperanza de la resurrecc\'ion,\\
  y de todos los que han muerto\\ 
  en tu misericordia;\\
  adm\'itelos a contemplar la luz de tu rostro.

  \noindent
  Ten misericordia de todos nosotros,\\
  y as\'i, con Mar\'a, la Virgen Madre de Dios,\\
  San Jos\'e su Castisimo Esposo, \\
  los ap\'ostoles\\
  y cuantos vivieron en tu amistad\\
  a trav\'es de los tiempos,\\
  merezcamos, por tu Hijo Jesucristo,\\
  compartir la vida eterna\\
  y cantar tus alabanzas. 

  \large{\textcolor{red}{Junta las manos. Toma la patena con el pan consagrado y el c\'aliz y, sosteni\'endolos elevados, dice:}}

  \lettrine[lines=2]{\bfseries \textcolor{red}{P}}{}\Large or Cristo, con \'el y en \'el,\\
  a ti, Dios Padre omnipotente,\\
  en la unidad del Esp\'iritu Santo,\\
  todo honor y toda gloria\\
  por los siglos de los siglos.

  \Large \hspace{-0.9cm} {\bfseries \textcolor{red}{R/.}} \hspace{0.5cm} Am\'en.

  \clearpage

  \begin{center}
    \Huge {\bfseries Rito de la Comuni\'on}
  \end{center}

  \lettrine[lines=2]{\bfseries \textcolor{red}{F}}{}\Large ieles a la recomendaci\'on del Salvador\\
  y siguiendo su divina ense\~nanza,\\
  nos atrevemos a decir:

  \large{\textcolor{red}{Extiende las manos y, junto con el ministro, continua:}}

  \noindent
  \Large Padre nuestro, que est\'as en el cielo,\\
  santificado sea tu nombre;\\
  venga a nosotros tu reino;\\
  h\'agase tu voluntad en la tierra como en el cielo.\\
  Danos hoy nuestro pan de cada d\'ia;\\
  perdona nuestras ofensas,\\
  como nosotros perdonamos\\
  a los que nos ofenden;\\
  no nos dejes caer en la tentaci\'on,\\
  y l\'ibranos del mal.

  \large{\textcolor{red}{Solo el celebrante, con las manos extendidas, prosigue diciendo:}}

  \noindent
  \Large L\'ibramos Se\~nor de todos lo males,\\ 
  y conc\'edenos la paz en nuestros d\'ias,\\ 
  para que, ayudados por tu misericordia,\\ 
  vivamos siempre libres de pecado\\ 
  y protegidos de toda perturbaci\'on,\\ 
  mientras esperamos la gloriosa venida\\ 
  de nuestro Salvador Jesucristo.

  \large{\textcolor{red}{Junta las manos. El pueblo concluye la oraci\'on aclamando:}}

  \noindent
  \Large {\bfseries \textcolor{red}{R/.}} \hspace{0.5cm} Tuyo es el reino, tuyo es el poder y la gloria, por siempre, Se\~nor.

  \large{\textcolor{red}{Despu\'es el sacerdote, con las manos extendidas, dice en voz alta:}}

  \noindent
  \Large Se\~nor Jesucristo,\\ 
  que dijiste a tus Ap\'ostoles:\\ 
  <<La paz les dejo, mi paz les doy>>,\\ 
  no tengas en cuenta nuestros pecados,\\ 
  sino la fe de tu Iglesia,\\ 
  y conforme a tu palabra,\\ 
  conc\'edele la paz y la unidad.

  \large{\textcolor{red}{Junta las manos.}}

  \Large T\'u que vives y reinas por los siglos de los siglos.

  \noindent
  \Large {\bfseries \textcolor{red}{R/.}} \hspace{0.5cm} Am\'en.

  \large{\textcolor{red}{Mientras el celebrante tama el pan consagrado, lo parte sobre la patena y pone una part\'icula dentro del caliz.}}

  \noindent
  \Large Cordero de Dios que quitas el pecado del mundo,\\
  ten piedad de nosotros.

  \noindent
  \Large Cordero de Dios que quitas el pecado del mundo,\\ 
  ten piedad de nosotros.

  \noindent
  \Large Cordero de Dios que quitas el pecado del mundo,\\ 
  danos la paz.

  \large{\textcolor{red}{El celebrante hace genuflexi\'on, toma el pan consagrado y, sosteni\'endolo un poco elevado, dice con voz clara:}}

  \noindent
  \Large \'Este es el Cordero de Dios,\\ 
  que quita el pecado del mundo.\\ 
  Dichosos los invitados a la cena del Se\~nor.

  \large{\textcolor{red}{Y, juntamente con el pueblo, a\~nade:}}

  \noindent
  \Large Se\~nor, no soy digno de que entres en mi casa,\\ 
  pero una palabra tuya bastar\'a para sanarme.

  \noindent
  \Large {\bfseries \textcolor{red}{R/.}} \hspace{0.5cm} Am\'en.

  \large{\textcolor{red}{Cuando el sacerdote comulga el Cuerpo de Cristo, comienza el canto de comuni\'on.}}

  \large{\textcolor{red}{Despu\'es el sacerdote puede ir a la sede. Si se juzga oportuno, se pueden guardar unos momentos de silencio o cantar un salmo o c\'antico de alabanza.}}

  \large{\textcolor{red}{Luego de la comunion, de pie en la sede o en el altar, el sacerdote dice:}}

  \lettrine[lines=1]{\bfseries \textcolor{red}{O}}{} \Large remos.

  \Large {\bfseries \textcolor{red}{Ant\'ifona de comuni\'on}}

  \Large {\bfseries \textcolor{red}{Oraci\'on despu\'es de la comuni\'on}}

  \noindent
  \Large {\bfseries \textcolor{red}{R/.}} \hspace{0.5cm} Am\'en.

  \clearpage
  
  \begin{center}
    \Huge {\bfseries Rito de Concluci\'on}
  \end{center}

  \large{\textcolor{red}{En este momento se hacen, si es necesario y con brevedad, los oportunos anuncios o advertencias al pueblo.}}

  \large{\textcolor{red}{Despu\'es tiene lugar la despedida. El sacerdote extiende las manos hacia el pueblo y dice:}}

  \noindent
  \Large {\bfseries \textcolor{red}{V/.}} \hspace{0.5cm} El Se\~nor est\'e con ustedes.\\
  \noindent
  \Large {\bfseries \textcolor{red}{R/.}} \hspace{0.5cm} Y con tu esp\'iritu.

  \lettrine[lines=2]{\bfseries \textcolor{red}{L}}{} \Large a bendici\'on de Dios todopoderoso, \\
  Padre, Hijo \Huge{\textcolor{red}{\ding{64}}} \Large y Esp\'iritu Santo, \\
  descienda sobre vosotros y os acompa\~ne siempre.\\
  \noindent
  \Large {\bfseries \textcolor{red}{R/.}} \hspace{0.5cm} Am\'en.

  \large{\textcolor{red}{Luego el celebrante, con las manos juntas, despide al pueblo:}}

  \lettrine[lines=1]{\bfseries \textcolor{red}{P}}{}\Large ueden ir en paz.\\
  \noindent
  \Large {\bfseries \textcolor{red}{R/.}} \hspace{0.5cm} Demos gracias a Dios.

  \large{\textcolor{red}{Despu\'es el celebrante besa con veneraci\'on el altar, como al comienzo, y, hecha la debida reverencia con los ministros, se retira a la sacrist\'ia.}}
\end{document}
