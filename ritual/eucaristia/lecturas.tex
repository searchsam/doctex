%%%%%%%%%%%%%%%%%%%%%%%%%%%%%%%%%%%%% C A B E C E R A %%%%%%%%%%%%%%%%%%%%%%%%%%%%%%%%%%
% Definimos el estilo del documento
\documentclass[12pt, letterpaper]{report}

% Utilizamos un paquete para gestionar acentos y e\~nes
\usepackage[latin1]{inputenc}
\usepackage[T1]{fontenc}
\usepackage{xcolor}
\usepackage{pifont}
% Utilizamos el paquete para gestionar imagenes jpg
\usepackage{graphicx}
\graphicspath{ {images/} }
% Letra capiral
\usepackage{lettrine}
\usepackage{enumitem}
\usepackage{lmodern}

% Definimos la zona de la pagina ocupada por el texto
\oddsidemargin -1.0cm
\headsep -1.0cm
\textwidth=18.5cm
\textheight=23cm

\setlength{\parskip}{\baselineskip}

%Empieza el documento %%%%%%%%%%%%%%%%%%%% P R I N C I P I O %%%%%%%%%%%%%%%%%%%%%%%%%%%%%%
\begin{document}

    %\begin{center}
    %\Large {\bfseries \textcolor{red}{TIEMPO ORDINARIO}}
    %\end{center}

    \begin{center}
    \Huge {\bfseries V DOMINGO DEL TIEMPO ORDINARIO}
    \end{center}

    \begin{center}
    \Large {\bfseries \textcolor{red}{PRIMERA LECTURA}}
    \end{center}

    \begin{center}
    \large {\bfseries \textit{ \textcolor{red}{Aqu\'i estoy, m\'andame.}}}
    \end{center}

    \Large {\bfseries Lectura de la profec\'ia de  Isa\'ias \hspace{1cm} \textcolor{red}{6, 1-2a. 3-8}}

    \lettrine[lines=2]{\bfseries \textcolor{red}{E}}{}\Large l a\~no de la muerte del rey Oz\'ias, vi al Se\~nor sentado sobre un trono alto y excelso: la orla de su manto llenaba el templo. \\
    Y vi serafines en pie junto a \'el. Y se gritaban uno a otro, diciendo:\\
    --<<!`Santo, santo, santo, el Se\~nor de los ej\'ercitos, la tierra est\'a llena de su gloria!>>.--\\
    Y temblaban los umbrales de las puertas al clamor de su voz, y el templo estaba lleno de humo. \\
    Yo dije:\\
    --<<!`Ay de m\'i, estoy perdido! Yo, hombre de labios impuros, que habito en medio de un pueblo de labios impuros, he visto con mis ojos al Rey y Se\~nor de los ej\'ercitos.>>--\\
    Y vol\'o hacia m\'i uno de los serafines, con un ascua en la mano, que hab\'ia cogido del altar con unas tenazas; la aplic\'o a mi boca y me dijo:\\
    --<<Mira; esto ha tocado tus labios, ha desaparecido tu culpa, est\'a perdonado tu pecado.>>--\\
    Entonces, escuch\'e la voz del Se\~nor, que dec\'ia:\\
    --<<?`A qui\'en mandar\'e? ?`Qui\'en ir\'a por m\'i?.>>--\\
    Contest\'e:\\
    --<<Aqu\'i estoy, m\'andame.>>--

    {\bfseries Palabra de Dios.}

    \Large {\bfseries \textcolor{red}{Salmo responsorial \hspace{1cm} Salmo 137, 1-2a. 2bc-3. 4-5. 7c-8 (R.: 1c)}}

    \Large {\bfseries \textcolor{red}{R/.}} \hspace{1cm} Delante de los \'angeles ta\~ner\'e para ti, Se\~nor.

    {\bfseries \textcolor{red}{V/.}} \hspace{1cm} Te doy gracias, Se\~nor, de todo coraz\'on;\\
    . \hspace{2.5cm} delante de los \'angeles ta\~ner\'e para ti,\\
    . \hspace{2.5cm} me postrar\'e hacia tu santuario.
    \hspace{1cm} {\bfseries \textcolor{red}{R/.}}

    {\bfseries \textcolor{red}{V/.}} \hspace{1cm} Dar\'e gracias a tu nombre:\\
    . \hspace{2.5cm} por tu misericordia y tu lealtad,\\
    . \hspace{2.5cm} porque tu promesa supera a tu fama;\\
    . \hspace{2.5cm} cuando te invoqu\'e, me escuchaste,\\
    . \hspace{2.5cm} acreciste el valor en mi alma.
    \hspace{1cm} {\bfseries \textcolor{red}{R/.}}

    {\bfseries \textcolor{red}{V/.}} \hspace{1cm} Que te den gracias, Se\~nor,\\
    . \hspace{2.5cm} los reyes de la tierra,\\
    . \hspace{2.5cm} al escuchar el or\'aculo de tu boca;\\
    . \hspace{2.5cm} canten los caminos del Se\~nor,\\
    . \hspace{2.5cm} porque la gloria del Se\~nor es grande.
    \hspace{1cm} {\bfseries \textcolor{red}{R/.}}

    {\bfseries \textcolor{red}{V/.}} \hspace{1cm} Tu derecha me salva.\\
    . \hspace{2.5cm} El Se\~nor completar\'a sus favores conmigo:\\
    . \hspace{2.5cm} Se\~nor, tu misericordia es eterna,\\
    . \hspace{2.5cm} no abandones la obra de tus manos.
    \hspace{1cm} {\bfseries \textcolor{red}{R/.}}

    \newpage

    \begin{center}
    \Large {\bfseries \textcolor{red}{SEGUNDA LECTURA}}
    \end{center}

    \begin{center}
    \large {\bfseries \textit{ \textcolor{red}{Esto es lo que predicamos; esto es lo que hab\'eis cre\'ido.}}}
    \end{center}

    \Large {\bfseries Lectura de la carta del ap\'ostol san Pablo a los Corintios \hspace{1cm} \textcolor{red}{15, 1-11}}

    \lettrine[lines=2]{\bfseries \textcolor{red}{O}}{}\Large s recuerdo, hermanos, el Evangelio que os proclam\'e y que vosotros aceptasteis, y en el que est\'ais fundados, y que os est\'a salvando, si es que conserv\'ais el Evangelio que os proclam\'e; de lo contrario, se ha malogrado vuestra adhesi\'on a la fe.\\
    Porque lo primero que yo os transmit\'i, tal como lo hab\'ia recibido, fue esto: que Cristo muri\'o por nuestros pecados, seg\'un las Escrituras; que fue sepultado y que resucit\'o al tercer d\'ia, seg\'un las Escrituras; que se le apareci\'o a Cefas y m\'as tarde a los Doce; despu\'es se apareci\'o a m\'as de quinientos hermanos juntos, la mayor\'ia de los cuales viven todav\'ia, otros han muerto; despu\'es se le apareci\'o a Santiago, despu\'es a todos los ap\'ostoles; por \'ultimo, se me apareci\'o tambi\'en a m\'i.\\
    Porque yo soy el menor de los ap\'ostoles y no soy digno de llamarme ap\'ostol, porque he perseguido a la Iglesia de Dios.\\
    Pero por la gracia de Dios soy lo que soy, y su gracia no se ha frustrado en m\'i. Antes bien, he trabajado m\'as que todos ellos. Aunque no he sido yo, sino la gracia de Dios conmigo. Pues bien; tanto ellos como yo esto es lo que predicamos; esto es lo que hab\'eis cre\'ido.

    {\bfseries Palabra de Dios.}

    \begin{center}
    \Large {\bfseries \textcolor{red}{Aleluya \hspace{1cm} Mt 4, 19}}\\
    Venid y seguidme --dice el Se\~nor--,\\
    y os har\'e pescadores de hombres.
    \end{center}

    \begin{center}
    \Large {\bfseries \textcolor{red}{EVANGELIO}}
    \end{center}

    \begin{center}
    \large {\bfseries \textit{ \textcolor{red}{Dej\'andolo todo, lo siguieron.}}}
    \end{center}

    \Huge \textcolor{red}{\ding{64}} \Large {\bfseries Lectura del santo Evangelio seg\'un San Lucas \hspace{1cm} \textcolor{red}{5, 1-11}}

    \lettrine[lines=2]{\bfseries \textcolor{red}{E}}{}\Large n aquel tiempo, la gente se agolpaba alrededor de Jes\'us para o\'ir la palabra de Dios, estando \'el a orillas del lago de Genesaret. Vio dos barcas que estaban junto a la orilla; los pescadores hab\'ian desembarcado y estaban lavando las redes.\\
    Subi\'o a una de las barcas, la de Sim\'on, y le pidi\'o que la apartara un poco de tierra. Desde la barca, sentado, ense\~naba a la gente.\\
    Cuando acab\'o de hablar, dijo a Sim\'on:\\
    --<<Rema mar adentro, y echad las redes para pescar>>.--\\
    Sim\'on contest\'o:\\
    --<<Maestro, nos hemos pasado la noche bregando y no hemos cogido nada; pero, por tu palabra, echar\'e las redes>>.--\\
    Y, puestos a la obra, hicieron una redada de peces tan grande que reventaba la red. Hicieron se\~nas a los socios de la otra barca, para que vinieran a echarles una mano. Se acercaron ellos y llenaron las dos barcas, que casi se hund\'ian. Al ver esto, Sim\'on Pedro se arroj\'o a los pies de Jes\'us diciendo:\\
    --<<Ap\'artate de m\'i, Se\~nor, que soy un pecador>>.--\\
    Y es que el asombro se hab\'ia apoderado de \'el y de los que estaban con \'el, al ver la redada de peces que hab\'ian cogido; y lo mismo les pasaba a Santiago y Juan, hijos de Zebedeo, que eran compa\~neros de Sim\'on.\\
    Jes\'us dijo a Sim\'on:\\
    --<<No temas; desde ahora ser\'as pescador de hombres>>.--\\
    Ellos sacaron las barcas a tierra y, dej\'andolo todo, lo siguieron.

    {\bfseries Palabra del Se\~nor.}
    % Termina el documento
\end{document}
%%%%%%%%%%%%%%%%%%%%%%%%%%%%% F I N %%%%%%%%%%%%%%%%%%%%%%%%%%%%%%%%%%%%%%%%%%%%%%%%%%%%%%
