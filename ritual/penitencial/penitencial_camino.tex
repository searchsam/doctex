%%%%%%%%%%%%%%%%%%%%%%%%%%%%%%%%%%%%% C A B E C E R A %%%%%%%%%%%%%%%%%%%%%%%%%%%%%%%%%%
% Definimos el estilo del documento
\documentclass[12pt, letterpaper]{report}

% Utilizamos un paquete para gestionar acentos y e\~nes
\usepackage[latin1]{inputenc}
\usepackage[T1]{fontenc}
\usepackage{xcolor}
\usepackage{pifont}
% Utilizamos el paquete para gestionar imagenes jpg
\usepackage{graphicx}
\graphicspath{ {images/} }
% Letra capiral
\usepackage{lettrine}
\usepackage{enumitem}
\usepackage{lmodern}

% Definimos la zona de la pagina ocupada por el texto
\oddsidemargin -1.0cm
\headsep -1.0cm
\textwidth=18.5cm
\textheight=23cm

\setlength{\parskip}{\baselineskip}

% Definimos TÍTULO
\title{
	\Huge {\bfseries CELEBRACI\'ON PENITENCIAL}
}

\author{
	\large {\bfseries Rito para reconciliar a varios penitentes}\\ 
	\large {\bfseries con confesi\'on y absoluci\'om individual}
}

\date{ }

%Empieza el documento %%%%%%%%%%%%%%%%%%%% P R I N C I P I O %%%%%%%%%%%%%%%%%%%%%%%%%%%%%%
\begin{document}

% Generamos título
\maketitle

\begin{center}
\Huge {\bfseries Ritos Iniciales}
\end{center}

\Large {\bfseries \textcolor{red}{Saludo}}

\large {\textcolor{red}{ Terminado el canto, el sacerdote saluda a los asistentes, diciendo:}}

\noindent
\Large {\bfseries \textcolor{red}{R/.}} \hspace{0.5cm} {La gracia, la misericordia y la paz de Dios Padre y de Jesucristo, nuestro salvador, este con todos vosotros.}

\noindent
\Large{{\bfseries \textcolor{red}{V/.}} \hspace{0.5cm} Y con tu esp\'iritu.}

\Large {\bfseries \textcolor{red}{Invocaci\'on al Esp\'iritu Santo}}

\large {\textcolor{red}{El presidente exhorta de manera espontanea a invocar todos juntos al Esp\'iritu Santo. Se canta la ``Invocaci\'on al Esp\'iritu Santo''.}}

\Large {\bfseries \textcolor{red}{Oraci\'on}}

\large {\textcolor{red}{El sacerdote invita a todos a la oraci\'on, con estas o parecidas palabras:}}

\lettrine[lines=1]{\bfseries \textcolor{red}{O}}{}\Large {remos:}

\large {\textcolor{red}{Todos oran en silencio durante algunos momentos. Luego, el sacerdote recita la siguiente plegaria:}}

\noindent
\Large {Padre de toda misericordia y Dios de todo consuelo, \\
que no te complaces en la muerte del pecador sino en que se convierta,\\
auxilia a tu pueblo para que vuelva a ti y viva.\\
Ay\'udanos a escuchar tu palabra,\\
confesar nuestros pecados\\
y darte gracias por el perd\'on que nos otorgas.}

\newpage

\noindent
\Large {Haz que, realizando la verdad en al amor,\\
hagamos crecer todas las cosas en Cristo,\\
tu hijo, que vive y reina por los siglos de los siglos.}

\noindent
\Large {\bfseries \textcolor{red}{V/.}} \hspace{1cm} Am\'en

\begin{center}
\Huge {\bfseries Lit\'urgia de la Palabra}
\end{center}

\large {\textcolor{red}{Comienza ahora la celebraci\'on de la Palabra. Si hay varias lecturas, puede intercalarse entre ellas un salmo, un canto apropiado o un momento de silencio, para conseguir as\'i que la Palabra de Dios sea mejor comprendida por cada uno, y se le preste una mayor adhesi\'on. Si hubiese solamente una lectura, conviene que se tome del Evangelio.}}

\Large {\bfseries \textcolor{red}{Homil\'ia}}

\large {\textcolor{red}{Sigue la homil\'ia que, partiendo del texto de las lecturas, debe conducir a los penitentes al examen de conciencia y a la renovaci\'on de vida.}}

\Large {\bfseries \textcolor{red}{Examen de conciencia}}

\large {\textcolor{red}{Es conveniente que se guarde un tiempo de silencio para examinar la conciencia y suscitar la verdadera contrici\'on de los pecados. El sacerdote o el di\'acono u otro ministro, pueden ayudar a los fieles con breves pensamientos o algunas preces lit\'anicas, teniendo siempre en cuenta su mentalidad, su edad, etc.}}

\newpage

\begin{center}
\Huge {\bfseries Rito de la Reconciliaci\'on}
\end{center}

\Large {\bfseries \textcolor{red}{Exhortaci\'on}} 

\large {\textcolor{red}{Puestos de pie el Presidente y la Asamblea; el Presidente o el Diacono hace la siguiente exhortaci\'on}} 

\lettrine[lines=2]{\bfseries \textcolor{red}{H}}{}\Large {ermanos:\\
El que esta en Cristo, es una nueva creaci\'on;\\
paso lo viejo, todo es nuevo.\\
Y todo proviene de Dios,\\
que nos reconcilio consigo por Cristo\\
y nos confi\'o el ministerio de la reconciliaci\'on.\\
Porque en Cristo estaba Dios reconciliando al mundo consigo,\\
no tomando en cuenta las transgresiones de los hombres,\\
sino poniendo en nosotros la palabra de la reconciliaci\'on.\\
Somos, pues, embajadores de Cristo,\\
como si Dios exhortara por medio de nosotros.\\
En nombre de Cristo os suplicamos: !`Reconciliaos con Dios!
}

\noindent
\Large {A quien no conoci\'o el pecado,\\
le hizo pecado por nosotros,\\
para que vini\'esemos a ser justicia de dios en \'el.\\
Y como cooperadores suyos que somos,\\
os exhortamos a que no recib\'ais en vano la gracia de Dios.\\
Pues dice \'el:\\
\em <<En el tiempo favorable te escuch\'e y en el d\'ia de salvaci\'on te ayud\'e.\\
Mirad ahora el momento favorable; mirad ahora el d\'ia de salvaci\'on.>>
}

\newpage

\Large {\bfseries \textcolor{red}{Confesi\'on general de los pecados}}

\large {\textcolor{red}{El presidente invita a la asamblea a ponerse de rodillas.}}

\large {\textcolor{red}{Todos juntos dicen:}}\\
\Large {Yo confieso ante Dios todopoderoso\\
y ante vosotros, hermanos,\\
que he pecado mucho\\
de pensamiento, palabra, obra y omisi\'on.\\
Por mi culpa, por mi culpa, por mi gran culpa.\\
Por eso ruego a Santa Mar\'ia, siempre Virgen,\\
a los \'Angeles, a los Santos\\
y a vosotros, hermanos,\\
que interced\'ais por m\'i ante Dios nuestro Se\~nor.}

\large {\textcolor{red}{Estando de pie solo el ministro dice:}}

\noindent
\Large {\bfseries \textcolor{red}{V/.}}\Large { Dios todopoderoso, tenga misericordia de nosotros
perdone nuestros pecados y nos lleve a la vida eterna.}

\noindent
\Large {\bfseries \textcolor{red}{R/.}} \hspace{0.5cm} Am\'en

\large {\textcolor{red}{El ministro invita a la asamblea a ponerse de pie y dice:}}

\lettrine[lines=2]{\bfseries \textcolor{red}{B}}{}\Large {seas T\'u, Se\~nor, Padre nuestro, Dios santo, Rey eterno,\\
que por tu gran bondad e infinita misericordia\\
has mostrado tu gran amor hacia nosotros en el cuerpo de tu hijo Jes\'us\\
crucificado por nuestro pecados.}

\noindent
\Large {Yo, indigno siervo tuyo, llamado a presidir hoy esta asamblea,\\
te pido perd\'on y me apoyo en tu longanimidad,\\
conociendo que tu mismo ser que has mosteado en tu hijo,\\
es tener compasi\'on de tu creatura de modo particular cuando recurre a ti;\\
y en vez de esconder su culpa la confiesa con sincero arrepentimiento.\\
Porque as\'i esta escrito:\\
\em <<Quien esconde sus propios pecados no prosperada;\\
mas quien los confiesa y los abandona alcanzara misericordia.>>
}

\lettrine[lines=2]{\bfseries \textcolor{red}{B}}{}\Large {seas T\'u, que has manifestado tu amor\\
cancelando nuestro pecado en la cruz de tu hijo.\\
Bendito esas T\'u, que lo has resucitado para nuestra justificaci\'on.}

\noindent
\Large {Por eso nosotros nos confesamos hoy pecadores delante de ti y de tu iglesia.\\
Es cierto que hemos sido insolentes, ad\'ulteros, violentos,\\
que hemos sido impuros por las bajas paciones.\\
Hemos enga\~nado, mentido, hemos sido murmuradores, rebeldes,\\
hemos violado tus ordenes despreciado tus mandatos:\\
!`Te hemos ofendido!\\
Hemos sido inicuos, opresores, estamos obstinados por el mal:\\
Somos culpables.}

\large {\textcolor{red}{Todos juntos dicen:}}\\
\noindent
\Large {\bfseries \textcolor{red}{R/.}} \hspace{0.5cm} !`Ten misericordia de nosotros!

\noindent
\Large {Es verdad que tantas veces aquello que para ti es importante\\
nosotros lo hemos juzgado \em <<No grave>>.}

\noindent
\Large {Se\~nor, T\'u que eres rico en misericordia,\\
tardo a la c\'olera, que perdonas la culpa,\\
en el nombre de tu hijo Jesucristo\\
acoge nuestra oraci\'on y nuestro canto\\
y da a nuestro coraz\'on la conversi\'on,\\
la penitencia y la vuelta a ti.}

\newpage

\noindent
\Large {Te lo pedimos con la oraci\'on que el mismo nos ha ense\~nado.\\
Por eso, unidos a el elevando las manos a ti nos atrevemos a decir:}

\large {\textcolor{red}{Todos juntos dicen:}}\\
\Large {Padre nuestro, que est\'as en el cielo,\\
santificado sea tu Nombre;\\
venga a nosotros tu reino;\\
h\'agase tu voluntad en la tierra como en el cielo.\\
Danos hoy nuestro pan de cada d\'ia;\\
perdona nuestras ofensas,\\
como tambi\'en nosotros perdonamos\\
a los que nos ofenden;\\
no nos dejes caer en la tentaci\'on,\\
y l\'ibranos del mal.}

\Large {\bfseries \textcolor{red}{Oraci\'on Colecta}}

\lettrine[lines=2]{\bfseries \textcolor{red}{O}}{}\Large {h Dios,\\
que has dispuestos los auxilios que necesita nuestra debilidad:\\
Conc\'edenos recibir con alegr\'ia y mantener con una vida santa,\\
los frutos de tu perd\'on. Por Jesucristo nuestro Se\~nor.
}

\noindent
\Large {\bfseries \textcolor{red}{R/.}} \hspace{0.5cm} \Large {Am\'en}

\Large {\bfseries \textcolor{red}{Exhortaci\'on del Presidente}}

\noindent
\Large {Ahora yo os exhorto, hermanos, a recibir este don del Se\~nor,\\
este don de su perd\'on que se hace presente en la Iglesia y en nuestro ministerio.}

\noindent
\Large {El Se\~nor, por su gran amor en \'El que nos ha elegido,\\
y por medio de este camino de verdad, de conversi\'on,\\
est\'a haci\'endonos descubrir la viga que est\'a en nuestros ojos,\\
nuestra realidad de pecado como un acto de misericordia,\\
como un don que nos lleva a la verdad de descubrir los que somos\\
y de descubrir su infinito amor, su infinita misericordia.}

\noindent
\Large {Por eso, si tambi\'en en este tiempo con hechos,\\
con pecados, con rebeld\'ias, con murmuraciones,\\
el Se\~nor nos ha permitido descubrir m\'as a fondo nuestros pecados,\\
no es para que estemos deprimidos o desesperados,\\
sino al contrario, es un acto de misericordia del Se\~nor.}

\Large {\bfseries \textcolor{red}{Confesi\'on y absoluci\'on individual}}

\large {\textcolor{red}{Mientras se realizan las confesiones individuales la asamblea acompa\~Na al salmista cantando salmos o alg\'un canto apropiado
para la liturgia. A continuaci\'on, los fieles se acercan a los sacerdotes que se hallan en lugares adecuados y confiesan sus pecados, de los que son absueltos cada penitente individualmente. Tras la confesi\'on y, si se juzga oportuno, despu\'es de una conveniente exhortaci\'on, el sacerdote invita al penitente a ponerse de rodillas y extendiendo ambas manos, o al menos la derecha, sobre la cabeza del penitente, da la absoluci\'on, diciendo:}}

\lettrine[lines=2]{\bfseries \textcolor{red}{D}}{} \Large ios, Padre misericordioso,\\
que reconcili\'o consigo al mundo\\
por la muerte y la resurrecci\'on de su Hijo\\
y derram\'o el Esp\'iritu Santo\\
para la remisi\'on de los pecados,\\
te conceda, por el ministerio de la Iglesia,\\
el perd\'on y la paz.\\
{\bfseries YO TE ABSUELVO DE TUS PECADOS\\
EN EL NOMBRE DEL PADRE, Y DEL HIJO,\\
\Huge{\textcolor{red}{\ding{64}}} \Large Y DEL ESP\'IRITU SANTO.}

\large {\textcolor{red}{El penitente responde:}}\\
\noindent
\Large {\bfseries \textcolor{red}{R/.}} \hspace{0.5cm} \Large {Am\'en}

\Large {\bfseries \textcolor{red}{Oraci\'on final de acci\'on de gracias}}

\noindent
\Large {\bfseries \textcolor{red}{V/.}} \hspace{0.5cm} El Se\~nor est\'e con vosotros.\\
\noindent
\Large {\bfseries \textcolor{red}{R/.}} \hspace{0.5cm} Y con tu esp\'iritu.

\noindent
\Large {\bfseries \textcolor{red}{V/.}} \hspace{0.5cm} Levantemos el coraz\'on.\\
\noindent
\Large {\bfseries \textcolor{red}{R/.}} \hspace{0.5cm} Lo tenemos levantado hacia el Se\~nor. 

\noindent
\Large {\bfseries \textcolor{red}{V/.}} \hspace{0.5cm} Demos gracias al Se\~nor, nuestro Dios.\\
\noindent
\Large {\bfseries \textcolor{red}{R/.}} \hspace{0.5cm} Es justo y necesario.

\lettrine[lines=2]{\bfseries \textcolor{red}{R}}{}\Large Realmente es justo y necesario,\\
es nuestro deber y salvaci\'on\\
glorificarte, siempre Se\~nor,\\
que admirablemente has creado al hombre,\\
y m\'as admirablemente has hecho en \'el\\
una nueva creaci\'on.

\noindent
\Large T\'u, no abandonas al pecador,\\
sino que lo llamas por la fuerza de tu amor.\\
T\'u, has enviado a tu Hijo al mundo,\\
para destruir el pecado y la muerte,\\
y en su resurrecci\'on\\
no has devuelto la vida y la alegr\'ia.

\lettrine[lines=2]{\bfseries \textcolor{red}{T}}{}\Large \'u, nos renuevas por la fuerza del Evangelio\\
y de los Sacramentos.

\noindent
\Large T\'u, has derramado el Esp\'iritu Santo\\
en nuestros corazones,\\
para hacernos herederos e hijos tuyos.

\noindent
\Large T\'u, has derramado el Esp\'iritu Santo\\
en nuestros corazones,\\
para hacernos herederos e hijos tuyos.

\noindent
\Large T\'u, nos libras de la esclavitud del pecado\\
y nos transformas d\'ia a d\'ia\\
en la imagen de tu Hijo.

\lettrine[lines=2]{\bfseries \textcolor{red}{A}}{}\Large labamos y bendicimos tu nombre\\
y te damos gracias\\
por las maravillas de tu misericordia.

\noindent
\Large Y con los \'angeles y los santos,\\
cantamos, cantamos\\
el himno de tu gloria.
    
\large{\textcolor{red}{Con las manos juntas y en uni\'on del pueblo cantan o recitan en voz alta:}}

\noindent
\Large {Santo, Santo, Santo es el Se\~nor, Dios del universo.\\
Llenos est\'an el cielo y la tierra de tu gloria.\\
Hosanna en el cielo.\\
Bendito el que viene en nombre del Se\~nor.\\
Hosanna en el cielo.}

\Large {\bfseries \textcolor{red}{Rito de la Paz}}

\large {\textcolor{red}{El celebrante, extendiendo y juntando las manos, dice:}}

\noindent
\Large {\bfseries \textcolor{red}{V/.}} \hspace{0.5cm} La paz del Se\~nor est\'e siempre con ustedes.

\noindent
\Large {\bfseries \textcolor{red}{R/.}} \hspace{0.5cm} Y con tu esp\'iritu. 

\noindent
\Large {\bfseries \textcolor{red}{V/.}} \hspace{0.5cm} Dense fraternalmente la paz. 

\begin{center}
\Huge {\bfseries Rito de Concluci\'on}
\end{center}

\large{\textcolor{red}{En este momento se hacen, si es necesario y con brevedad, los oportunos anuncios o advertencias al pueblo.}}

\large{\textcolor{red}{Despu\'es tiene lugar la despedida. El sacerdote invita a la asamblea a inclinar la cabeza y bendice a todos diciendo:}}

\noindent
\Large {\bfseries \textcolor{red}{V/.}} \hspace{0.5cm} El Se\~nor este con vosotros.\\
\noindent
\Large {\bfseries \textcolor{red}{R/.}} \hspace{0.5cm} Y con tu Esp\'iritu.

\noindent
\Large {\bfseries \textcolor{red}{V/.}} \hspace{0.5cm} Que os bendiga Dios Padre Omnipotente y os acompa\~ne siempre su misericordia.\\
\noindent
\Large {\bfseries \textcolor{red}{R/.}} \hspace{0.5cm} Am\'en.

\noindent
\Large {\bfseries \textcolor{red}{V/.}} \hspace{0.5cm} Que os bendiga nuestro Se\~nor Jesucristo y os acompa\~ne siempre su amor y su alegr\'ia.\\
\noindent
\Large {\bfseries \textcolor{red}{R/.}} \hspace{0.5cm} Am\'en.

\noindent
\Large {\bfseries \textcolor{red}{V/.}} \hspace{0.5cm} Que os bendiga el Esp\'iritu Santo y os consuele siempre en vuestros sufrimientos.\\
\noindent
\Large {\bfseries \textcolor{red}{R/.}} \hspace{0.5cm} Am\'en.

\noindent
\Large {\bfseries \textcolor{red}{V/.}} \hspace{0.5cm} La bendici\'on de Dios todopoderoso, Padre, Hijo \Huge{\textcolor{red}{\ding{64}}} \Large y Esp\'iritu Santo, descienda sobre vosotros y os acompa\~ne siempre.\\
\noindent
\Large {\bfseries \textcolor{red}{R/.}} \hspace{0.5cm} Am\'en.

\large{\textcolor{red}{Luego el celebrante, con las manos juntas, despide al pueblo:}}

\noindent
\Large {\bfseries \textcolor{red}{V/.}} \hspace{0.5cm} El Se\~nor ha perdonado vuestros pecados.\\
\noindent
\Large {\bfseries \textcolor{red}{R/.}} \hspace{0.5cm} Demos gracias a Dios.


% Termina el documento %%%%%%%%%%%%%%%%%%%%%%%%%%%%% F I N %%%%%%%%%%%%%%%%%%%%%%%%%%%%%%%%%%%%%%%%%%%%%%%%%%%%%%
\end{document}
