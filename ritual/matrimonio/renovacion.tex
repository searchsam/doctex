% - Preambulo
\documentclass[12pt, letterpaper]{article}

%%% - Paquetes
\usepackage{babel}
\usepackage[utf8]{inputenc}
\usepackage{xcolor}
\usepackage{pifont}
\usepackage{lettrine}
\usepackage{lmodern}

%%% - Definimos la zona de la pagina ocupada por el texto
\oddsidemargin -1.0cm
\headsep -1.0cm
\textwidth=18.5cm
\textheight=23cm

%%% - Definmos el tiempo de salto de linea
\setlength{\parskip}{\baselineskip}

% - Cuerpo
\begin{document}
%%%%%%%%%%%%%%%%%%%% P R I N C I P I O %%%%%%%%%%%%%%%%%%%%%%%%%%%%%%
    \begin{center}
    \Large {\bfseries \textcolor{red}{CICLO C - SOLEMNIDAD}}
    \end{center}

    \begin{center}
    \Huge {\bfseries XXII DOMINGO DEL TIEMPO ORDINARIO}
    \end{center}

    \begin{center}
    \Large {\bfseries \textcolor{red}{PRIMERA LECTURA}}
    \end{center}

    \begin{center}
    \large {\bfseries \textit{ \textcolor{red}{Hazte pequeño y alcanzarás el favor de Dios.}}}
    \end{center}

    \Large {\bfseries Lectura del libro de Eclesiástico \hspace{1cm} \textcolor{red}{3, 17-18. 20. 28-29}}

    \lettrine[lines=2]{\bfseries \textcolor{red}{H}}{}\Large ijo mío, en tus asuntos procede con humildad\\
    y te querrán más que al hombre generoso.\\
    Hazte pequeño en las grandezas humanas,\\
    y alcanzarás el favor de Dios;\\
    porque es grande la misericordia de Dios,\\
    y revela sus secretos a los humildes.\\
    No corras a curar la herida del cínico,\\
    pues no tiene cura,\\
    es brote de mala planta.\\
    El sabio aprecia las sentencias de los sabios,\\
    el oído atento a la sabiduría se alegrará.

    {\bfseries Palabra de Dios.}

    \newpage

    \Large {\bfseries \textcolor{red}{Salmo responsorial \hspace{1cm} 67, 4-5ac. 6-7ab. 10-11 (R.: cf. 11b)}}

    \Large {\bfseries \textcolor{red}{R/.}} \hspace{1cm} Preparaste, oh Dios, casa para los pobres.

    {\bfseries \textcolor{red}{V/.}} \hspace{1cm} Los justos se alegran,\\
    . \hspace{2.5cm} gozan en la presencia de Dios,\\
    . \hspace{2.5cm} rebosando de alegría.\\
    . \hspace{2.5cm} Cantad a Dios, tocad en su honor;\\
    . \hspace{2.5cm} su nombre es el Señor.
    \hspace{1cm} {\bfseries \textcolor{red}{R/.}}

    {\bfseries \textcolor{red}{V/.}} \hspace{1cm} Padre de huérfanos,\\
    . \hspace{2.5cm} protector de viudas,\\
    . \hspace{2.5cm} Dios vive en su santa morada.\\
    . \hspace{2.5cm} Dios prepara casa a los desvalidos,\\
    . \hspace{2.5cm} libera a los cautivos y los enriquece.
    \hspace{1cm} {\bfseries \textcolor{red}{R/.}}

    {\bfseries \textcolor{red}{V/.}} \hspace{1cm} Que te den gracias, Se\~nor,\\
    . \hspace{2.5cm} los reyes de la tierra,\\
    . \hspace{2.5cm} al escuchar el or\'aculo de tu boca;\\
    . \hspace{2.5cm} canten los caminos del Se\~nor,\\
    . \hspace{2.5cm} porque la gloria del Se\~nor es grande.
    \hspace{1cm} {\bfseries \textcolor{red}{R/.}}

    {\bfseries \textcolor{red}{V/.}} \hspace{1cm} Derramaste en tu heredad,\\
    . \hspace{2.5cm} oh Dios, una lluvia copiosa,\\
    . \hspace{2.5cm} aliviaste la tierra extenuada;\\
    . \hspace{2.5cm} y tu rebaño habitó en la tierra\\
    . \hspace{2.5cm} que tu bondad, oh Dios, preparó para los pobres.
    \hspace{1cm} {\bfseries \textcolor{red}{R/.}}

    \newpage

    \begin{center}
    \Large {\bfseries \textcolor{red}{SEGUNDA LECTURA}}
    \end{center}

    \begin{center}
    \large {\bfseries \textit{ \textcolor{red}{Os habéis acercado al monte Sión, ciudad del Dios vivo.}}}
    \end{center}

    \Large {\bfseries Lectura de la carta del ap\'ostol san Pablo a los Hebreos \hspace{1cm} \textcolor{red}{12, 18-19. 22-24a}}

    \lettrine[lines=2]{\bfseries \textcolor{red}{H}}{}\Large ermanos:\\
    Vosotros no os habéis acercado a un monte tangible, a un fuego encendido, a densos nubarrones, a la tormenta, al sonido de la trompeta; ni habéis oído aquella voz que el pueblo, al oírla, pidió que no les siguiera hablando.\\
    Vosotros os habéis acercado al monte de Sión, ciudad del Dios vivo, Jerusalén del cielo, a millares de ángeles en fiesta, a la asamblea de los primogénitos inscritos en el cielo, a Dios, juez de todos, a las almas de los justos que han llegado a su destino y al Mediador de la Nueva Alianza, Jesús.

    {\bfseries Palabra de Dios.}

    \begin{center}
    \Large {\bfseries \textcolor{red}{Aleluya \hspace{1cm} Mt 11, 29ab}}\\
    Cargad con mi yugo y aprended de mí\\
    —dice el Señor—,\\
    que soy manso y humilde de corazón.
    \end{center}

    \newpage

    \begin{center}
    \Large {\bfseries \textcolor{red}{EVANGELIO}}
    \end{center}

    \begin{center}
    \large {\bfseries \textit{ \textcolor{red}{El que se enaltece será humillado y el que se humilla será enaltecido.}}}
    \end{center}

    \Huge \textcolor{red}{\ding{64}} \Large {\bfseries Lectura del santo Evangelio seg\'un San Lucas \hspace{1cm} \textcolor{red}{14, 1. 7-14}}

    \lettrine[lines=2]{\bfseries \textcolor{red}{U}}{}\Large n sábado, entró Jesús en casa de uno de los principales fariseos para comer, y ellos le estaban espiando.\\
    Notando que los convidados escogían los primeros puestos, les propuso esta parábola:\\
    —<<Cuando te conviden a una boda, no te sientes en el puesto principal, no sea que hayan convidado a otro de más categoría que tú; y vendrá el que os convidó a ti y al otro y te dirá:\\
    ``Cédele el puesto a éste''.\\
    Entonces, avergonzado, irás a ocupar el último puesto.\\
    Al revés, cuando te conviden, vete a sentarte en el último puesto, para que, cuando venga el que te convidó, te diga: ``Amigo, sube más arriba''.\\
    Entonces quedarás muy bien ante todos los comensales.\\
    Porque todo el que se enaltece será humillado, y el que se humilla será enaltecido>>.\\
    Y dijo al que lo había invitado:\\
    —<<Cuando des una comida o una cena, no invites a tus amigos, ni a tus hermanos, ni a tus parientes, ni a los vecinos ricos; porque corresponderán invitándote, y quedarás pagado.\\
    Cuando des un banquete, invita a pobres, lisiados, cojos y ciegos; dichoso tú, porque no pueden pagarte; te pagarán cuando resuciten los justos>>.

    {\bfseries Palabra del Se\~nor.}
%%%%%%%%%%%%%%%%%%%%%%%%%%%%% F I N %%%%%%%%%%%%%%%%%%%%%%%%%%%%%%%%%%
\end{document}
