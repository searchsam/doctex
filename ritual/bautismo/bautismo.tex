%%%%%%%%%%%%%%%%%%%%%%%%%%%%%%%%%%%%% C A B E C E R A %%%%%%%%%%%%%%%%%%%%%%%%%%%%%%%%%%
% Definimos el estilo del documento
\documentclass[12pt, letterpaper]{report}

% Utilizamos un paquete para gestionar acentos y e\~nes
\usepackage[latin1]{inputenc}
\usepackage[T1]{fontenc}
\usepackage{xcolor}
\usepackage{pifont}
% Utilizamos el paquete para gestionar imagenes jpg
\usepackage{graphicx}
\graphicspath{ {images/} }
% Letra capiral
\usepackage{lettrine}
\usepackage{enumitem}
\usepackage{lmodern}

% Definimos la zona de la pagina ocupada por el texto
\oddsidemargin -1.0cm
\headsep -1.0cm
\textwidth=18.5cm
\textheight=23cm

\setlength{\parskip}{\baselineskip}

%Empieza el documento %%%%%%%%%%%%%%%%%%%% P R I N C I P I O %%%%%%%%%%%%%%%%%%%%%%%%%%%%%%
\begin{document}

\begin{center}
\Huge {\bfseries RITUAL DEL BAUTISMO}
\end{center}

\large {\textcolor{red}{El padre y la madre, en uni\'on con los padrinos, deben presentar al ni\~no a la Iglesia para ser bautizado.}}

\begin{center}
\Huge {\bfseries Rito de Acogida}
\end{center}

\large {\textcolor{red}{Mientras los fieles cantan el celebrante se dirigir\'a al lugar donde se encuentran los padres y padrinos con los bautizandos.}}

\lettrine[lines=1]{\bfseries \textcolor{red}{H}}{}\Large {ermanos:}\\
\Large{Con gozo han vivido en el seno de su familia el nacimiento de un ni\~no. Con gozo vienen ahora a la Iglesia a dar gracias a Dios y celebrar el nuevo y definitivo nacimiento por el Bautismo.}

\noindent
\Large {Todos los aqu\'i presentes nos alegramos en este momento, porque se va a acrecentar el n\'umero de los bautizados en Cristo.}

\noindent
\Large {Que este encuentro con el Se\~nor reavive la fe de ustedes y que la paz y la alegr\'ia est\'en ahora con ustedes.}

\Large{{\bfseries \textcolor{red}{V/.}} \hspace{1cm} Y con tu esperitu.}

\large {\textcolor{red}{El celebrante interroga a los padres de cada ni\~no:}}

\Large {\bfseries \textcolor{red}{R/.}} \hspace{1cm} ?`Qu\'e nombre le han puesto a este\textcolor{red}{(}os\textcolor{red}{)} ni\~no\textcolor{red}{(}s\textcolor{red}{)}?

\large {\textcolor{red}{Padres:}}

{\bfseries \textcolor{red}{V/.}} \hspace{1cm} \Large {\bfseries \textcolor{red}{N.}}

\Large {{\bfseries \textcolor{red}{R/.}} \hspace{1cm} ?`Qu\'e ped\'is a la Iglesia para \bfseries \textcolor{red}{N.}}?

\large {\textcolor{red}{Padres:}}

{\bfseries \textcolor{red}{V/.}} \hspace{1cm} El Bautismo.

\large {\textcolor{red}{Los padres tambi\'en pueden responder: <<La gracia de Cristo>>, o bien: <<La entrada en la iglesia>>, o bien: <<La vida eterna>>.}}

\large {\textcolor{red}{El celebrante se dirige a los padres:}}

\Large {\bfseries \textcolor{red}{R/.}} \hspace{1cm} \Large {Al pedir el Bautismo para su\textcolor{red}{(}s\textcolor{red}{)} hijo\textcolor{red}{(}s\textcolor{red}{)}, ?`Saben que contraen el compromiso de educarlo\textcolor{red}{(}s\textcolor{red}{)} en la fe, para que este\textcolor{red}{(}os\textcolor{red}{)} ni\~no\textcolor{red}{(}s\textcolor{red}{)}, cumplimiendo los mandamientos de Dios, ame\textcolor{red}{(}n\textcolor{red}{)} al Se\~nor y al pr\'ojimo, como Cristo nos ense\~na en el Evangelio?}

\large {\textcolor{red}{Padres:}}

{\bfseries \textcolor{red}{V/.}} \hspace{1cm} S\'i, lo sabemos.

\large {\textcolor{red}{Dirigi\'endose despu\'es a los padrinos:}}

\Large {\bfseries \textcolor{red}{R/.}} \hspace{1cm} Y los padrinos, ?`Est\'an dispuestos a ayudar a los padres en esa tarea?

\large {\textcolor{red}{Padrinos:}}

{\bfseries \textcolor{red}{V/.}} \hspace{1cm} S\'i, estamos dispuestos. 

\large {\textcolor{red}{Prosigue el celebrante diciendo:}} 

\noindent
\Large {\bfseries \textcolor{red}{N.}}, \Large {\bfseries \textcolor{red}{N.}}, \Large {la Iglesia cristiana te\textcolor{red}{(}los\textcolor{red}{)} recibe con gran alegr\'ia. Yo, en su nombre, te\textcolor{red}{(}los\textcolor{red}{)} signo con la se\~nal de Cristo Salvador.}

\large {\textcolor{red}{Y, en silencio, signa a cada ni\~no en la frente. Despu\'es invita a los padres, y si parece oportuno a los padrinos, para que hagan lo mismo, diciendo:}} 

\noindent
\Large {Y vosotros, padres \textcolor{red}{(}y padrinos\textcolor{red}{)}, haced tambi\'en sobre ellos la se\~nal de la cruz.}

\begin{center}
\Huge {\bfseries Ritos Iniciales}
\end{center}

\large {\textcolor{red}{Reunido el pueblo, el celebrante con los ministros va al altar mientras se entona el canto de entrada.}}

\Large {\bfseries \textcolor{red}{R/.}} \hspace{1cm} {En el nombre del Padre, y del Hijo, y del Esp\'iritu Santo.}

\Large{{\bfseries \textcolor{red}{V/.}} \hspace{1cm} Am\'en.}

\large {\textcolor{red}{El celebrante, extendiendo las manos, saluda al pueblo con una de las f\'ormulas siguientes:}}

\Large {\bfseries \textcolor{red}{R/.}} \hspace{1cm} La gracia de nuestro Se\~nor Jesucristo, el amor del Padre y la comuni\'on del Esp\'iritu Santo est\'e con todos ustedes.

{\bfseries \textcolor{red}{V/.}} \hspace{1cm} \Large Y con tu esp\'iritu.

\large {\bfseries \textcolor{red}{ACTO PENITENCIAL}}

\large {\textcolor{red}{A continuaci\'on se hace el acto penitencial, y el sacerdote invita a los fieles al arrepentimiento
diciendo:}}

\noindent
\Large {Hermanos: para celebrar dignamente estos sagrados misterios reconozcamos nuestros pecados.}

\large {\textcolor{red}{Se hace una breve pausa en silencio. Despu\'es el sacerdote dice:}}

\Large {\bfseries \textcolor{red}{R/.}} \hspace{1cm} T\'u que has sido enviado a sanar los corazones afligidos: Se\~nor, ten piedad.

\Large {\bfseries \textcolor{red}{V/.}} \hspace{1cm} Se\~nor, ten piedad. 

\Large {\bfseries \textcolor{red}{R/.}} \hspace{1cm} T\'u que has venido a llamar a los pecadores: Cristo ten piedad.

\Large {\bfseries \textcolor{red}{V/.}} \hspace{1cm} Cristo ten piedad.

\Large {\bfseries \textcolor{red}{R/.}} \hspace{1cm} T\'u que est\'as sentado a la derecha del Padre para interceder por nosotros: Se\~nor, ten piedad.

{\bfseries \textcolor{red}{V/.}} \hspace{1cm} Se\~nor, ten piedad. 

\large {\textcolor{red}{El sacerdote concluye con la siguiente plegaria:}} 

\Large {\bfseries \textcolor{red}{R/.}} \hspace{1cm} Dios todopoderoso tenga misericordia de nosotros, perdone nuestros pecados y nos lleve a la vida eterna.

\Large {\bfseries \textcolor{red}{V/.}} \hspace{1cm} Am\'en.

\noindent
\Large {Oremos.}

\large {\bfseries \textcolor{red}{ANT\'IFONA DE ENTRADA \hspace{1cm} Lc 2. 16} }

\noindent
\Large {Llegaron los pastores a toda prisa y encontraron a Mar\'ia y a Jos\'e, y al ni\~no recostado en un pesebre.}

\large {\textcolor{red}{Se dice} Gloria.}

\large {\bfseries \textcolor{red}{ORACI\'ON COLECTA}}

\lettrine[lines=1]{\bfseries \textcolor{red}{S}}{}\Large {e\~nor Dios, que te dignaste dejarnos el m\'as perfecto ejemplo en la Sagrada Familia de tu Hijo, conc\'edenos benignamente que, imitando sus virtudes dom\'esticas y los lazos de caridad que la uni\'o, podamos gozar de la eterna recompensa en la alegr\'ia de tu casa. Por nuestro Se\~nor Jesucristo.}

\Large {\bfseries \textcolor{red}{V/.}} \hspace{1cm} Am\'en.

\newpage

\begin{center}
\Huge {\bfseries Lit\'urgia de la Palabra}
\end{center}

\large {\textcolor{red}{El celebrante invita a los padres, padrinos y dem\'as asistentes a participar en la celebraci\'on de la Palabra de Dios. Si las circunstancias lo permiten, h\'agase una procesi\'on con cantos hasta el lugar previsto.}} 

\Large {\bfseries \textcolor{red}{HOMIL\'IA}} 

\large {\textcolor{red}{Despu\'es de la lectura el celebrante hace una breve homil\'ia, para ilustrar a los oyentes sobre 1o que han o\'ido, haci\'endoles penetrar m\'as profundamente en el misterio del Bautismo e invit\'andoles a abrazar con entusiasmo la misi\'on que les concierne especialmente como padres y padrinos.}}

\Large {\bfseries \textcolor{red}{ORACI\'ON DE LOS FIELES}} 

\large {\textcolor{red}{Seguidamente se tiene la oraci\'on de los fieles.}} 

\lettrine[lines=1]{\bfseries \textcolor{red}{H}}{}\Large ermanos:\\
Oremos a Jesucristo nuestro Se\~nor por este\textcolor{red}{(}os\textcolor{red}{)} ni\~no\textcolor{red}{(}s\textcolor{red}{)} que ser\'a\textcolor{red}{(}n\textcolor{red}{)} bautizado\textcolor{red}{(}s\textcolor{red}{)}, por sus padres y padrinos, y por todo el pueblo santo de Dios.

\noindent
\large {\textcolor{red}{Intenciones:}} 

\noindent
\Large {Para que este\textcolor{red}{(}os\textcolor{red}{)} ni\~no\textcolor{red}{(}s\textcolor{red}{)}, al participar en el misterio de la muerte y resurrecci\'on de Cristo, alcancen nueva vida, y por el Bautismo se incorporen a su santa Iglesia.
\hspace{1cm} \bfseries \textcolor{red}{R/.}}

\noindent
\Large {Para que el Bautismo y la Confirmaci\'on le\textcolor{red}{(}los\textcolor{red}{)} hagan fiel\textcolor{red}{(}es\textcolor{red}{)} disc\'ipulo\textcolor{red}{(}s\textcolor{red}{)} suyo\textcolor{red}{(}s\textcolor{red}{)}, que de\textcolor{red}{(}n\textcolor{red}{)} testimonio del Evangelio en el mundo. 
\hspace{1cm} \bfseries \textcolor{red}{R/.}}

\noindent
\Large {Para que a trav\'es de una vida santa llegue\textcolor{red}{(}n\textcolor{red}{)} al Reino de los cielos.
\hspace{1cm} \bfseries \textcolor{red}{R/.}}

\newpage

\noindent
\Large {Para que los padres y padrinos sean ejemplo de fe viva para este\textcolor{red}{(}os\textcolor{red}{)} ni\~no\textcolor{red}{(}s\textcolor{red}{)}.
\hspace{1cm} \bfseries \textcolor{red}{R/.}}

\noindent
\Large {Para que Dios guarde siempre en su amor a estas familias.
\hspace{1cm} \bfseries \textcolor{red}{R/.}}

\noindent
\Large {Para que renueve en todos nosotros la gracia del Bautismo.
\hspace{1cm} \bfseries \textcolor{red}{R/.}}

\large {\textcolor{red}{Despu\'es el celebrante invita a los presentes a invocar a los santos.}} 

\noindent
\Large \begin{tabular}{ll}
Santa Mar\'ia, Madre de Dios.& Ruega por nosotros. \\
San Jos\'e, esposo de la Virgen.& Ruega por nosotros. \\
San Juan Bautista.& Ruega por nosotros. \\
Santos ap\'ostoles Pedro y Pablo.& Ruegad por nosotros.
\end{tabular}

\large {\textcolor{red}{Combiene a\~nadir los nombres de otros santos especialmente los patronos de los ni\~nos, de la iglesia o del lugar. Despu\'es el celebrante invita a los presentes a invocar a los santos.}} 

\large {\textcolor{red}{Se termina as\'i:}} 

\noindent
\Large Todos los santos y santas de Dios. Rogad por nosotros. 

\Large {\bfseries \textcolor{red}{ORACI\'ON DE EXORCISMO}} 

\large {\textcolor{red}{Acabadas las invocaciones, el celebrante dice:}} 

\lettrine[lines=1]{\bfseries \textcolor{red}{D}}{}\Large ios todopoderoso y eterno, que has enviado tu Hijo al mundo, para librarnos del dominio de Satan\'as, esp\'iritu del mal, y llevarnos as\'i, arrancados de las tinieblas, al Reino de tu luz admirable; te pedimos que este\textcolor{red}{(}os\textcolor{red}{)} ni\~no\textcolor{red}{(}s\textcolor{red}{)}, lavado\textcolor{red}{(}s\textcolor{red}{)} del pecado original, sea\textcolor{red}{(}n\textcolor{red}{)} templo tuyo, y que el Esp\'iritu Santo habite en \'el\textcolor{red}{(}ellos\textcolor{red}{)}. Por Jesucristo nuestro Se\~nor.

\newpage

\large {\textcolor{red}{Todos:}}

\Large {\bfseries \textcolor{red}{R/.}} \hspace{0.5cm} Am\'en.


\Large {\bfseries \textcolor{red}{UNCI\'ON PREBAUTISMAL}} 

\large {\textcolor{red}{Prosigue el celebrante:}}

\lettrine[lines=1]{\bfseries \textcolor{red}{P}}{}\Large ara que el poder de Cristo Salvador te\textcolor{red}{(}los\textcolor{red}{)} fortalezca, te\textcolor{red}{(}los\textcolor{red}{)} ungimos con este \'oleo de salvaci\'on en el nombre del mismo Jesucristo, Se\~nor nuestro, que vive y reina por los siglos de los siglos.

\large {\textcolor{red}{Todos:}}

\Large {\bfseries \textcolor{red}{R/.}} \hspace{0.5cm} Am\'en.

\large {\textcolor{red}{Se hace la unci\'on con el \'oleo de los catec\'umenos en el pecho.}}

%SALTO DE PAGINA%%%
\newpage

\begin{center}
\Huge {\bfseries Lit\'urgia Bautismal}
\end{center}

\large {\textcolor{red}{El celebrante, los padres y padrinos con los ni\~nos se acercar\'an a1 baptisterio, permaneciendo los dem\'as en su sitio.}} 

\Large {\bfseries \textcolor{red}{BENDICI\'ON DEL AGUA BAUTISMAL}} 

\large {\textcolor{red}{Cuando hubieren llegado a la fuente bautismal, el celebrante recordar\'a brevemente a los presentes la admirable providencia de Dios, que ha querido santificar el alma y el cuerpo del hombre por medio del agua.}} 

\lettrine[lines=1]{\bfseries \textcolor{red}{O}}{}\Large remos, hermanos, al Se\~nor Dios todopoderoso para que conceda a estos ni\~nos la vida nueva por el agua y el Esp\'iritu Santo.

\large {\textcolor{red}{El Sacerdote, vuelto hacia la fuente, dice la siguiente bendici\'on:}}

\begin{center}
\Large {\bfseries \textcolor{red}{I}} 
\end{center}

\lettrine[lines=1]{\bfseries \textcolor{red}{O}}{}\Large h Dios,\\ 
que por medio de los signos sacramentales\\ 
t\'u obras con invisible potencia,\\ 
las maravillas de la Salvaci\'on.

\noindent
De muchos modos,\\ 
a trav\'es de los tiempos\\ 
has preparado el agua, tu criatura,\\ 
para que fuese signo del Bautismo.

\noindent
Desde los or\'igenes tu Esp\'iritu\\ 
aleteaba sobre las aguas,\\ 
para que contuviesen la fuerza de santificar.

\noindent
Y tambi\'en, en el diluvio\\ 
has prefigurado el Bautismo,\\ 
para que hoy como ayer\\ 
el agua se\~nalase el fin del pecado\\ 
y el inicio de la Vida Nueva.

\noindent
T\'u has liberado de la esclavitud\\ 
a los hijos de Abraham,\\ 
haci\'endoles pasar ilesos el Mar Rojo,\\ 
para que fuesen la imagen del futuro pueblo\\ 
de bautizados.

\noindent
Por fin, en la plenitud de los tiempos,\\ 
tu Hijo, bautizado en el agua del Jord\'an, \\ 
fue consagrado por el Esp\'iritu Santo.

\noindent
Levantado en la Cruz\\ 
de su costado sali\'o\\ 
sangre y agua.

\noindent
Y despu\'es de su resurrecci\'on\\ 
orden\'o a sus disc\'ipulos:\\ 
``Id y anunciad el Evangelio\\ 
a todos los pueblos,\\ 
bautiz\'andolos en el nombre del Padre,\\ 
y del Hijo, y del Esp\'iritu Santo''

\newpage

\lettrine[lines=1]{\bfseries \textcolor{red}{A}}{}\Large{hora, ahora Padre,\\ 
mira con amor a tu Iglesia,\\ 
y haz brotar para ella\\ 
la fuente del Bautismo.}

\noindent
Infunde en este agua,\\ 
por obra del Esp\'iritu Santo,\\ 
la Gracia de tu \'Unico Hijo.

\noindent
Para que por el Sacramento del Bautismo\\ 
el hombre hecho a tu imagen\\ 
sea lavado de todos sus pecados\\ 
y, del agua del Esp\'iritu Santo,\\ 
renazca como nueva criatura.

\large {\textcolor{red}{El celebrante toca el agua con la mano derecha y prosigue:}}

\lettrine[lines=1]{\bfseries \textcolor{red}{D}}{}\Large escienda Padre,\\ 
en este agua, por obra de tu Hijo,\\ 
la potencia del Esp\'iritu Santo.

\noindent
Para que todos aquellos que hoy reciban el Bautismo\\ 
sean sepultados con Cristo.

\noindent
Y muertos con \'El. !`Resurjan!\\ 
!`Resuciten! a la Vida Inmortal.

\noindent
Por Cristo, nuestro Se\~nor.

\large {\textcolor{red}{Todos:}}

\Large {\bfseries \textcolor{red}{R/.}} \hspace{0.5cm} Am\'en.

\newpage

\begin{center}
\Large {\bfseries \textcolor{red}{II}} 
\end{center}

\lettrine[lines=1]{\bfseries \textcolor{red}{T}}{}\Large e bendecimos, Padre todopoderoso, que hiciste le agua para purificarnos y darnos vida.

\large {\textcolor{red}{Todos:}}

\Large {\bfseries \textcolor{red}{R/.}} \hspace{0.5cm} Bendito seas por siempre, Se\~nor.

\noindent
Te bendecimos, Jesucristo, Hijo \'unico del Padre, que hiciste brotar de tu costado abierto sangre y agua, para que de tu Muerte y Resurrecci\'on naciera la Iglesia.

\large {\textcolor{red}{Todos:}}

\Large {\bfseries \textcolor{red}{R/.}} \hspace{0.5cm} Bendito seas por siempre, Se\~nor.

\noindent
Te bendecimos, Esp\'iritu Santo, que ungiste a Cristo al ser bautizado en las aguas del Jord\'an, para que todos seamos bautizados en ti.

\large {\textcolor{red}{Todos:}}

\Large {\bfseries \textcolor{red}{R/.}} \hspace{0.5cm} Bendito seas por siempre, Se\~nor.

\lettrine[lines=1]{\bfseries \textcolor{red}{E}}{} \Large sc\'uchanos, Se\~nor, Padre \'unico, y santifica esta agua, criatura tuya, para que los bautizados en ella sean purificados del pecado y renazcan a la vida de los hijos de Dios.

\large {\textcolor{red}{Todos:}}

\Large {\bfseries \textcolor{red}{R/.}} \hspace{0.5cm} Esc\'uchanos, Se\~nor.

\noindent
\Large Santifica esta agua, criatura tuya, para que los bautizados por ella en la muerte y resurrecci\'on de Cristo, respondan a la imagen de tu Hijo.

\large {\textcolor{red}{Todos:}}

\Large {\bfseries \textcolor{red}{R/.}} \hspace{0.5cm} Esc\'uchanos, Se\~nor.

\noindent
\Large Santifica esta agua, criatura tuya, para que el Esp\'iritu Santo, d\'e la vida nueva a tus elegidos y sean miembros de tu pueblo santo.

\large {\textcolor{red}{Todos:}}

\Large {\bfseries \textcolor{red}{R/.}} \hspace{0.5cm} Esc\'uchanos, Se\~nor.

\large {\textcolor{red}{El celebrante toca el agua con la mano derecha y prosigue:}}

\lettrine[lines=1]{\bfseries \textcolor{red}{B}}{}\Large endice \Huge{\textcolor{red}{\ding{64}}} \Large esta agua con la que va\textcolor{red}{(}n\textcolor{red}{)} a ser bautizado\textcolor{red}{(}s\textcolor{red}{)} este\textcolor{red}{(}os\textcolor{red}{)} hijo\textcolor{red}{(}s\textcolor{red}{)} tuyo\textcolor{red}{(}s\textcolor{red}{)} {\bfseries \textcolor{red}{N.}} y {\bfseries \textcolor{red}{N.}} llamado\textcolor{red}{(}s\textcolor{red}{)} al Bautismo por la fe de la Iglesia, a fin de que \'el\textcolor{red}{(}ellos\textcolor{red}{)} alcance\textcolor{red}{(}n\textcolor{red}{)} la vida eterna. Por Jesucristo nuestro Se\~nor.

\large {\textcolor{red}{Todos:}}

\Large {\bfseries \textcolor{red}{R/.}} \hspace{0.5cm} Am\'en.

\Large {\bfseries \textcolor{red}{RENUNCIAS}}

\large {\textcolor{red}{El celebrante amonesta a los padres y padrinos.}}

\lettrine[lines=1]{\bfseries \textcolor{red}{Q}}{}\Large ueridos padres y padrinos: \\ En el sacramento del Bautismo, este\textcolor{red}{(}os\textcolor{red}{)} ni\~no\textcolor{red}{(}s\textcolor{red}{)} que ha\textcolor{red}{(}n\textcolor{red}{)} presentado a la Iglesia va\textcolor{red}{(}n\textcolor{red}{)} a recibir, por el agua y el Esp\'iritu Santo, una nueva vida que brota del amor de Dios.

\noindent
Ustedes, por su parte, deben esforzarse en educarlo\textcolor{red}{(}s\textcolor{red}{)} en la fe, de tal manera que esta vida divina quede preservada del pecado y crezca en \'el\textcolor{red}{(}ellos\textcolor{red}{)} de d\'ia tras d\'ia.

\noindent
Si est\'an dispuestos a aceptar esta obligaci\'on, recordando su propio bautismo, renuncien al pecado y confiesen su fe en Cristo Jes\'us, que es la fe de la Iglesia, por la que esto\textcolor{red}{(}s\textcolor{red}{)} ni\~no\textcolor{red}{(}s\textcolor{red}{)} ser\'a\textcolor{red}{(}n\textcolor{red}{)} bautizado\textcolor{red}{(}s\textcolor{red}{)}.

\Large {\bfseries \textcolor{red}{V/.}} \hspace{0.5cm} ?`Renuncian a Satan\'as?

\large {\textcolor{red}{Padres y Padrinos:}}

\Large {\bfseries \textcolor{red}{R/.}} \hspace{0.5cm} S\'i, renuncio. 

\Large {\bfseries \textcolor{red}{V/.}} \hspace{0.5cm} ?`Renuncian a sus obras?

\large {\textcolor{red}{Padres y Padrinos:}}

\Large {\bfseries \textcolor{red}{R/.}} \hspace{0.5cm} S\'i, renuncio. 

\Large {\bfseries \textcolor{red}{V/.}} \hspace{0.5cm} ?`Renuncian a todas sus seducciones?

\large {\textcolor{red}{Padres y Padrinos:}}

\Large {\bfseries \textcolor{red}{R/.}} \hspace{0.5cm} S\'i, renuncio. 

\Large {\bfseries \textcolor{red}{PROFESI\'ON DE FE}} 

\large {\textcolor{red}{Seguidamente el celebrante pide esta triple profesi\'on de fe a los padres y padrinos:}} 

\Large {\bfseries \textcolor{red}{V/.}} \hspace{0.5cm} ?`Creen en Dios, Padre todopoderoso, Creador del cielo y de la tierra?

\large {\textcolor{red}{Padres y Padrinos:}}

\Large {\bfseries \textcolor{red}{R/.}} \hspace{0.5cm} S\'i, creo. 

\newpage

\Large {\bfseries \textcolor{red}{V/.}} \hspace{0.5cm} ?`Creen en Jesucristo, su \'unico Hijo, nuestro Se\~nor, que naci\'o de Santa Mar\'ia Virgen, muri\'o, fue sepultado, resucit\'o de entre los muertos y est\'a sentado a la derecha del Padre?

\large {\textcolor{red}{Padres y Padrinos:}}

\Large {\bfseries \textcolor{red}{R/.}} \hspace{0.5cm} S\'i, creo. 

\Large {\bfseries \textcolor{red}{V/.}} \hspace{0.5cm} ?`Creen en el Esp\'iritu Santo, en la Santa Iglesia Cat\'olica, en la comuni\'on de los Santos, en el perd\'on de los pecados, en la resurrecci\'on de los muertos y en la vida eterna?

\large {\textcolor{red}{Padres y Padrinos:}}

\Large {\bfseries \textcolor{red}{R/.}} \hspace{0.5cm} S\'i, creo. 

\large {\textcolor{red}{A esta profesi\'on de fe asiente el celebrante y la comunidad, diciendo:}}

\noindent
\Large \'Esta es nuestra fe. \'Esta es la fe de la Iglesia, que nos gloriamos de profesar en Cristo Jes\'us, Se\~nor nuestro.

\large {\textcolor{red}{Todos:}}

\Large {\bfseries \textcolor{red}{R/.}} \hspace{0.5cm} Am\'en. 

\Large {\bfseries \textcolor{red}{BAUTISMO}} 

\large {\textcolor{red}{El celebrante invita a la primera familia para que se acerque a la fuente. Despu\'es de conocer el nombre del ni\~no, pregunta a los padres y padrinos:}} 

\Large {\bfseries \textcolor{red}{V/.}} \hspace{0.5cm} \Large {?`Quieren que {\bfseries \textcolor{red}{N.}} reciba el Bautizmo, por la fe de la Iglesia, que todos juntos acabamos de profesar?}

\large {\textcolor{red}{Padres y Padrimos:}}

\Large {\bfseries \textcolor{red}{R/.}} \hspace{0.5cm} S\'i, quieremos. 

\large {\textcolor{red}{E inmediatamente el celebrante bautiza al ni\~no diciendo:}}

\noindent
\LARGE{ \bfseries \textcolor{red}{N.}, yo te bautizo en el nombre del Padre,}

\large {\textcolor{red}{Primera inmersi\'on o infusi\'on de agua.}}

\noindent
\LARGE {\bfseries y del Hijo,}

\large {\textcolor{red}{Segunda inmersi\'on o infusi\'on de agua.}}

\noindent
\LARGE {\bfseries y del Esp\'iritu Santo.}

\large {\textcolor{red}{Tercera inmersi\'on o infusi\'on de agua.}}

\large {\textcolor{red}{De modo semejante pregunta y hace con cada uno de los bautizandos.}}

\Large {\bfseries \textcolor{red}{UNCI\'ON CON EL SANTO CRISMA}} 

\lettrine[lines=1]{\bfseries \textcolor{red}{D}}{}\Large ios todopoderoso, Padre de nuestro Se\~nor Jesucristo, que te\textcolor{red}{(}los\textcolor{red}{)} ha liberado del pecado y te\textcolor{red}{(}los\textcolor{red}{)} incorpor\'o a la nueva vida por el agua y el Esp\'iritu Santo, te\textcolor{red}{(}los\textcolor{red}{)} consagre con el crisma de la salvaci\'on para que entres a formar parte de su pueblo y seas\textcolor{red}{(}sean\textcolor{red}{)} para siempre miembro\textcolor{red}{(}s\textcolor{red}{)} de Cristo, celebrante, profeta y rey.

\large {\textcolor{red}{Todos:}} 

\Large {\bfseries \textcolor{red}{R/.}} \hspace{0.5cm} Am\'en. 

\large {\textcolor{red}{Seguidamente, en silencio, el Sacerdote unge la coronilla de la ni\~no con el santo crisma.}}

\newpage

\Large {\bfseries \textcolor{red}{IMPOSICI\'ON DE LA VESTIDURA BLANCA}} 

\noindent
\Large {{\bfseries \textcolor{red}{N.}} y {\bfseries \textcolor{red}{N.}}, eres\textcolor{red}{(}son\textcolor{red}{)} ya nueva creatura y has\textcolor{red}{(}n\textcolor{red}{)} sido revestido de Cristo. Esta vestidura blanca sea signo de tu\textcolor{red}{(}su\textcolor{red}{)} dignidad de cristiano. Ayudado\textcolor{red}{(}s\textcolor{red}{)} por la palabra y el ejemplo de sus familiares, logres\textcolor{red}{(}n\textcolor{red}{)} mantenerla sin mancha hasta la vida eterna. }

\large {\textcolor{red}{Todos:}} 

\Large {\bfseries \textcolor{red}{R/.}} \hspace{0.5cm} Am\'en. 

\Large {\bfseries \textcolor{red}{ENTREGA DEL CIRIO}} 

\large {\textcolor{red}{El celebrante muestra el cirio pascual y dice:}} 

\Large {\bfseries \textcolor{red}{V/.}} \hspace{0.5cm} Recibe\textcolor{red}{(}an\textcolor{red}{)} la luz de Cristo. 

\large {\textcolor{red}{El padrino, enciende la vela del ni\~no en el cirio pascual.}} 

\lettrine[lines=1]{\bfseries \textcolor{red}{A}}{} \Large ustedes, padres y padrinos, se les conf\'ia acrecentar esta luz.\\
Que vuestra este\textcolor{red}{(}os\textcolor{red}{)} ni\~no\textcolor{red}{(}s\textcolor{red}{)}, iluminado\textcolor{red}{(}s\textcolor{red}{)} por Jesucristo, camine siempre como hijo\textcolor{red}{(}s\textcolor{red}{)} de la luz. Y perseverando en la fe, pueda\textcolor{red}{(}n\textcolor{red}{)} salir con todos los Santos al encuentro del Se\~nor. 

\Large {\bfseries \textcolor{red}{ \'EFFETA}} 

\large {\textcolor{red}{El celebrante , tocando con el dedo pulgar los o\'idos y la boca del ni\~no, dice:}} 

\lettrine[lines=1]{\bfseries \textcolor{red}{E}}{}\Large l Se\~nor Jes\'us, que hizo o\'ir a los sordos y hablar a los mudos, te conceda, a su tiempo, escuchar su Palabra y proclamar la fe, para alabanza y gloria de Dios Padre.

\large {\textcolor{red}{Todos:}}

\Large {\bfseries \textcolor{red}{R/.}} \hspace{0.5cm} Am\'en.

\Large {\bfseries \textcolor{red}{RITO DE LA PAZ}}

\large {\textcolor{red}{Despu\'es, a no ser que el Bautismo haya tenido lugar en el mismo presbiterio, se va procesionalmente al altar llevando encendidos los cirios de los bautizados.}}


\Large {\bfseries \textcolor{red}{V/.}} \hspace{0.5cm} La paz del Se\~nor est\'e siempre con ustedes.

\Large {\bfseries \textcolor{red}{R/.}} \hspace{0.5cm} Y con tu esp\'iritu. 

\Large {\bfseries \textcolor{red}{V/.}} \hspace{0.5cm} Dense fraternalmente la paz. 

%SALTO DE PAGINA%%%
\newpage

\begin{center}
\Huge {\bfseries Lit\'ugia Eucar\'istica}
\end{center}

\lettrine[lines=1]{\bfseries \textcolor{red}{O}}{} \Large ren, hermanos, para que este sacrificio, m\'io y de ustedes, sea agradable a Dios, Padre todopoderoso. 

\Large {\bfseries \textcolor{red}{R/.}} \hspace{0.5cm} El Se\~nor reciba de tus manos este sacrificio para alabanza y gloria de su nombre, para nuestro bien y el de toda su santa Iglesia. 

\large {\textcolor{red}{Luego el celebrante, con las manos extendidas, dice la oraci\'on sobre las ofrendas. Al final el
pueblo aclama:}}

\Large {\bfseries \textcolor{red}{R/.}} \hspace{0.5cm} Am\'en.

\begin{center}
\Large PLEGARIA EUCARISTICA
\end{center}


\Large {\bfseries \textcolor{red}{PREFACIO}}

\Large {\bfseries \textcolor{red}{V/.}} \hspace{0.5cm} El Se\~nor est\'e con vosotros.

\Large {\bfseries \textcolor{red}{R/.}} \hspace{0.5cm} Y con tu esp\'iritu. 


\Large {\bfseries \textcolor{red}{V/.}} \hspace{0.5cm} Levantemos el coraz\'on.

\Large {\bfseries \textcolor{red}{R/.}} \hspace{0.5cm} Lo tenemos levantado hacia el Se\~nor. 


\Large {\bfseries \textcolor{red}{V/.}} \hspace{0.5cm} Demos gracias al Se\~nor, nuestro Dios.

\Large {\bfseries \textcolor{red}{R/.}} \hspace{0.5cm} Es justo y necesario.

\newpage

\lettrine[lines=1]{\bfseries \textcolor{red}{E}}{}\Large n verdad es justo y necesario,\\
    es nuestro deber y salvaci\'on\\
    darte gracias\\
    siempre y en todo lugar,\\
    Se\~nor, Padre santo,\\
    Dios todopoderoso y eterno.

\noindent
\Large Porque gracias al misterio\\
    de la Palabra hecha carne,\\
    la luz de tu gloria brill\'o ante nuestros ojos\\
    con nuevo resplandor\\
    para que conociendo a Dios visiblemente,\\
    \'el nos lleve al amor de lo invisible.

\lettrine[lines=1]{\bfseries \textcolor{red}{P}}{}\Large or eso,\\
    con los \'angeles y arc\'angeles\\
    y con todos los coros celestiales,\\
    cantamos sin cesar\\
    el himno de tu gloria:

\large{\textcolor{red}{Al final del Prefacio, junta las manos y en uni\'on del pueblo, concluye el prefacio, cantando o diciendo en voz alta:}}

\noindent
\Large {Santo, Santo, Santo es el Se\~nor, Dios del universo.\\
Llenos est\'an el cielo y la tierra de tu gloria.\\
Hosanna en el cielo.\\
Bendito el que viene en nombre del Se\~nor.\\
Hosanna en el cielo.}

%SALTO DE PAGINA%%%
\newpage

\Large {\bfseries \textcolor{red}{PREFACIO DEL BAUTISMO}} 

\Large {\bfseries \textcolor{red}{V/.}} \hspace{0.5cm} El Se\~nor est\'e con vosotros.

\Large {\bfseries \textcolor{red}{R/.}} \hspace{0.5cm} Y con tu esp\'iritu. 

\Large {\bfseries \textcolor{red}{V/.}} \hspace{0.5cm} Levantemos el coraz\'on.

\Large {\bfseries \textcolor{red}{R/.}} \hspace{0.5cm} Lo tenemos levantado hacia el Se\~nor. 

\Large {\bfseries \textcolor{red}{V/.}} \hspace{0.5cm} Demos gracias al Se\~nor, nuestro Dios.

\Large {\bfseries \textcolor{red}{R/.}} \hspace{0.5cm} Es justo y necesario.

\lettrine[lines=1]{\bfseries \textcolor{red}{E}}{}\Large n verdad es justo darte gracias, \\ 
y exaltar tu nombre \\ 
Padre Santo y misericordioso, \\
por Jesucristo, Se\~nor y redentor nuestro.

\noindent
Te alabamos, \\
te bendecimos y te glorificamos \\
por el sacramento del nuevo nacimiento.

\noindent
T\'u has querido que del coraz\'on abierto de tu Hijo \\
manara para nosotros el don nupcial de Bautismo, \\
primera pascua de los creyentes, \\
puerta de nuestra salvaci\'on. \\
Inicio de la vida en Cristo, \\
fuente de la humanidad nueva.

\newpage

\noindent
Del agua y del Esp\'iritu \\
engendraste en el seno de la Iglesia, virgen y madre, \\
un pueblo de sacerdotes y reyes, \\
congregado de entre todas las naciones \\
en la unidad y santidad del amor.

\lettrine[lines=1]{\bfseries \textcolor{red}{P}}{}\Large or este don de tu benevolencia \\
tu familia te adora \\
y, unida a los angeles y los santos, \\
cantan el himno de tu gloria:

\large{\textcolor{red}{Al final del Prefacio, junta las manos y en uni\'on del pueblo, concluye el prefacio, cantando o diciendo en voz alta:}}

\noindent
\Large {Santo, Santo, Santo es el Se\~nor, Dios del universo.\\
Llenos est\'an el cielo y la tierra de tu gloria.\\
Hosanna en el cielo.\\
Bendito el que viene en nombre del Se\~nor.\\
Hosanna en el cielo.}

%SALTO DE PAGINA%%%
\newpage

\Large {\bfseries \textcolor{red}{CONSAGRACI\'ON}} 

\large{\textcolor{red}{El celebrante, con las manos extendidas, dice:}}

\lettrine[lines=1]{\bfseries \textcolor{red}{S}}{}\Large anto eres en verdad, Se\~nor,\\
fuente de toda santidad;

\large{\textcolor{red}{Junta las manos y, manteni\'endolas extendidas sobre las ofrendas, dice:}}

\noindent
\Large por eso te pedimos que santifiques estos dones\\
con la efusi\'on de tu Esp\'iritu,

\large{\textcolor{red}{Junta las manos y traza el signo de la cruz sobre el pan y el c\'aliz conjuntamente, diciendo:}}

\noindent
\Large de manera que se conviertan para nosotros\\
en el Cuerpo y \Huge{\textcolor{red}{\ding{64}}} \Large Sangre\\
de Jesucristo, nuestro Se\~nor.

\large{\textcolor{red}{Junta las manos.}}

\lettrine[lines=1]{\bfseries \textcolor{red}{E}}{}\Large l cual,\\
cuando iba a ser entregado a su Pasi\'on,\\
voluntariamente aceptada,

\large{\textcolor{red}{Toma el pan y, sosteni\'endolo un poco elevado sobre el altar, prosigue:}}

\noindent
\Large tom\'o pan, d\'andote gracias, lo parti\'o\\
y lo dio a sus disc\'ipulos, diciendo:

\large{\textcolor{red}{Se inclina un poco.}} 

\noindent
\LARGE{ \bfseries{ Tomen y coman todos de \'el,\\
porque esto es mi Cuerpo,\\
que ser\'a entregado por ustedes.}}

\large{\textcolor{red}{Muestra el pan consagrado al pueblo, lo deposita luego sobre la patena y lo adora haciendo genuflexi\'on. Despu\'es prosigue:}}

\noindent
\Large Del mismo modo, acabada la cena,

\large{\textcolor{red}{Toma el c\'aliz y, sosteni\'endolo un poco elevado sobre el altar, prosigue:}}

\noindent
\Large tom\'o el c\'aliz,\\
y, d\'andote gracias de nuevo,\\
lo pas\'o a sus disc\'ipulos, diciendo:

\large{\textcolor{red}{Se inclina un poco.}}

\noindent
\LARGE{ \bfseries{ Tomen y beban todos de \'el,\\
porque este es el c\'aliz de mi Sangre,\\
Sangre de la alianza nueva y eterna,\\
que ser\'a derramada\\
por ustedes y por muchos\\
para el perdon de los pecados.\\
Hagan esto en conmemoraci\'on m\'ia.}}

\large{\textcolor{red}{Muestra el c\'aliz al pueblo, lo deposita luego sobre el corporal y lo adora haciendo genuflexi\'on.}}

\Large {\bfseries \textcolor{red}{V/.}} \hspace{0.5cm} \'Este es el Sacramento de nuestra f\'e.

\Large {\bfseries \textcolor{red}{R/.}} \hspace{0.5cm} Anunciamos tu muerte, Se\~nor,\\
. \hspace{1.5cm} proclamamos tu resurrecci\'on.\\
. \hspace{1.5cm} Ven, Se\~nor Jes\'us!

\large{\textcolor{red}{Despu\'es el celebrante, con las manos extendidas, dice:}} 

\Large {\bfseries \textcolor{red}{ANAMNESIS}}

\lettrine[lines=1]{\bfseries \textcolor{red}{A}}{}\Large s\'i, pues, Padre,\\
al celebrar ahora el memorial\\
de la muerte y resurrecci\'on de tu Hijo,\\
te ofrecemos el pan de vida y el c\'aliz de salvaci\'on,\\
y te damos gracias\\
porque nos haces dignos de servirte en tu presencia.

\lettrine[lines=1]{\bfseries \textcolor{red}{T}}{}\Large e pedimos humildemente\\
que el Esp\'iritu Santo congregue en la unidad\\
a cuantos participamos\\
del Cuerpo y Sangre de Cristo.

\noindent
Acu\'erdate, Se\~nor,\\
de tu Iglesia extendida por toda la tierra; \\
y con el Papa {\bfseries \textcolor{red}{N.}}, con nuestro Obispo {\bfseries \textcolor{red}{N.}},\\
y todos los pastores que cuidan de tu pueblo,\\
llevala a su perfecci\'on por la caridad.

\noindent
Acu\'erdate tambi\'en de nuestros hermanos {\bfseries \textcolor{red}{N.}} y {\bfseries \textcolor{red}{N.}} \\
que hoy has hecho renacer \\
del agua y del Esp\'iritu Santo, \\
libr\'andolo\textcolor{red}{(}s\textcolor{red}{)} del pecado; \\
t\'u que lo\textcolor{red}{(}s\textcolor{red}{)} has incorporado, \\
como miembro vivo\textcolor{red}{(}s\textcolor{red}{)}, al cuerpo de Cristo, \\
inscribe tambi\'en su\textcolor{red}{(}s\textcolor{red}{)} nombre\textcolor{red}{(}s\textcolor{red}{)} en el libro de la vida.

\lettrine[lines=1]{\bfseries \textcolor{red}{A}}{}\Large cu\'erdate tambi\'en de nuestros hermanos\\
que se durmieron en la esperanza\\
de la resurrecc\'ion,\\
y de todos los que han muerto en tu misericordia;\\
admitelos a contemplar la luz de tu rostro.

\noindent
Ten misericordia de todos nosotros,\\
y as\'i, con Mar\'a, la Virgen Madre de Dios,\\
San Jos\'e su Casto Esposo, \\
los ap\'ostoles\\
y cuantos vivieron en tu amistad\\
a trav\'es de los tiempos,\\
merezcamos, por tu Hijo Jesucristo,\\
compartir la vida eterna\\
y cantar tus alabanzas. 

\large{\textcolor{red}{Junta las manos. Toma la patena con el pan consagrado y el c\'aliz y, sosteni\'endolos elevados, dice:}}

\lettrine[lines=1]{\bfseries \textcolor{red}{P}}{}\Large or Cristo, con \'el y en \'el,\\
a ti, Dios Padre omnipotente,\\
en la unidad del Esp\'iritu Santo,\\
todo honor y toda gloria\\
por los siglos de los siglos.

\Large {\bfseries \textcolor{red}{R/.}} \hspace{0.5cm} Am\'en.

\newpage

\begin{center}
\Huge {\bfseries Rito de la Comuni\'on}
\end{center}

\lettrine[lines=1]{\bfseries \textcolor{red}{H}}{}\Large ermanos: \\
Este\textcolor{red}{(}os\textcolor{red}{)} ni\~no\textcolor{red}{(}s\textcolor{red}{)} nacido\textcolor{red}{(}s\textcolor{red}{)} de nuevo por el Bautismo, se llama\textcolor{red}{(}n\textcolor{red}{)} y es\textcolor{red}{(}son\textcolor{red}{)} hijo\textcolor{red}{(}s\textcolor{red}{)} de Dios. Un d\'ia recibir\'a\textcolor{red}{(}n\textcolor{red}{)} por la Confirmaci\'on la plenitud del Esp\'iritu Santo. Se acercar\'a\textcolor{red}{(}n\textcolor{red}{)} al altar del Se\~nor, participar\'a\textcolor{red}{(}n\textcolor{red}{)} en la mesa de su sacrificio y lo invocar\'a\textcolor{red}{(}n\textcolor{red}{)} como Padre en medio de su Iglesia. Ahora nosotros, en nombre de este\textcolor{red}{(}os\textcolor{red}{)} ni\~no\textcolor{red}{(}s\textcolor{red}{)} por el esp\'iritu de adopci\'on que todos hemos recibido, oremos juntos como Cristo nos ense\~n\'o.

\large{\textcolor{red}{Extiende las manos y, junto con el ministro, continua:}}

\noindent
\Large Padre nuestro, que est\'as en el cielo,\\
santificado sea tu nombre;\\
venga a nosotros tu reino;\\
h\'agase tu voluntad en la tierra como en el cielo.\\
Danos hoy nuestro pan de cada d\'ia;\\
perdona nuestras ofensas,\\
como nosotros perdonamos\\
a los que nos ofenden;\\
no nos dejes caer en la tentaci\'on,\\
y l\'ibranos del mal.

\large{\textcolor{red}{Solo el celebrante, con las manos extendidas, prosigue diciendo:}}

\noindent
\Large L\'ibramos Se\~nor de todos lo males,\\ 
y conc\'edenos la paz en nuestros d\'ias,\\ 
para que, ayudados por tu misericordia,\\ 
vivamos siempre libres de pecado\\ 
y protegidos de toda perturbaci\'on,\\ 
mientras esperamos la gloriosa venida\\ 
de nuestro Salvador Jesucristo.

\large{\textcolor{red}{Solo el celebrante, con las manos extendidas, prosigue diciendo:}}

\noindent
\Large L\'ibramos Se\~nor de todos lo males, y conc\'edenos la paz en nuestros d\'ias, para que, ayudados por tu misericordia, vivamos siempre libres de pecado y protegidos de toda perturbaci\'on, mientras esperamos la gloriosa venida de nuestro Salvador Jesucristo.

\noindent
\Large {\bfseries \textcolor{red}{R/.}} \hspace{0.5cm} Tuyo es el reino, tuyo es el poder y la gloria, por siempre, Se\~nor.

\large{\textcolor{red}{Despu\'es el sacerdote, con las manos extendidas, dice en voz alta:}}

\noindent
\Large Se\~nor Jesucristo, que dijiste a tus Ap\'ostoles: <<La paz les dejo, mi paz les doy>>, no tengas en cuenta nuestros pecados, sino la fe de tu Iglesia, y conforme a tu palabra, conc\'edele la paz y la unidad.

\large{\textcolor{red}{Junta las manos.}}

\noindent
\Large T\'u que vives y reinas por los siglos de los siglos.

\Large {\bfseries \textcolor{red}{R/.}} \hspace{0.5cm} Am\'en.

\large{\textcolor{red}{Mientras el celebrante tama el pan consagrado, lo parte sobre la patena y pone una part\'icula dentro del caliz.}}

\noindent
\Large Cordero de Dios que quitas el pecado del mundo,\\
ten piedad de nosotros.

\noindent
\Large Cordero de Dios que quitas el pecado del mundo,\\ 
ten piedad de nosotros.

\noindent
\Large Cordero de Dios que quitas el pecado del mundo,\\ 
danos la paz.

\large{\textcolor{red}{El celebrante hace genuflexi\'on, toma el pan consagrado y, sosteni\'endolo un poco elevado, dice con voz clara:}}

\noindent
\Large \'Este es el Cordero de Dios,\\ 
que quita el pecado del mundo.\\ 
Dichosos los invitados a la cena del Se\~nor.

\large{\textcolor{red}{Y, juntamente con el pueblo, a\~nade:}}

\noindent
\Large Se\~nor, no soy digno de que entres en mi casa,\\ 
pero una palabra tuya bastar\'a para sanarme.

\Large {\bfseries \textcolor{red}{R/.}} \hspace{0.5cm} Am\'en.

\large{\textcolor{red}{Cuando el sacerdote comulga el Cuerpo de Cristo, comienza el canto de comuni\'on.}}

\large{\textcolor{red}{Despu\'es el sacerdote puede ir a la sede. Si se juzga oportuno, se pueden guardar unos momentos de silencio o cantar un salmo o c\'antico de alabanza. Luego, de pie en la sede o en el altar, el sacerdote dice:}}

\noindent
\Large Oremos.

\large {\bfseries \textcolor{red}{ANT\'IFONA DE COMUNI\'ON \hspace{1cm} Bar 3, 38}}

\noindent
\Large Nuestro Dios apareci\'o en el mundo y convivi\'o con los hombres.

\large {\bfseries \textcolor{red}{ORACI\'ON DESPU\'ES DE LA COMUNI\'ON}}

\lettrine[lines=1]{\bfseries \textcolor{red}{P}}{}\Large adre misericordioso, haz que reanimados con este sacramento celestial, imitemos constantemente los ejemplos de la Sagrada Familia, para que, superadas las aflicciones de esta vida, consigamos gozar eternamente de su compa\~n\'ia. Por Jesucristo, nuestro Se\~nor.

\Large {\bfseries \textcolor{red}{R/.}} \hspace{0.5cm} Am\'en.


%SALTO DE PAGINA%%%
\newpage

\begin{center}
\Huge {\bfseries Rito de Concluci\'on}
\end{center}

\large{\textcolor{red}{En este momento se hacen, si es necesario y con brevedad, los oportunos anuncios o advertencias al pueblo.}}

\Large {\bfseries \textcolor{red}{BENDICI\'ON SOLEMNE AL FINAL DE LA MISA}}

\large{\textcolor{red}{Seguidamente el Sacerdote bendice a la madre y padre del(de los) y a todos los presentes, diciendo:}}

\lettrine[lines=1]{\bfseries \textcolor{red}{E}}{}\Large l Se\~nor todopoderoso, por su Hijo, nacido de Mar\'ia la Virgen, bendiga a estas madres y alegre su coraz\'on con la esperanza de la vida eterna, alumbrada hoy en sus hijos, para que del mismo modo que le agradecen el fruto de sus entra\~nas, perseveren con ellos en constante acci\'on de gracias. Por Jesucristo nuestro Se\~nor.

\Large {\bfseries \textcolor{red}{R/.}} \hspace{0.5cm} Am\'en.

\noindent
\Large El Se\~nor todopoderoso, dispensador de la vida temporal y la eterna, bendiga a estos padres, que junto con sus esposas, sean los primeros que, de palabra y obra, den testimonio de la fe ante su\textcolor{red}{(}s\textcolor{red}{)} hijo\textcolor{red}{(}s\textcolor{red}{)}, en Jesucristo nuestro Se\~nor.

\Large {\bfseries \textcolor{red}{R/.}} \hspace{0.5cm} Am\'en.

\noindent
\Large El Se\~nor Todopoderoso, que nos ha hecho renacer a la vida eterna por el agua y el Esp\'iritu Santo, bendiga a \'estos fieles, para que siempre y en todo lugar sean miembros vivos de su pueblos; y conceda la abundancia de su paz a todos los aqu\'i presentes, en Jesucristo nuestro Se\~nor.

\Large {\bfseries \textcolor{red}{R/.}} \hspace{0.5cm} Am\'en.

\newpage

\lettrine[lines=1]{\bfseries \textcolor{red}{Y}}{} \Large la bendici\'on de Dios todopoderoso, \\
Padre, Hijo \Huge{\textcolor{red}{\ding{64}}} \Large y Esp\'iritu Santo, \\
descienda sobre vosotros y os acompa\~ne siempre.

\Large {\bfseries \textcolor{red}{R/.}} \hspace{0.5cm} Am\'en.

\large{\textcolor{red}{Luego el celebrante, con las manos juntas, despide al pueblo:}}

\lettrine[lines=1]{\bfseries \textcolor{red}{P}}{}\Large ueden ir en paz.

\Large {\bfseries \textcolor{red}{R/.}} \hspace{0.5cm} Demos gracias a Dios.

\large{\textcolor{red}{Despu\'es el celebrante besa con veneraci\'on el altar, como al comienzo, y, hecha la debida reverencia con los ministros, se retira a la sacrist\'ia.}}

% Termina el documento %%%%%%%%%%%%%%%%%%%%%%%%%%%%% F I N %%%%%%%%%%%%%%%%%%%%%%%%%%%%%%%%%%%%%%%%%%%%%%%%%%%%%%
\end{document}
