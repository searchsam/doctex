\documentclass[oneside,numbers]{ezthesis}
%% # Opciones disponibles para el documento #
%%
%% Las opciones con un (*) son las opciones predeterminadas.
%%
%% Modo de compilar:
%%   draft            - borrador con marcas de fecha y sin im'agenes
%%   draftmarks       - borrador con marcas de fecha y con im'agenes
%%   final (*)        - version final de la tesis
%%
%% Tama'no de papel:
%%   letterpaper (*)  - tama'no carta (Am'erica)
%%   a4paper          - tama'no A4    (Europa)
%%
%% Formato de impresi'on:
%%   oneside          - hojas impresas por un solo lado
%%   twoside (*)      - hijas impresas por ambos lados
%%
%% Tama'no de letra:
%%   10pt, 11pt, o 12pt (*)
%%
%% Espaciado entre renglones:
%%   singlespace      - espacio sencillo
%%   onehalfspace (*) - espacio de 1.5
%%   doublespace      - a doble espacio
%%
%% Formato de las referencias bibliogr'aficas:
%%   numbers          - numeradas, p.e. [1]
%%   authoryear (*)   - por autor y a'no, p.e. (Newton, 1997)
%%
%% Opciones adicionales:
%%   spanish         - tesis escrita en espa'nol
%%
%% Desactivar opciones especiales:
%%   nobibtoc   - no incluir la bibiolgraf'ia en el 'Indice general
%%   nofancyhdr - no incluir "fancyhdr" para producir los encabezados
%%   nocolors   - no incluir "xcolor" para producir ligas con colores
%%   nographicx - no incluir "graphicx" para insertar gr'aficos
%%   nonatbib   - no incluir "natbib" para administrar la bibliograf'ia

%% Paquetes adicionales requeridos se pueden agregar tambi'en aqu'i.
%% Por ejemplo:
%\usepackage{subfig}
%\usepackage{multirow}
%\usepackage[spanish,activeacute]{babel}
\renewcommand{\contentsname}{Contenido}
\renewcommand{\partname}{Parte}
\renewcommand{\indexname}{Lista Alfab\'etica}
\renewcommand{\appendixname}{Ap\'endice}
\renewcommand{\figurename}{Figura}
\renewcommand{\listfigurename}{Lista de Figuras}
\renewcommand{\tablename}{Tabla}
\renewcommand{\listtablename}{Lista de Tablas}
\renewcommand{\abstractname}{Resumen}
\renewcommand{\chaptername}{Cap\'itulo}
\renewcommand{\refname}{Bibliograf\'ia}

%% # Datos del documento #
%% Nota que los acentos se deben escribir: \'a, \'e, \'i, etc.
%% La letra n con tilde es: \~n.

\author{
	Federico Alfonso Matus Olivares\\
	Norma Steffani Cerrato\\
	Samuel Jos\'e Guti\'errez Avil\'es
}
\title{Interfaces responsivas para la plataforma LRN de la UNI Online.}
\degree{Ingeniero en Computaci\'on}
\supervisor{Nombre de mi Asesor}
\institution{Universidad Nacional de Ingenieria}
\faculty{Facultad de Electrotecnia y Computac\'on}
\department{Departamento de Lenguajes y Simulaci\'on}

%% # M'argenes del documento #
%% 
%% Quitar el comentario en la siguiente linea para austar los m'argenes del
%% documento. Leer la documentaci'on de "geometry" para m'as informaci'on.

%\geometry{top=40mm,bottom=33mm,inner=40mm,outer=25mm}

%% El siguiente comando agrega ligas activas en el documento para las
%% referencias cruzadas y citas bibliogr'aficas. Tiene que ser *la 'ultima*
%% instrucci'on antes de \begin{document}.
\hyperlinking
\begin{document}

%% En esta secci'on se describe la estructura del documento de la tesis.
%% Consulta los reglamentos de tu universidad para determinar el orden
%% y la cantidad de secciones que debes de incluir.

%% # Portada de la tesis #
%% Mirar el archivo "titlepage.tex" para los detalles.
\include{titlepage}

%% # Prefacios #
%% Por cada prefacio (p.e. agradecimientos, resumen, etc.) crear
%% un nuevo archivo e incluirlo aqu'i.
%% Para m'as detalles y un ejemplo mirar el archivo "gracias.tex".
%% \include{gracias}

%% # 'Indices y listas de contenido #
%% Quitar los comentarios en las lineas siguientes para obtener listas de
%% figuras y cuadros/tablas.
\tableofcontents
%\listoffigures
%\listoftables

%% # Cap'itulos #
%% Por cada cap'itulo hay que crear un nuevo archivo e incluirlo aqu'i.
%% Mirar el archivo "intro.tex" para un ejemplo y recomendaciones para
%% escribir.
%% Los cap\'itulos inician con \chapter{T\'itulo}, estos aparecen numerados y
%% se incluyen en el \'indice general.
%%
%% Recuerda que aqu\'i ya puedes escribir acentos como: \'a, \'e, \'i, etc.
%% La letra n con tilde es: \'n.

\chapter{Introducci\'on}

El presente protocolo tiene como prop\'osito brindar acceso a la herramienta educativa con el uso de dispositivos m\'oviles, ya que estos se han convertido en un medio sumamente accesible y popular para la interacci\'on social. Por tanto se pretende usar este medio con el prop\'ositos de permitir al estudiante mayor accesibilidad desde la web.

Existen dos tipos de citas bibliograf\'icas: usa \verb|\citep{..}| para
citas en \emph{par\'entesis} y \verb|\citet{..}| para citas
en el \emph{texto}. Por ejemplo, estudios reciente han mostrado nuevos e
interesantes modelos que se pueden aplicar para reformular teor\'ias
f\'isicas~\citep{NewCam97}. Mientras que, el trabajo de \citet{Rofl06} fue
considerado muy divertido por una significativa fracci\'on de la comunidad
de investigadores. Tambi\'en es posible citar a varios trabajos en una sola
referencia \citep{Lamport86,Knuth84}.

Estos comandos para producir citas bibliograficas son provistos por
el paquete \textsf{natbib}. Para obtener m\'as informaci\'on, consulta la
documentaci\'on de ese paquete~\citep{doc:natbib}. Por su parte, en
la documentaci\\'on de \textsf{geometry} puedes encontrar detalles
adicionales sobre el sistema para ajustar los m\'argenes del
documento~\citep{doc:geometry}. Lo que sigue
es un mont\'on de texto sin sentido en lat\'in que utilizaremos para llenar
algunas p\'aginas.

%% Los cap\'itulos inician con \chapter{T\'itulo}, estos aparecen numerados y
%% se incluyen en el \'indice general.
%%
%% Recuerda que aqu\'i ya puedes escribir acentos como: \'a, \'e, \'i, etc.
%% La letra n con tilde es: \'n.

\chapter{Justificaci\'on}

La UNI Online, como centro centro tecnol\'ogico esta interesado en el uso de los entornos m\'oviles con capacidad de grabar, reproducir, navegar en la WEB editar e intercambiar documentos, adem\'as de las funciones tradicionales de comunicaci\'on uno a uno y en redes sociales; aprovecharlas como medio masivo de acceso a las herramientas educativas de la misma.

%% Los cap\'itulos inician con \chapter{T\'itulo}, estos aparecen numerados y
%% se incluyen en el \'indice general.
%%
%% Recuerda que aqu\'i ya puedes escribir acentos como: \'a, \'e, \'i, etc.
%% La letra n con tilde es: \'n.

\chapter{Objetivos}

\section{Objetivo general}
Implementaci\'on de interfaces responsivas para la plataforma LRN de la UNI Online.

\section{Objetivos espec\'ificos}
	\begin{itemize}
		\item Determinar los requerimientos necesario para la implementaci\'on de tecnolog\'ias responsivas en las interfaces de la plataforma LRN de la UNI Online.
		\item Redise\~no de la plataforma LRN de la UNI Online para la adaptabilidad en dispositivos m\'oviles.
		\item Desarrollar interfaz gr\'afica responsiva para plataforma LRN de la UNI Online.
	\end{itemize}

%% Los cap\'itulos inician con \chapter{T\'itulo}, estos aparecen numerados y
%% se incluyen en el \'indice general.
%%
%% Recuerda que aqu\'i ya puedes escribir acentos como: \'a, \'e, \'i, etc.
%% La letra n con tilde es: \'n.

\chapter{Antecedentes}

El proyecto de aprendizaje m\'ovil llevado a cabo en el Tecnol\'ogico de Monterrey tuvo una cobertura inicial de 1237 alumnos de primer semestre de las 33 carreras ofrecidas en ese periodo por la Universidad, con una oferta de 34 materias cubriendo 84 grupos. A dos a\~nos de su inicio el incremento observado ha sido de 47 materias, ahora incluyendo educaci\'on media, atendi\'endose a 362 grupos de ense\~nanza media y 135 de educaci\'on superior, con un total de  3,365 alumnos participantes y 222 profesores.

La estrategia de arranque del proyecto contempl\'o que en la innovaci\'on, los costos relacionados con la incorporaci\'on del equipo a las pr\'acticas de aprendizaje corriera a cargo de la Instituci\'on, esto ocurri\'o mediante una estrategia en la que se entregaron equipos BlackBerry a todos los estudiantes y profesores participantes en el modelo, as\'i como la cobertura de un a\~no de servicios de acceso ilimitado a datos y en la cobertura mínima a llamadas. Para apoyar la incorporaci\'on de los docentes al modelo se dise\~naron talleres donde participaron en la primera generaci\'on, profesores considerados l\'ideres por su apertura al cambio y disponibilidad para participar en estrategias de innovaci\'on y en los subsiguientes todos los profesores impartiendo las materias definidas por la academia.

%% Los cap\'itulos inician con \chapter{T\'itulo}, estos aparecen numerados y
%% se incluyen en el \'indice general.
%%
%% Recuerda que aqu\'i ya puedes escribir acentos como: \'a, \'e, \'i, etc.
%% La letra n con tilde es: \'n.

\chapter{Marco Te\'orico}

\section{Dise\~no web resposivo}
Tambi\'en llamado dise\~no adaptable creado por Ethan Marcotte es una t\'ecnica de dise\~no y desarrollo web que mediante el uso de estructuras e im\'agenes fluidas, as\'i como de media-queries en la hoja de estilo CSS, consigue adaptar el sitio web al entorno del usuario. (W3C, Mobile Web Best Practices - One Web, 2008).

\section{Media queries}

\section{Dispositivos M\'oviles}
Tambi\'en conocidos como computadora de mano, palmtop o simplemente handset) son aparatos de peque\~no tama\~no, con algunas capacidades de procesamiento, con conexió\'on permanente o intermitente a una red, con memoria limitada, dis\~nados espec\'ificamente para una funci\'on, pero que pueden llevar a cabo otras funciones m\'as generales.

\section{LRN}
%\include{conclu}

\appendix
%% Cap'itulos incluidos despues del comando \appendix aparecen como ap'endices
%% de la tesis.
%\include{apendiceA}
%\include{apendiceB}
%\include{apendiceC}

%% Incluir la bibliograf'ia. Mirar el archivo "biblio.bib" para m'as detales
%% y un ejemplo.
\bibliography{biblio}

\end{document}
