% - Preambulo
\documentclass[12pt, letterpaper]{article}

%%% - Paquetes
\usepackage{babel}
\usepackage[utf8]{inputenc}
\usepackage{xcolor}
\usepackage{pifont}
\usepackage{lettrine}
\usepackage{lmodern}

%%% - Definimos la zona de la pagina ocupada por el texto
\oddsidemargin -1.0cm
\headsep -1.0cm
\textwidth=18.5cm
\textheight=23cm

%%% - Definmos el tiempo de salto de linea
\setlength{\parskip}{\baselineskip}

% - Cuerpo
\begin{document}
%%%%%%%%%%%%%%%%%%%% P R I N C I P I O %%%%%%%%%%%%%%%%%%%%%%%%%%%%%%
    \begin{center}
    \Large {\bfseries ORIENTACIONES PARA LA CELEBRACIÓN DEL JUEVES SANTO}
    \end{center}

    \begin{center}
    \large {\bfseries \textcolor{red}{LAVATORIO DE LOS PIES}}
    \end{center}
    \normalsize {\textcolor{red}{Cada comunidad debera de preparar lectores, monitores y cantos adecuados según su etapa de camino.}\\
    \normalsize {\textcolor{red}{Si la comunidad cuenta con presbitero este necesitara:}
    \textcolor{red}{
        \begin{itemize}
            \item Alba.
            \item Capa Pluvial blanca
            \item Cíngulo.
            \item Estola blanca.
        \end{itemize}
    }

    \large {\bfseries Monición Ambiental.}

    \large {\bfseries Canto de Entrada.}\\
    \normalsize {\textcolor{red}{Si la comunidad cuenta con presbitero este entrar durante el canto de entrada a un lugar reservado para el.}

    \large {\bfseries Oración del presidente.}\\
    \normalsize {\textcolor{red}{Si la comunidad cuenta con presbitero este hara una oración segun la formula del Misal Romano sino el responsable solo invita a invocar el Espíritu Santo.}

    \large {\bfseries Invocación al Espíritu Santo.}\\
    \normalsize {\textcolor{red}{Oh Señor, envía tu Espíritu, que renueve la faz de la tierra.\\ Oh Señor, envía tu Espíritu, que renueve la faz de la tierra.}

    \large {\bfseries Primera Lectura.}
    \begin{itemize}
     \item \normalsize Monición
     \item Lectura: Juan 13, 31-14, 31
     \item Canto
    \end{itemize}

    \large {\bfseries Segunda Lectura.}
    \begin{itemize}
     \item \normalsize Monición
     \item Lectura: Juan 15, 1-16, 33
     \item Canto
    \end{itemize}

    \large {\bfseries Tercera Lectura.}
    \begin{itemize}
     \item \normalsize Monición
     \item Lectura: Juan 13, 1-20
    \end{itemize}

    \large {\bfseries Eco de la Palabra.}\\
    \normalsize {\textcolor{red}{Recordar que el eco es manifestar lo que el Espíritu Santo a través de la palabra te ilumina en tu vida centrados en la vivencia del Jueves Santo (Día del Amor Fraterno). No es momento de contar la vida por contarla, ni de pedir perdón nadie. Se debe ser concreto y, de ser posible, breve para que hable el mayor número posible de hermanos.}

    \large {\bfseries Breve Homilía.}\\
    \normalsize {\textcolor{red}{Si no se cuenta con presbitero la homilia se omite. Si la comunidad cuenta con presbitero este hará una breve homilia que terminará introdusiendo el rito del lavado de pies.}

    \large {\bfseries Lavatorio de los Pies.}\\
    \normalsize {\textcolor{red}{Para el lavado de pies se neceita:}
    \textcolor{red}{
        \begin{itemize}
            \item Dos toallas blancas.
            \item Pana grande.
            \item Perfume.
            \item Pichel.
        \end{itemize}
    }

    \normalsize {\textcolor{red}{Si la comunidad cuenta con presbitero este se quitara la capa pluvial y se pondra una toalla a cintura y ayudado del responsable lava los pies a toda la comunidad, si no se cuenta con presbitero el responsable y pondra una toalla a cintura y ayudado por otro miembro del equipo de responsales lavara los pies a la comunidad. Despúes, los hermanos que lo deseen hacen lo mismo con algún otro hermano. No hay que decir nada ni explicar a nadie por qué se le laban los pies. Durante este rito se canta, los cantos continuan hasta que se termine el rito.}

    \normalsize {\textcolor{red}{A continuacon algunos cantos recomendados, tener en cuenta que se debera cantar solo los cantos que correspondan a su estapa de camino:}
    \textcolor{red}{
        \begin{itemize}
            \item Caritas Christi.
            \item Como es maravilloso.
            \item El Señor es mi pastor.
            \item Este es el mandamiento mio.
            \item Gritar jubilosos.
            \item Himno a la Caridad.
            \item Himno a la Kenosis.
            \item Mirad que estupendo.
            \item Por el amor de mis amigos.
            \item Quien nos separara.
            \item Cualquier otro que sea apropiado.
        \end{itemize}
    }

    \large {\bfseries Peticiones.}

    \large {\bfseries Padre Nuestro y Ave María.}

    \large {\bfseries Paz.}

    \large {\bfseries Bendición Final.}\\
    \normalsize {\textcolor{red}{Si la comunidad cuenta con presbitero este hara una bención según el Misal Romano, sino, el responsable hara el cierra según la liturgia de las horas:\\ \bfseries ``El Señor nos bendiga, nos guarde de todo mal y nos lleve a la vida eterna''.}

    \large {\bfseries Canto Final.}\\
    \normalsize {\textcolor{red}{Si la comunidad cuenta con presbitero este saldra durante el canto final.}
%%%%%%%%%%%%%%%%%%%%%%%%%%%%% F I N %%%%%%%%%%%%%%%%%%%%%%%%%%%%%%%%%%
\end{document}
