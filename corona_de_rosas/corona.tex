\documentclass[letterpaper, landscape, 10pt, twocolumn]{article}

\usepackage[letterpaper,margin=1cm]{geometry}
\usepackage{graphicx}
\usepackage{xcolor}
\usepackage{amsmath,amssymb,latexsym}

\graphicspath{ {images/}}
\setlength{\columnsep}{2cm}

\begin{document}
  \thispagestyle{empty}
  \noindent
  {\color{red} IV.} Dómini Nóstri Iésu Chrísti \textbf{Crucisbaiulatiónem}. {\color{red} \textit{(Mc 15, 20-22)\\
  (Fruto: Paciencia en las tribulaciones.)}}\\
  {\color{red} V.} Dómini Nóstri Iésu Chrísti \textbf{Crucifixiónem} et mortem. {\color{red} \textit{(Lc 23, 33-34/Jn 19, 26-34) (Fruto: La Perseverancia Final y entregarnos a la obra de la Redención.)}}\\

  \Large {\color{red} Mystéria Gloriósa}\\
  \normalsize {\color{red} \textit{(in Dominica, in feria quarta)}}\\
  {\color{red} I.} Dómini Nóstri Iésu Chrísti \textbf{Resurrectiónem}. {\color{red} \textit{(Jn 20, 11-28)}}\\
  {\color{red} \textit{(Fruto: La fe y la conversión.)}}\\
  {\color{red} II.} Dómini Nóstri Iésu Chrísti in cælum \textbf{Ascensiónem}. {\color{red} \textit{(Hch 1, 3-11)}}\\
  {\color{red} \textit{(Fruto: Deseo y ansias de llegar al Cielo.)}}\\
  {\color{red} III.} Spíritus Sáncti descensiónem ad \textbf{Pentecosten}. {\color{red} \textit{(Hch 1, 3-11; 2, 1-4)}}\\
  {\color{red} \textit{(Frutos: Los dones del Espíritu Santo y la caridad.)}}\\
  {\color{red} IV.} Beátæ Maríæ Vírginis corporis et animae in cælum \textbf{Assumptiónem}. {\color{red} \textit{(Lc 1, 46-47)}} {\color{red} \textit{(Fruto: La gracia de una buena muerte.)}}\\
  {\color{red} V.} Beátæ Maríæ Vírginis Reginae et Dominae totius creationis
  \textbf{Coronatiónem}. {\color{red} \textit{(Ap 12, 1-5)}} {\color{red} \textit{(Fruto: La verdadera devoción a la Santísima Virgen y su auxilio en las tentaciones, especialmente, en la hora de la muerte.)}}\\

  \Large {\color{red} Mysteria Luminosa}\\
  \normalsize {\color{red} \textit{(in feria quarta)}}\\
  {\color{red} I.} Dómini Nóstri Iésu Chrísti \textbf{Baptismus} in Iordane. {\color{red} \textit{(Lc 3, 15-16. 21-22)}}\\
  {\color{red} II.} Dómini Nóstri Iésu Chrísti \textbf{Revelatio} in the nuptiae Canae. {\color{red} \textit{(Jn 2, 1-11)}}\\
  {\color{red} III.} Regni Dei \textbf{Annuntiatio} ad conversionem invitans. {\color{red} \textit{(Mc 1, 14-15)}}\\
  {\color{red} IV.} Dómini Nóstri Iésu Chrísti \textbf{Transfiguratio} in monte Thabor.
  {\color{red} \textit{(Mc 9, 2-9)}}\\
  {\color{red} V.} Dómini Nóstri Iésu Chrísti de novissima cena et \textbf{Institutio Eucharistiae}. {\color{red} \textit{(1 Cor 11, 23-28)}}\\

  \Large {\color{red} Salve Regina}\\
  \normalsize {\color{red} S}alve, Regína, Mater misericórdiae, vita, dulcédo et spes nostra, salve. Ad Te clamámus, éxsules, fílii Evae. Ad Te suspirámus, geméntes et flentes, in hac lacrimárum valle. Eia ergo, Advocáta Nostra, illos tuos misericórdes Óculos ad nos convérte. Et Jesum, Benedíctum fructum ventris tui, nobis, post hoc exsílium osténde. O clemens! O pia! O dulcis Virgo María! {\color{red} O}ra pro nobis, Sancta Dei Génitrix. {\color{red} U}t digni efficiámur promissiónibus Christi. {\color{red} A}men.\\

  \Large {\color{red} Concede Nos}\\
  \normalsize {\color{red} \textit{Antífona I}}\\
  {\color{red} C}oncede nos, fámulos tuos quæsumus Dómine Deus, perpetua mentis et córporis sanitáte gaudére, et gloriósa beatæ Maríæ semper Vírginis intercessione, a præsenti liberári tristitia, et æterna pérfrui lætitia. Per Christum Dóminum nostrum. {\color{red} A}men.\\
  \vfill

  \begin{center}
    \Huge Corona Rosarum\\
    \large {\color{red} \textit{Sanctum Rosarium Virginis Mariae}}

    \vspace{1cm}
    \includegraphics{rosa2}
    \vspace{1cm}

    \Large Como rezar el Rosario.
  \end{center}

  \noindent
  \normalsize 1. Ofrecimiento.\\
  2. Persignarse.\\
  3. Acto de contrición.\\
  {\color{red}\textit{El rosario esta centrado en el Crucifijo, que habre y cierra el proceso mismo de la oración.}}\\
  4. Credo.\\
  5. Rezar 1 Padre nuestro.\\
  6. Rezar las 3 Ave Maria dedicadas a la Santisima Trinidad.\\
  7. Rezar 1 Gloria al Padre.\\
  6. Rezar los misterios correspondientes. {\color{red} \textit{(Se reza lo siguiente por cada misterio.)}}\\
  .\hspace{0.75cm} 5.1. Anunciar cada misterio.\\
  .\hspace{0.75cm} 5.2. Rezar un 1 Padre nuestro.\\
  .\hspace{0.75cm} 5.3. Rezar 10 Dios te Salve.\\
  .\hspace{0.75cm} 5.4. Rezar 1 Gloria al Padre.\\
  .\hspace{0.75cm} 5.5. Rezar la Jaculatoria.\\
  7. Rezar las Letanías Lauretanas.\\
  8. Rezar La Salve.\\
  9. Rezar una de las Antífonas del Rosario.\\

  \Large {\color{red} Ofrecimiento}\\
  \normalsize {\color{red} O}frezco{\color{red} \textit{(Ofrecemos)}} el rezo de este Santo Rosario por:... {\color{red} \textit{(intención general o particular)}}\\
  \clearpage

  \thispagestyle{empty}
  \Large {\color{red} Per signum Crucis}\\
  \normalsize {\color{red} P}er signum Sanctae Crucis {\color{red} \large \maltese} de inimícis nostris {\color{red} \large \maltese} líbera nos, Deus noster {\color{red} \large \maltese}. {\color{red} I}n nómine Patris, et Fílii, {\color{red} \large \maltese} et Spíritus Sancti.{\color{red} A}men.\\

  \Large {\color{red} Oh mi Dómine Iesu}\\
  \normalsize {\color{red} O}h mi Dómine Iesu, verus Deus et Homo verus, Creator, Pater et Redemptor meus, in qui credo et spero, et quem super omnia diligo, me pœnitent propter peccata mea, quia tu Deus bonus est, ac me pœnis inferni punire potest et tua gratia adiuvante futuris polliceor. {\color{red} A}mén.\\

  \Large {\color{red} Credo}\\
  \normalsize {\color{red} \textit{ Sýmbolum Apostolórum (Mientras se sostiene el Crucifijo en la mano.)}}\\
  {\color{red} C}redo in Deum, Patrem omnipoténtem, Creatórem cœli et terræ. Et in Jesum Christum, Fílium ejus únicum, Dóminum nostrum, qui cocéptus est de Spíritu Sancto, natus ex María Vírgine, passus sub Póntio Piláto; crucifíxus, mórtuus, et sepúltus: Descéndit ad Inféros; tértia die resurréxit a mórtuis: ascéndit ad cœlos, sedet ad déxteram Dei Patris omnipoténtis; inde ventúrus est judicáre vivos et mórtuos. Credo in Spíritum Sanctum, Sanctam Ecclésiam Cathólicam, Sanctórum Communiónem, remissiónem peccatórum, carnis resurrectiónem, vitam ætérnam. {\color{red} A}men.\\

  \Large {\color{red} Las 3 Ave Maria dedicadas a la Santisima Trinidad}\\
  \normalsize {\color{red} \textit{Deus Pater}}\\
  {\color{red} A}ve Sanctíssima Maria, æterni Patris Filia, Virgo purissima ante partum, in manus tuas commendo fidem meam illuminandam. Gratia plena; Dóminus tecum; benedícta Tu in muliéribus, et benedíctus fructus ventris tui, Jesus. {\color{red} S}ancta María, Mater Dei, ora pro nobis, peccatóribus, nunc et in hora mortis nostrae. {\color{red} A}men.\\

  \noindent {\color{red} \textit{Deus Filius}}\\
  {\color{red} A}ve Sanctíssima Maria, Filii Dei Mater, Virgo purissima in partum, in manus tuas commendo spem meam erigendam. Gratia plena; Dóminus tecum; benedícta Tu in muliéribus, et benedíctus fructus ventris tui, Jesus. {\color{red} S}ancta María, Mater Dei, ora pro nobis, peccatóribus, nunc et in hora mortis nostrae. {\color{red} A}men.\\

  \noindent {\color{red} \textit{Deus Spiritus Sanctus}}\\
  {\color{red} A}ve Sanctíssima Maria, Spiritus Sancti Sponsa, Virgo purissima post partum, in manus tuas commendo caritate meam inflamandam. Gratia plena; Dóminus tecum; benedícta Tu in muliéribus, et benedíctus fructus ventris tui, Jesus. {\color{red} S}ancta María, Mater Dei, ora pro nobis, peccatóribus, nunc et in hora mortis nostrae. {\color{red} A}men.\\

  \Large {\color{red} Pater Noster}\\
  \normalsize {\color{red} P}ater noster, qui es in caelis. Sanctificétur Nomen tuum. Advéniat regnum tuum. Fiat volúntas tua, sicut in caelo et in terra.\\
  {\color{red} P}anem nostrum quotidiánum da nobis hódie, et dimítte nobis débita nostra, sicut et nos dimíttimus debitóribus nostris. Et ne nos indúcas in tentatiónem. Sed líbera nos a malo. {\color{red} A}men.\\

  \Large {\color{red} Ave Maria}\\
  \normalsize {\color{red} A}ve María, grátia plena; Dóminus tecum; benedícta Tu in muliéribus, et benedíctus fructus ventris tui, Jesus. {\color{red} S}ancta María, Mater Dei, ora pro nobis, peccatóribus, nunc et in hora mortis nostrae. {\color{red} A}men.\\

  \Large {\color{red} Glória Patri}\\
  \normalsize {\color{red} G}lória Patri, et Fílio, et Spirítui Sancto. {\color{red} S}icut erat in princípio, et nunc, et semper, et in saécula saeculórum. {\color{red} A}men.\\

  \Large {\color{red}Jaculatorias}\\
  \normalsize {\color{red} \textit{María Madre de gracia}}\\
  {\color{red} M}aria Mater gratiæ, Mater misericordiæ. {\color{red} T}u nos ab hoste protege et hora mortis suscipe.\\

  \noindent {\color{red} \textit{Oración de Fátima}}\\
  {\color{red} O}, mi Jesu! Dimítte nobis; líbera nos ab ígne Inférni. Alléva Ánimas Purgatórii, præsértim illas quæ maxíme relíctæ sunt. {\color{red}A}men.\\

  \Large {\color{red} Mystéria Gaudiósa}\\
  \normalsize {\color{red} \textit{(in feria secunda et in sabbato)}}\\
  {\color{red} I.} Beátæ Maríæ Vírginis \textbf{Anuntiatiónem}, et Incarnatione Filii Dei. {\color{red} \textit{(Lc 1, 26-38) (Fruto: Hacer la Voluntad de Dios.)}}\\
  {\color{red} II.} Beátæ Maríæ Vírginis \textbf{Visitatiónem} ad Sanctam Elisabeth. {\color{red} \textit{(Lc 1, 39-56)\\
  (Fruto: La caridad y el amor al prójimo.)}}\\
  {\color{red} III.} Dómini Nóstri Iésu Chrísti \textbf{Nativitátem} in praesepe. {\color{red} \textit{(Lc 2, 7-12)\\
  (Fruto: El desprendimiento de las riquezas.)}}\\
  {\color{red} IV.} Dómini Nóstri Iésu Chrísti \textbf{Presentatiónem} in templo et Beátæ Maríæ Vírginis Purificationis. {\color{red} \textit{(Lc 2, 22-30) (Fruto: La obediencia y la pureza.)}}\\
  {\color{red} V.} Dómini Nóstri Iésu Chrísti \textbf{Inventiónem} in templo. {\color{red} \textit{(Lc 2, 41-52)\\
  (Fruto: Buscar a Dios en todas las cosas.)}}\\

  \Large {\color{red} Mystéria Dolorósa}\\
  \normalsize {\color{red} \textit{(in feria tertia et feria sexta)}}\\
  {\color{red} I.} Dómini Nóstri Iésu Chrísti \textbf{Agonia} in horto Oliveti. {\color{red} \textit{(Mc 14, 32-45/Lc 22, 44) (Fruto: Dolor por haber pecado, contrición por nuestros pecados y propósito de enmienda.)}}\\
  {\color{red} II.} Dómini Nóstri Iésu Chrísti \textbf{Flagellatiónem}. {\color{red} \textit{(Jn 18, 38; 19, 1)\\
  (Fruto: Mortificación de la carne y de los sentidos.)}}\\
  {\color{red} III.} Dómini Nóstri Iésu Chrísti \textbf{Spiniscoronationem}. {\color{red} \textit{(Mt 27, 27-31)\\
  (Fruto: Quitarnos la soberbia y el orgullo.)}}\\
\end{document}
